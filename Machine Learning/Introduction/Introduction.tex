\chapter{Introduction}

\section{General Concern}
\subsection{Types of Learning}

\subsubsection{Supervised Learning}
\begin{itemize}
\item Overview
	\begin{itemize}
	\item training data comprises examples of input vectors with corresponding target vectors
	\end{itemize}
\item Regression
	\begin{itemize}
	\item output one or more continuous variable
	\end{itemize}
\item Classification
	\begin{itemize}
	\item assign input to one of a finite number of discrete categories
	\end{itemize}
\end{itemize}

\subsubsection{Unsupervised Learning}
\begin{itemize}
\item Overview
	\begin{itemize}
	\item training data consists of a set of input
	vectors without target vectors
	\end{itemize}
\item Clustering
	\begin{itemize}
	\item Goal: discover groups of similar examples
	\end{itemize}
\item Density Estimation
	\begin{itemize}
	\item Goal: determine the distribution of data within the input space
	\end{itemize}
\item Dimension Reduction
	\begin{itemize}
	\item Goal: project data into low dimension, for the purpose of such as visualization
	\end{itemize}
\end{itemize}

\subsubsection{Reinforcement Learning}
\begin{itemize}
\item Overview
	\begin{itemize}
	\item input with time series \& discover optimal output by a process of trial and error
	\end{itemize}
\item Goal
	\begin{itemize}
	\item find actions to take under given circumstance to maximize a reward
	\end{itemize}
\end{itemize}

%\subsection{Uncertainty}
%\subsubsection{Cause}
%\begin{itemize}
%
%\end{itemize}

%%%%%%%%%%%%%%%%%%%%%%%%%%%%%%%%%%%%%%%%%%%%%%%%%%%%%%%%%%%%%%%%%%%%%%%%%%%%%%%%%%%%%%%%%%%%%%%%%%%%%%%%%%%%%%%%%%%%%%%%%%%%%%%%
\section{Decision Theory}
\subsection{Overview}
\subsubsection{Formulation}
\begin{itemize}
\item Analysis Target
	\begin{itemize}
	\item Human Behavior
		\begin{itemize}
		\item goal-directed behaviors in the presence of options (alternative actions)
		\end{itemize}
	\end{itemize}
\item Goal
	\begin{itemize}
	\item Normative Decision Theory
		\begin{itemize}
		\item how the decision should be made, in order to be national, etc. \\
		$\Rightarrow$
		\item an normative theory is \textit{weakly falsified}, if $\exists$ a decision problem in which an agent can perform in contradiction with the theory without being irrational
		\item an normative theory is \textit{strictly falsified}, if $\exists$ a decision problem in which an agent performing in accordance with the theory cannot be a rational agent
		\end{itemize}
	\item Descriptive Decision Theory
		\begin{itemize}
		\item how decisions are actually made in practice
		\end{itemize}
	\item Specific Concern
		\begin{itemize}
		\item interaction of agents: collective decision-making
		\item behavior of individuals in decisions
		\item rationality in decisions
		\item coordinating decision over time, in a changing environment
		\end{itemize}
	\end{itemize}
\item Preference in Decision Making
	\begin{itemize}
	\item Preference Logic
		\begin{itemize}
		\item defined relation $\ge$ for two options $a,b$, according to agent's preference \\
		$\Rightarrow$ form a partially ordered set $(S, \ge)$, where $S$ the set of options
		\item other relation $<, \equiv$ can be derived from $\ge$
		\end{itemize}
	\item Completeness
		\begin{itemize}
		\item complete: $\forall i,j\in X, $ a relation is defined \\
		(otherwise incomplete preference)
		\item yet, usually incomplete preference, as determining preference takes effort
		\end{itemize}
	\item Transitivity
		\begin{itemize}
		\item $a\ge b \land b\ge c \rightarrow a\ge c$ holds true for all $a,b,c$ \\
		(hence all other relation $<,\equiv$)
		\item $\Rightarrow$ problem of accumulated negligible indifference ($\equiv$) \\
		e.g. $a_0$ as no sugar, $a_{100}$ as full sugar, given $a_{i}\equiv a_{i+1}$ due to negligible difference \\
		$\Rightarrow$ finally $a_0\equiv a_{100}$ due to transitivity
		\end{itemize}
	\item Guideline
		\begin{itemize}
		\item given the constructed partial order set $(S, \ge)$, choose one of its upper bound \\
		(with prevalent assumption of transitivity and completeness)
		\end{itemize}
	\end{itemize}
\item Utility in Decision Making
	\begin{itemize}
	\item Numerical Representation
		\begin{itemize}
		\item using pre-defined relation in number, with a preference of maximal utility \\
		$\Rightarrow$ naturally transitive \& complete
		\item relative number $\Rightarrow$ meaningful in comparison within current option set \\
		(i.e. not comparable across different decision making process)
		\end{itemize} 
	\item \textbf{Comparability}
		\begin{itemize}
		\item only comparable between options in the same decision process
		\end{itemize}
	\item Guideline
		\begin{itemize}
		\item maximization: choose one of the options with maximal utility \\ 
		(employed by default)
		\item satisfying: choose one of the options with sufficient utility
		\end{itemize}
	\end{itemize}
\item Decision Matrices
	\begin{itemize}
	\item $\text{Outcome } O_{n\times m}=\text{actions }\{a_1,...,a_n\} \times\text{states }\{s_1,...,s_m\}$
		\begin{itemize}
		\item with $\{a_1,a_2\} =$ \{umbrella, no umbrella\}; $\{s_1,s_2\}=$ \{rain, no rain\} \\
		\begin{tabular*}{0.5\linewidth}{l|ll}
			            & rain       & no rain    \\
			\hline
			umbrella    & dry\&heavy & dry\&heavy \\
			no umbrella & wet\&light & dry\&light
		\end{tabular*}
		\end{itemize}
	\item Utility Assignment
		\begin{itemize}
		\item requirement: shared perspective \\
		$\Rightarrow$ all participants share a common concern \\
		(otherwise, different incomparable utility assigned to the same outcome)
		\item assign each outcomes with utility $\Rightarrow$ utility matrices $U$ \\
		\begin{tabular*}{0.5\linewidth}{l|ll}
			            & rain       & no rain    \\
			\hline
			umbrella    & 15 & 15 \\
			no umbrella & 0  & 18
		\end{tabular*}
		\end{itemize}
	\end{itemize}
\item Probability Estimation of Environment
	\begin{itemize}
	\item Modeling State
		\begin{itemize}
		\item certainty: deterministic knowledge of the environment after each action \\
		$\Rightarrow$ thus deterministic outcomes for each action
		\item risk: complete probabilistic knowledge of environment \\
		(may conditioned on action)
		\item uncertainty: partial probabilistic knowledge
		\item ignorance: no probabilistic knowledge, or not meaningful
		\end{itemize}
	\item Objective Probability
		\begin{itemize}
		\item based on empirical known frequencies
		\item indirect estimation (e.g. similar experiment), with calibration \\
		may use subjective probability, which is unreliable due to psycho-effect \\
		$\Rightarrow$ better not use subjective estimates of prob
		\end{itemize}
	\item Bayesian Description
		\begin{itemize}
		\item a coherent and complete set of probabilistic beliefs \\
		(in compliance with laws of probability)
		\item probability subjected to the observation
		\end{itemize}		
	\end{itemize}
\item Action Discovering
	\begin{itemize}
	\item Category
		\begin{itemize}
		\item closure: settle down with current available actions
		\item active postponement: searching for other possible actions (postpone the decision)
		\item semi-closure: consider only reversible actions \& searching for other actions
		\end{itemize}
	\item Guideline
		\begin{itemize}
		\item all current actions have severe drawbacks?
		\item is searching possible actions costly?
		\item is problem aggregating over time? \\
		etc...
		\end{itemize}
	\item Understanding
		\begin{itemize}
		\item a meta-decision problem before the scenario-specific decision
		\end{itemize}
	\end{itemize}
\item $\Rightarrow$ Assumption
	\begin{itemize}
	\item Closed Set of Actions
		\begin{itemize}
		\item no new alternative action allowed to be added \\
		(v.s. open: new actions can be discovered and taken into consideration)
		\end{itemize}
	\item Mutually Exclusive Actions
		\begin{itemize}
		\item no two actions can be both realized
		\end{itemize}
	\item Defined States
		\begin{itemize}
		\item all possible states of environment are recognized
		\end{itemize}
	\item Outcome \& Utility
		\begin{itemize}
		\item the joint result of environment (state) and chosen action
		\item utility assigned to each outcome (similar to rewards in RL) \\
		(though the utility for each result is usually subject to agent) \\
		$\Rightarrow$ completeness \& transitivity assumed		
		\end{itemize}
	\end{itemize}
\end{itemize}
\subsubsection{Decision under Risk}
\begin{itemize}
\item Maximizing Expected Utility
	\begin{itemize}
	\item Procedure
		\begin{itemize}
		\item for action $a$, expected utility $\displaystyle u = \mathbb E_{p(s)} U(a,s) = \sum_{s\in S} p(s)U(a,s)$, \\ 
		where $U(a,s)$ the utility assigned to outcome $O(a,s)$
		\end{itemize}
	\item Objective Utility
		\begin{itemize}
		\item utility assigned according to the objective rewards (e.g. money)
		\item $\Rightarrow$ St, Petersburg paradox: a fair coin tossed, if its head comes up at $n^{th}$ toss, $2^n$ gold coins are rewarded, stop tossing if its back comes up \\
		$\Rightarrow$ expected wealth $=\lim_{n\rightarrow \infty} \frac 12\cdot 1 + \frac 14\cdot 2 + ... + \frac 1{2^n}\cdot 2^{n-1} = +\infty$ \\
		$\Rightarrow$ should play the game for any finite entry fee
		\end{itemize}
	\item Subjective Utility
		\begin{itemize}
		\item in economic: utility of next increment of wealth $\propto \frac 1w$, where $w$ the current wealth \\
		$\Rightarrow$ utility of wealth $=\log(w)$
		\end{itemize}
	\item Understanding
		\begin{itemize}
		\item objective utility to maintain inter-subjective validity \\
		(hence expert advice remains effective)
		\item maximizing utility to maximize long-term outcome \\ 
		(suitable for large-scale / long-term decision) \\
		$\Rightarrow$ law of large number to level out randomness
		\item $\Rightarrow$ suitability depends on the scale (the leveling-out effect) \\ 
		(e.g. isolated cases may not suitable to maximize expected utility)
		\item may not suitable in extreme effect, to maintain fairness \\
		(e.g. avoid imposing high-prob risk on individuals)
		\end{itemize}
	\end{itemize}
\item Variations
	\begin{itemize}
	\item Conditionalized Expected Utility
		\begin{itemize}
		\item prob of state $p(s)$ described conditional on action \\
		$\displaystyle \Rightarrow u = \mathbb E_{p(s|a)} U(a,s) = \sum_{s\in S} p(s|a)U(a,s)$
		\item model the influence of actions on states
		\end{itemize}
	\item Generalized Expected Utility
		\begin{itemize}
		\item process utility: accounts for attitude towards risk and certainty
		\item maximize expected (utility of outcome + process utility)
		\end{itemize}
	\item Regret Theory
		\begin{itemize}
		\item regret: the gap between reward received and highest possible reward from other actions
		\item maximize expected utility of outcome \& minimize regret
		\end{itemize}
	\item Prospect Theory
		\begin{itemize}
		\item function f$(0,1)\rightarrow(0,1)$ to adjust probability \\
		(description to be merely weights, not satisfying prob law)
		$\Rightarrow$ overweight the prob close to $0\&1$, as tendentious to certainty
		\item maximize weighted utility
		\end{itemize}
	\end{itemize}
\item Causal Decision
\end{itemize}
\subsubsection{Decision under Uncertainty}
\begin{itemize}
\item Estimation of Environment
	\begin{itemize}
	\item Binary Measure
		\begin{itemize}
		\item segment out a range for probability describing the state \\
		e.g. $p(s)\in(0.05, 0.2)$
		\end{itemize}
	\item Second-Order Probability
		\begin{itemize}
		\item a Bayesian probability for each state $s$
		\item a further measurement of the reliability of probability estimates $p(s)$ \\
		(may consider as a meta subjective prob)
		\end{itemize}
	\item Fuzzy Set Membership
		\begin{itemize}
		\item vagueness, instead of randomness, to describe uncertainty \\
		$\Rightarrow$ assign a degree of membership for each $p(s)$ \\
		$\Rightarrow$ based on laws of fuzzy membership (non-statistical concept)
		\item measure the degree of "$p(s)$ is the prob of state $s$ happening" being true
		\end{itemize}
	\item Epistemic Reliability
		\begin{itemize}
		\item a weight $\in (0,1)$ as the measure of reliability of a prob, \\
		with NO mathematical properties pre-defined
		\end{itemize}
	\end{itemize}
\item Decision Criteria
	\begin{itemize}
	\item Maxmin Expected Utility
		\begin{itemize}
		\item maximize the minimal expected utility \\
		$\Rightarrow$ an extremely prudent/pessimistic criterion
		\end{itemize}
	\item Reliability-weighted Expected Utility
		\begin{itemize}
		\item use the weighted average probability as true probability \\
		(then reduced to maximizing expected utility) \\
		$\Rightarrow$ an optimistic criterion
		\end{itemize}
	\item Index
		\begin{itemize}
		\item $p'$ the best prob estimate and $\rho$ as its degree of confidence \\
		$\Rightarrow$ calculate $u_\text{best}$ using $p'$; $u_\text{min}$ the minimal expected utility
		\item summarize as $u = \rho u_\text{best} + (1-\rho)u_\text{min}$ \\
		$\Rightarrow$ maximize the $u$ (choose the action with maximal $u$)
		\item $\rho\in(0,1)$ as a balance factor between pessimism \& optimism 
		\end{itemize}
	\item Clipped Maxmin Expected Utility
		\begin{itemize}
		\item discard any prob with a reliability lower than a threshold \\
		(then max-min expected utility with remaining prob)
		\end{itemize}
	\item Filtered Maxmin Expected Utility
		\begin{itemize}
		\item $3$ filters applied in sequential order on actions (rather than prob)
		\item $E$-test: action $a$ passes, iff $\exists$ reliable prob $p(s)>0$ \& $\forall a'\in A, U(a,s)\ge U(a',s)$ \\
		(action $a$ is possible to be the best)
		\item $P$-test: passes, iff it can be the best in all candidates
		\item $S$-test: passes, iff it is the best under maxmin expected utility in all candidates \\
		$\Rightarrow$ directly produce action to choose
		\end{itemize}
	\end{itemize}
\end{itemize}
\subsubsection{Decision under Ignorance}
\begin{itemize}
\item Category
	\begin{itemize}
	\item Known States
		\begin{itemize}
		\item the prob for each state $p(s) > 0 \Rightarrow$ unknown non-zero prob
		\end{itemize}
	\item Unknown States
		\begin{itemize}
		\item the prob $p(s) \ge 0 \Rightarrow$ unknown prob
		\end{itemize}
	\end{itemize}
\item Unknown Non-zero Prob
	\begin{itemize}
	\item Max-Min Utility
		\begin{itemize}
		\item maximize the minimal utility (extremely pessimistic)
		\item lexicographic maximin: consider their $2^\text{nd}$ worst outcome
		\end{itemize}
	\item Max-Max Utility
		\begin{itemize}
		\item maximize the maximal utility (extremely optimistic)
		\end{itemize}
	\item Index
		\begin{itemize}
		\item calculate $u_\text{max}, u_\text{min}$ by max-max, max-min
		\item $u = \alpha u_\text{} + (1-\alpha) u_\text{min}$ \\
		$\Rightarrow \alpha$ as tge degree of optimism
		\end{itemize}
	\item Min-Max Regret
		\begin{itemize}
		\item produce a regret matrix $R$ from utility matrix $U$, \\ 
		where $R(a,s) = $ maximal possible utility in state $s - U(a,s)$
		\item then, minimize the maximal regret, according to $R$
		\end{itemize}
	\item Common Prior
		\begin{itemize}
		\item assign uniform distribution over states, thus reduced to decision under risk
		\item yet, depends on how the states are partitioned \\ 
		i.e. the num of states affect the prob
		\end{itemize}
	\end{itemize}
\item Unknown Prob
	\begin{itemize}
	\item Challenge
		\begin{itemize}
		\item action may lead to catastrophic outcomes (unforeseen outcomes) \\
		$\Rightarrow$ hard to decide if the chance of a severe consequence is negligible or not \\
		$\Rightarrow$ uncertainty towards severe consequence
		\end{itemize}
	\item Empirical Guideline
		\begin{itemize}
		\item choosing / not choosing an action may have different uncertainty \\
		(asymmetry of uncertainty)
		\item novelty: empirically, novelty brings in more uncertainty
		\item spatial and temporal limitations: more limitation, less uncertainty
		\item more interference with complex system, the more uncertainty
		\end{itemize}
	\end{itemize}
\end{itemize}
\subsubsection{Social Decision}
\begin{itemize}
\item Assumption
	\begin{itemize}
	\item Conflicting Concern
		\begin{itemize}
		\item participants usually have different/conflicting goals\&concerns
		\end{itemize}
	\end{itemize}
\item Cyclic Preference
	\begin{itemize}
	\item No Stable Actions
		\begin{itemize}
		\item the directed set (actions, preference) forms a directed graph \\
		$\Rightarrow \forall$ action $a$, $\exists $ action $a', a' > a$
		\end{itemize}
	\item Arrow's Theorem
		\begin{itemize}
		\item 4 rationality criteria are satisfied by decision rule \\
		$\Rightarrow$ then cyclic preference unavoidable
		\item understanding: some rationality demands are NOT compatible
		\end{itemize}
	\end{itemize}
\end{itemize}

\subsubsection{Decision for Statistical Model}
\begin{itemize}
\item Formulation
	\begin{itemize}
	\item Probability Description
		\begin{itemize}
		\item known distribution $p(\mathbf t|\mathbf x)$ from model inference, where $\mathbf x$ the input, $\mathbf t$ the label \\
		$\Rightarrow$ all states (label) estimated with probability
		\end{itemize}		
	\end{itemize}
\item Decision Criteria
	\begin{itemize}
	\item Minimizing Misclassification Rate (Maximizing Correct-classification Rate)
		\begin{itemize}
		\item action: for each class $k$, assign a region $\mathcal R_k$ s.t. if $\mathbf x\in \mathcal R_k$, predict $\hat {\mathbf t} := \mathcal C_k$
		\begin{align*}\Rightarrow \displaystyle P(\text{mistake}) &= \sum_{k=1}^K P(\mathbf x\not\in \mathcal R_k, \mathbf t=\mathcal C_k) = 1-\underbrace{\sum_{k=1}^K P(\mathbf x\in\mathcal R_k, \mathbf t=\mathcal C_k)}_{P(\text{correct})} \\ &= 1-\sum_k\int_{\mathcal R_k}p(\mathcal C_k|\mathbf x) p(\mathbf x) d\mathbf x \end{align*}
		$\Rightarrow$ segment $\mathbf x$ to be in region $\mathcal R_k$ governed by the maximal $p(\mathcal C_k | \mathbf x)$ \\
		i.e. predict $\hat {\mathbf t}$ to be the $\mathcal C_k$ with maximal $p(\mathcal C_k | \mathbf x)$
		\item in binary classification ($K=2$) with $\mathbf x = x$ being a scalar: \\
		\includegraphics[width=0.5\linewidth, center]{"./Introduction/decision theory-decision region".jpg}
		where $\hat x$ a sub-optimal decision region segment, $x_0$ the optimal one
		\end{itemize}
	\item Minimizing Expected Loss (Maximizing Expected Utility)
		\begin{itemize}
		\item loss for a kind of misclassification can vary from other kinds \\
		(e.g. recall v.s. precision) \\
		$\Rightarrow$ a loss function (utility function) $f:$ (label $\mathbf t=\mathcal C_k$, pred ${\mathbf x}\in\mathcal R_j$) $\rightarrow $ loss $L_{kj}$ \\
		$\Rightarrow L_{kj}$ as the loss of mis-classify $\mathcal C_k$ into $\mathcal C_j$ \\
		($\forall k, L_{kk}=0$ \& utility as negative loss)
		\item expected loss $\displaystyle \mathbb E[L] = \sum_{k,j}L_{kj}P(\mathbf x\in \mathcal R_{j}, \mathbf t=\mathcal C_k) = \sum_{k,j}\int_{\mathcal R_j}L_{kj} p(\mathcal C_k|\mathbf x)p(\mathbf x)d\mathbf x$ \\
		$\Rightarrow$ expected loss from $\displaystyle \mathcal R_j: \sum_k \int_{\mathcal R_j}L_{kj} p(\mathcal C_k|\mathbf x)p(\mathbf x) d\mathbf x$ \\
		$\Rightarrow$ segment $\mathbf x$ into region $\mathcal R_j$ governed by the minimal $\displaystyle \sum_{k}L_{kj}p(\mathcal C_k|\mathbf x)$ \\
		i.e. predict $\hat{\mathbf t}$ to be the $\mathcal C_j$ with minimal $\displaystyle \sum_{k}L_{kj}p(\mathcal C_k|\mathbf x)$
		\end{itemize}
	\end{itemize}
\item Regret Option
	\begin{itemize}
	\item Decision Error Source
		\begin{itemize}
		\item the largest posterior $p(\mathcal C_k|\mathbf x) << 1$ \\
		$\Rightarrow$ large $p(\mathbf t \neq \mathcal C_k)$ on region $\mathcal R_k$
		\item $\Leftrightarrow$ all posteriors are comparable
		\end{itemize}
	\item Reject by Thresholding
		\begin{itemize}
		\item reject to make decision when the maximal posterior $p(\mathcal C_k|\mathbf x) <$ threshold $\theta$ \\
		($\theta=frac 1 k$ to not reject any example)
		\end{itemize}
	\item Reject Criterion to Minimize Expected Loss
		\begin{itemize}
		\item consider the expected loss instead of raw posterior
		\end{itemize}
	\end{itemize}
\end{itemize}

\section{Information Theory}
\subsection{}


%%%%%%%%%%%%%%%%%%%%%%%%%%%%%%%%%%%%%%%%%%%%%%%%%%%%%%%%%%%%%%%%%%%%%%%%%%%%%%%%%%%%%%%%%%%%%%%%%%%%%%%%%%%%%%%%%%%%%%%%%%%%%%%%
\section{Recommended Practice}

\subsection{Data}
\subsubsection{Data Augmentation}
\begin{itemize}
\item Artificial Data Synthesis
	\begin{itemize}
	\item Practice
		\begin{itemize}
		\item easily prepare a large amount of similar (yet, different) data
		\item add canonical noise to the similar data
		\end{itemize}
	\item Understanding
		\begin{itemize}
		\item convert similar data to the desired distribution
		\end{itemize}
	\item Caution
		\begin{itemize}
		\item collected canonical noise may only represent a subset of all possible noise \\
		$\Rightarrow$ may overfit to those collected noise \\
		(e.g. distortion on image from game for car detection: too less unique cars)
		\end{itemize}
	\item Examples
		\begin{itemize}
		\item in the task of classifying image uploaded by users \\ 
		$\Rightarrow$ collect web image \& blur the image
		\item in the task of in-car NLP interaction \\
		$\Rightarrow$ collect well-recorded sentence audio \& add in-car noise
		\end{itemize}
	\end{itemize}
	
\item Distortion
	\begin{itemize}
	\item Practice in Computer Vision
		\begin{itemize}
		\item mirroring horizontally/vertically, rotation, shirring, etc.
		\item random crop a reasonably large subset of image
		\item color shifting: e.g. PCA color augmentation
		\end{itemize}
	\item 
	\end{itemize}
\end{itemize}
\subsubsection{Data Preprocessing}
\begin{itemize}
\item Mean Centering
	\begin{itemize}
	\item Practice
		\begin{itemize}
		\item for all training examples, compute mean (on each features) $\displaystyle \mu=\frac 1 N \sum_{n=1}^N{\mathbf x_n}$, \\
		where $\{\mathbf x_1,...,\mathbf x_N\}=X_\text{train}$ the training set
		\item preprocess each $\mathbf x \in X_\text{train}, X_\text{val}, X_\text{test}$ to be $\mathbf x'=\mathbf x -\mu$
		(all data go through the same process)
		\end{itemize}
	\item Pros
		\begin{itemize}
		\item (training) data has a zero mean (statistically, most data close to $0$)
		\end{itemize}
	\item Cons
		\begin{itemize}
		\item different features may reside in various scales
		\end{itemize}
	\end{itemize}
\item Standardizing
	\begin{itemize}
	\item Practice
		\begin{itemize}
		\item compute mean $\mu$, standard deviation $\displaystyle \sigma=\left( \frac 1 N \sum_{n=1}^N{(\mathbf x_n-\mu)^2} \right)^{1/2}$, \\
		where $\{\mathbf x_1,...,\mathbf x_N\}=X_\text{train}$ the training set
		\item preprocess each $\mathbf x \in X_\text{train}, X_\text{val}, X_\text{test}$ to be $\mathbf x'=\frac{\mathbf x -\mu} \sigma$ \\ 
		(all data go through the same process)
		\item note: with big data, usually computed iteratively due to limited memory
		\end{itemize}
	\item Pros
		\begin{itemize}
		\item (training) data has zero mean \& unit variance \\
		$\Rightarrow$ approximated to normal distribution
		\item for deep learning: different features in same small range close to $0$ \\
		$\Rightarrow$ weights for different features are in roughly the same scale \\
		$\Rightarrow$ easier to train
		\end{itemize}
	\end{itemize}

\item Cleaning Incorrect Label
	\begin{itemize}
	\item Practice 
		\begin{itemize}
		\item before cleaning: measure its contribution to the error rate \& its cause
		\item random error (e.g. occasional mistake, etc.) with a big dataset: okay to ignore
		\item systematic error: should be corrected, at least for val\&test set \\
		$\Rightarrow$ to evaluate the model on the target data distribution
		\item if mislabeled data cause inability to evaluate\&compare model: must be cleaned
		\item val\&test set should be cleaned together \\
		$\Rightarrow$ to have the same distribution
		\end{itemize}
	\item Understanding
		\begin{itemize}
		\item in train set: statistic model (deep net etc.) quite robust to random errors \\
		while model can learn the systematic error $\Rightarrow$ not able to generalize
		\item in val set: random error can cause inability  \\ 
		$\Rightarrow$ 
		\end{itemize}
	\end{itemize}
\end{itemize}

\subsection{Dataset}
\subsubsection{Train-Val-Test}
\begin{itemize}
\item Reason
	\begin{itemize}
	\item Iterative Process
		\begin{figure}[!ht]
		\includegraphics[width=0.3\linewidth, center]{"./Deep Learning/experiment practice-iterative process".png}
		\end{figure}
		\begin{itemize}
		\item intuition usually do NOT transfer across domains (NLP, CV, Search, etc.)
		\item do NOT hope to have the correct hyperparameters at the first try
		\end{itemize}
		$\Rightarrow$ need feedback from experiment result \\
		$\Rightarrow$ make sure the feedback is CORRECT and FAST
	\end{itemize}
\item Recommended Usage
	\begin{itemize}
	\item Splitting 
		\begin{itemize}
		\item classic split for small dataset $\Rightarrow$ train:val:test $= 60:20:20$, or K-fold
		\item in big data (e.g. $100$ million) $\Rightarrow$ train:val:test $= 98:1:1$
		\end{itemize}
		(as long as val-test sets cover enough data variance)
	\item Training Set
		\begin{itemize}
		\item to find the model parameter estimation (used for learning process of model) \\
		$\Rightarrow$ over-fit by complex model
		\item can incorporate methods to train the model to have desired property \\
		(where augment data goes)
		\end{itemize}
	\item Validation Set (Val)
		\begin{itemize}
		\item to indicate generalization ability of a range of trained models on target data \\
		(correct if enough various input covered) \\
		$\Rightarrow$ for model comparison, selection \& hyperparameters tunning
		\item should have consistent distribution with test set \\
		(as val set is also evaluating the generalization ability)
		\end{itemize}
	\item Test Set
		\begin{itemize}
		\item to evaluate the \textbf{generalization ability} of final model on target data \\
		(correct if enough various input covered)
		\item should represent the distribution of target data \\
		i.e. data that the deployed model will need to handle
		\end{itemize}
	\item Training-Validation Set (Train-Val)
		\begin{itemize}
		\item another val set split from original training set
		\item used if training set are from different distribution then the val/test set \\
		(e.g. due to augmented data etc.)
		\item performance gap between train\&val set: variance + distribution mismatch \\
		$\Rightarrow$ separate each measurement
		\item performance gap between train\&train-val set: measuring variance \\
		$\Rightarrow$ performance gap between train-val\&val set: measuring distribution mismatch
		\end{itemize}
	\end{itemize}

\item Potential Problem
	\begin{itemize}
	\item Mismatched Distribution across Sets
		\begin{itemize}
		\item classic supervised learning assumption: all sets drawn from SAME distribution \\
		(yet transfer/adaptive learning focus on violation of such assumption)
		\item measured by train-val set
		\item should ensure at least that val\&test set have the SAME distribution
		\end{itemize}
		$\Rightarrow$ yet, make sure val\&test set from the SAME distribution as the desired one
	\item \textbf{Overfitting Val Set}
		\begin{itemize}
		\item iteratively tunning model is a processing of learning (fitting to the val set) \\
		$\Rightarrow$ with enough iteration, val set can be overfit
		\item may consider test set as $2^\text{nd}$ val set, and further have $3^\text{rd}, 4^\text{th}...$ val sets
		\end{itemize}
	\item Limited Data
		\begin{itemize}
		\item better model $\Rightarrow$ more training data 
		\item $\Rightarrow$ less validation $\Rightarrow$ noisy estimation of generalization ability
		\end{itemize}
	\end{itemize}
\end{itemize}

\subsubsection{Train-Test}
\begin{itemize}
\item No Val Set
	\begin{itemize}
	\item Practice
		\begin{itemize}
		\item may use the "test" set as val set $\Rightarrow$ generalization ability NOT reported
		\item should be confident in that dataset cover/represent true distribution of data \\ 
		(yet, not recommended)
		\end{itemize}
	\item Understanding
		\begin{itemize}
		\item to utilize as many data as possible for ultimate performance
		\end{itemize}
	\end{itemize}
	
\item K-fold Cross Validation
	\begin{itemize}
	\item Procedure
		\begin{itemize}
		\item split all data into $K$ folds, $K-1$ folds for train, $1$ for validation
		
		\item $\Rightarrow$ average over all $C^1_K$ combination to indicate the generalization ability
	
		\item extreme case: leave-out-one $\Rightarrow K=N$, where $N$ is number of all data
		\end{itemize}
	
	\item Cons:
		\begin{itemize}
		\item $\mathcal O(K) \Rightarrow$ slow, especially if training process already slow \\
		$\Rightarrow$ trade off between time vs. constraint on validation
		\item hence, \textbf{not} often used in big data era
		\end{itemize}
	\end{itemize}
\end{itemize}

\subsection{Orthogonalization Practice}
\subsubsection{Definition}
\begin{itemize}
\item Decoupling Goals and Models
	\begin{itemize}
	\item Designing Metrics
		\begin{itemize}
		\item to evaluate the models \\ 
		$\Rightarrow$ capture how well the problem solved as desired
		\item decouple different aspect of concern into different metrics
		\end{itemize}
	\item Designing Models
		\begin{itemize}
		\item to do well on the previously chosen metrics \\
		(including training \& tunning hyperparameters)
		\end{itemize}
	\end{itemize}
\item Decoupling Hyperparameters
	\begin{itemize}
	\item disjoint set of hyperparameters to optimize for train-val-test set
	\item hyperparameter taking effect on single goal \\
	at least, NOT to impose negatively related effect on multiple goals \\
	(e.g. early stopping on performances on train and val sets $\Rightarrow$ NOT preferred)
	\item $\Rightarrow$ clearer control on model behaviors
	\end{itemize}
\end{itemize}

\subsubsection{Practice}
\begin{itemize}
\item Designing Metrics (Goals)
	\begin{itemize}
	\item Single Metric Reporting Overall Performance
		\begin{itemize}
		\item a metric accounting for multiple metrics \\
		e.g. F1 score instead of precision and recall
		\item weighted average over metrics \\
		(capturing tendency by different weights)
		\end{itemize}
	\item Satisficing Metrics
		\begin{itemize}
		\item optimizing single metric with some minimum requirement must being satisfied \\ 
		$\Rightarrow$ single optimizing metric + several satisficing metrics \\ 
		$\Rightarrow$ optimizing under constraints \\
		e.g. optimizing accuracy with false positive rate $< 0.2$ satisfied
		\end{itemize}
	\end{itemize}
	
\item Tunning Hyperparameters (Designing Model)
	\begin{itemize}
	\item Inherent Separation for Set-level Goals
		\begin{itemize}
		\item fit model on train set for good fitting \\
		$\Rightarrow$ tune model/network structure, optimization, preprocessing, etc.

		\item evaluate on val set for good generalization ability \\
		$\Rightarrow$ tune regularization, etc.

		\item evaluate on test set for hopefully good generalization ability reported \\
		$\Rightarrow$ consider bigger val set (if an overfit val set indicated)

		\item apply in real world hopping model to generalize well indeed \\
		$\Rightarrow$ consider mismatched data distribution, redesign cost function / metrics etc. \\
		(if failed to generalize)
		\end{itemize}
		note: size of dataset can be hyperparameter sometimes
	\item Separating Tunning for Performance Metrics
		\begin{itemize}
		\item 
		\end{itemize}
	\end{itemize}
\end{itemize}

\subsubsection{Orthogonalization}
\begin{itemize}
\item Definition
	\begin{itemize}
	\item Decoupling Tunning for Different Goals
		\begin{itemize}
		\item disjoint set of hyperparameters to optimize on train-val-test set
		\item hyperparameter taking effect on single goal \\
		(at least, NOT to impose negatively related effect on multiple goals)
		\end{itemize}
	\item Single Metric Reporting Performance
		\begin{itemize}
		\item a metric accounting for multiple metrics \\
		e.g. F1 score instead of precision and recall
		\item optimizing under constraints (must-satisfied metrics) \\
		e.g. optimizing accuracy with false positive rate $< 0.2$ satisfied
		\end{itemize}
	\end{itemize}
\item Practice
	\begin{itemize}
	\item Separating Tuning for Set-level Goals
		\begin{itemize}
		\item for train set: model/network structure, optimization, preprocessing, etc.
		\item for val set:  regularization, etc.
		\item for test set: bigger val set (as indicating an overfit val set)
		\item for real world: mismatched cost function / data distribution, etc.
		\end{itemize}	
	\item Separating Tunning for Performance Metrics
		\begin{itemize}
		\item 
		\end{itemize}
	\end{itemize}
\item Understanding
	\begin{itemize}
	\item Inherently Separated Goals
		\begin{itemize}
		\item fit model on train set: adjust model for good fitting
		\item evaluate on val set: adjust model for good result on metrics \\ 
		(indicating generalization ability)
		\item evaluate on test set: hope to report good generalization ability
		\item apply in real world: hope to generalize well indeed
		\end{itemize}
	\item Clear Control on Behaviors
		\begin{itemize}
		\item form iterative process among various goals
		\item prevent tunning practice with unaware negatively coupled effect on different goals \\
		(e.g. early stopping on performances on train and val sets $\Rightarrow$ NOT preferred)
		\end{itemize}
	\end{itemize}
\end{itemize}

\subsection{Tunning Hyperparameters}
especially for deep learning
\subsubsection{Hyperparameters}
\begin{itemize}
\item Overview
	\begin{itemize}
	\item Structures and Architectures
		\begin{itemize}
		\item type of layers and size of layers
		\item type of activation
		\item depth of networks
		\end{itemize}
	\item Learning
		\begin{itemize}
		\item learning rate
		\item optimizer (learning process)
		\end{itemize}
	\item Robustness and Generalizability
		\begin{itemize}
		\item regularization(s)
		\item data preprocessing/augmentation
		\end{itemize}
	\end{itemize}
\item Challenges
	\begin{itemize}
	\item NO Consistent Prescience
		\begin{itemize}
		\item popular choices from one domain usually NOT carry over to other domains
		\end{itemize}
	\item NOT Predictable Effect
		\begin{itemize}
		\item hyperparameter does NOT have predictable effect on specific model behavior \\
		$\Rightarrow$ need a search for the best one
		\end{itemize}
	\end{itemize}
\end{itemize}

\subsubsection{Systematic Searching}
\begin{itemize}
\item Random Sampling
	\begin{itemize}
	\item Reason
		\begin{itemize}
		\item NOT able to know the importance of different hyperparameters \\
		$\Rightarrow$ not wasting grid search step on the unimportant
		\item NOT able to know the effective range of a hyperparameters \\
		$\Rightarrow$ may be skipped by grid search step
		\item decouple the search for different hyperparameters $\Rightarrow$ more richly explore \\
		(whereas grid search fix one while searching on others)
		\end{itemize}
	\item Coarse to Fine Scheme
		\begin{itemize}
		\item explore whole space uniformly (equally random)
		\item exploit region where good results show up (with more densely sampled)
		\end{itemize}
	\item Sampling on Scale
		\begin{itemize}
		\item instead of sampling the value of hyperparameter, sample the scale of it \\
		e.g. sampling learning rate $\alpha = 10^r, r\sim U(-4,0)$
		\item $\Rightarrow$ distribute the density across desired scale\textbf{s} \\ 
		(by using transfered scale, e.g. applying $\log, e^x$ e.t.c)
		\item reason: depends on the 
			\begin{itemize}
			\item use of hyperparameter e.g. in an exponential/linear/log way
			\item whether intend to sample on scale e.g. across one or more scales
			\end{itemize}
		\end{itemize}
	\end{itemize}

\item Swarm Optimization
	\begin{itemize}
	\item Intuition
		\begin{itemize}
		\item searching over a space with continuous$\times$discrete across various scale \\ 
		$\Rightarrow$ encoded into a list
		\item search using permutation / group behavior \\ 
		$\Rightarrow$ inherently imposing explore-exploit strategy
		\end{itemize}
	\item Popular Framework
		\begin{itemize}
		\item genetic algorithm (GA)
		\item particle swarm optimization (PSO)
		\end{itemize}
	\end{itemize}
\end{itemize}

\subsubsection{Tunning Practice}
\begin{itemize}
\item Single Model
	\begin{itemize}
	\item Practice
		\begin{itemize}
		\item supervising one model at a time
		\item interactively justify the hyperparameter in training process  \\
		$\Rightarrow$ gain knowledge through interaction \& ensure a good performance \\
		$\Rightarrow$ early feedback
		\end{itemize}
	\item Reason
		\begin{itemize}
		\item too many data (online advertisement, computer vision etc.)
		\item few computing resource
		\end{itemize}
	\end{itemize}
\item Parallel Training
	\begin{itemize}
	\item Practice
		\begin{itemize}
		\item shoot out multiple model with various settings
		\item compare at the end (after trained \& evaluated)
		\end{itemize}
	\item Reason
		\begin{itemize}
		\item small data/model, enough computability / fast training process
		\end{itemize}
	\end{itemize}
\end{itemize}
%\subsection{Training Model}
%\subsubsection{Data Feeding}
%\begin{itemize}
%\item Multi-processing/threading
%	\begin{itemize}
%	\item Loading Data
%		\begin{itemize}
%		\item load the data from disk \& wait for io
%		\end{itemize}
%	\item Applying Preprocessing
%		\begin{itemize}
%		\item normalization
%		\item augmentation
%		\end{itemize}
%	\end{itemize}
%\end{itemize}

%%%%%%%%%%%%%%%%%%%%%%%%%%%%%%%%%%%%%%%%%%%%%%%%%%%%%%%%%%%%%%%%%%%%%%%%%%%%%%%%%%%%%%%%%%%%%%%%%%%%%%%%%%%%%%%%%%%%%%%%%%%%%%%%%%%%%
\section{Model Analysis}

\subsection{Measurements of Problem}
\subsubsection{Performance Metrics}
\begin{itemize}
\item Intersection over Union (IoU)
	\begin{itemize}
	\item Input
		\begin{itemize}
		\item two regions, from prediction and label \\
		e.g. bounding box, segmentation mask
		\end{itemize}
	\item Definition
		\begin{itemize}
		\item $\text{I}=$ intersection of the regions
		\item $\text{U}=$ union of the regions
		\item $\Rightarrow \text{IoU} = \frac I U$ 
		\end{itemize}
	\item Use Case
		\begin{itemize}
		\item evaluation of object detection/segmentation
		\item post-processing in object detection/segmentation \\
		(e.g. \hyperref[DL_CV_Objdet_nonmax]{non-max suppress})
		\end{itemize}
	\item Understanding
		\begin{itemize}
		\item measure how well two regions overlap \\
		$\Rightarrow$ usually $\text{IoU}>0.5$ considered a decent match
		\item average IoU over each prediction class \& scale of threshold for an overview report \\
		(e.g. average over all obstacle types, threshold range $[0.05~0.95]$ with step $0.05$)
		\end{itemize}
	\item Extension
		\begin{itemize}
		\item intersection over label/prediction region $\Rightarrow$ in case of unstable label/prediction region \\ 
		(while matching relation considered important in the task)
		\end{itemize}
	\end{itemize}
\item Bilingual Evaluation Understudy (BLEU)
	\begin{itemize}
	\item Input
		\begin{itemize}
		\item one sequence as prediction; other sequence(s) as references \\ 
		(can be more than one) i.e. label
		\end{itemize}
	\item Definition
		\begin{itemize}
		\item $n$-gram $g_n$: $n$ continuous tokens
		\item count of $g_n$: the number of appearance of $g_n$ in a sequence \\
		$\Rightarrow$ num of $g_n \displaystyle =\sum_{g_n} (\text{count of }g_n) = l-n+1$, in sequence of len $l$
		\item ref count: $\displaystyle \sum_{g_n\in\text{pred}} (1 \text{ if it appears in any reference; else } 0)$ \\
		$\Rightarrow$ count of $g_n$ in prediction that also appears in a reference \\
		$\Rightarrow g_n\in$ pred matched as long as $g_n\in$ a reference
		\item $\Rightarrow$ precision $\displaystyle = \frac {\text{true positive}} {\text{positive pred}} = \frac {\text{ref count}} {\text{num of }g_n\in \text{pred}}$\\
		(yet, "the the the the" has high score when $n=1$, as "the" almost always appear)	
		\item clipped ref count: $\displaystyle \min\{ \text{ count of }g_n \text{ in pred }, \max(\text{ num of }g_n\text{ in a reference }) \}$ \\
		$\Rightarrow$ count only unique $n$-gram in pred \\
		$\Rightarrow g_n\in$ pred matched up to the max num of $g_n\in$ reference
		\item $\Rightarrow$ clipped precision $\displaystyle p_n = \frac {\text{clipped true positive}}{\text{positive pred}} = \frac {\text{clipped ref count}} {\text{num of }g_n \in \text{pred}}$
		\item $\Rightarrow$ BLUE score $\displaystyle = BP\exp\left(\frac 1N \sum_{n=1}^N p_n\right)$, \\
		where $BP=1$ if $\text{len}_\text{pred}>\text{len}_\text{ref}$; else $\exp(1-\text{len}_\text{pred}/\text{len}_\text{ref})$ \\
		$\Rightarrow$ the brevity penalty to penalize too short pred \\
		(as short pred has larger chance to have all its gram contained in ref)
		\end{itemize}
	\item Use Case
		\begin{itemize}
		\item machine translation
		\item image caption
		\end{itemize}
	\item Understanding
		\begin{itemize}
		\item clipped ref count: has a maximal number for $g_n$ appearance, given the reference\\
		$\Rightarrow$ same $g_n$ exceeding the number becomes false positive, as can NOT be matched \\
		(analogy: metrics in obj detection with bounding box)
		\item evaluate the appearance of generated tokens (prediction) in references (label)
		\item highly correlated with human evaluation
		\item bias towards statistical model (when compared against rule-based model)
		\end{itemize}
	\end{itemize}
\end{itemize}

\subsubsection{Expected Generalization Ability Measurement}
\begin{itemize}
\item Bayes Optimal Performance
	\begin{itemize}
	\item Measurement
		\begin{itemize}
		\item the theoretically best performance on all data \\ 
		$\Rightarrow$ modeling only the indent mapping without the noise (a perfect model) \\
		(note: in practice, only approximated bayes performance available)
		\end{itemize}
	\item Understanding
		\begin{itemize}
		\item measure the inherent noise in data as the best possible performance on all data
		\item $\Rightarrow$ measure \textbf{avoidable} bias: indicate the upper-bound performance
		\item $\Rightarrow$ measure degree of \textbf{overfitting}: indicate how much the model fit to the noise
		\end{itemize}
	\end{itemize}

\item Human Performance: Approximation to Bayes Performance
	\begin{itemize}
	\item Definition
		\begin{itemize}
		\item for best approximation: best achievable performance by human \\  (e.g. group of experts, as bayes optimal performance is even better)
		\item for specific focus: depends on use case \\
		e.g. for self-diagnose model, may define as the performance of a normal doctor
		\end{itemize}
	\item Practice
		\begin{itemize}
		\item usually done in supervised learning \\
		note: the label (i.e. $0$-error) is NOT bayes performance
		\item on unstructured data, human almost achieves bayes performance \\
		(as human good at natural perception task, like cv, nlp)
		\end{itemize}
	\item Cons
		\begin{itemize}
		\item hard to distinguish surpassing human performance from overfitting training set
		\end{itemize}
	\end{itemize}

\item (Avoidable) Bias
	\begin{itemize}
	\item Measurement
		\begin{itemize}
		\item gap between train set performance and bayes performance \\
		(note: in practice, only approximated bayes performance available)
		\end{itemize}
	\item Understanding
		\begin{itemize}
		\item measure the model capacity of handling given (train) data \\
		as (approximately) measuring the gap between the theoretically best model
		\end{itemize}
	\end{itemize}

\item Variance
	\begin{itemize}
	\item Measurement
		\begin{itemize}
		\item performance gap between val set \& train set (if under same distribution)
		\item if different distribution for train\&val set $\Rightarrow$ train-val set instead of val set
		\end{itemize}
	\item Understanding
		\begin{itemize}
		\item measure the generalization ability: \\ 
		as measuring how much model can cope with unseen data \\ 
		$\Rightarrow$ model the indent mapping, instead of the noise
		\end{itemize}
	\end{itemize}

\item Mismatch Distribution
	\begin{itemize}
	\item Measurement
		\begin{itemize}
		\item performance gap between train-val set \& val set
		\end{itemize}
	\item Understanding
		\begin{itemize}
		\item train-val set contains the unseen train data; val set the unseen target data \\
		$\Rightarrow$ gap only caused by different distribution between sets
		\end{itemize}
	\end{itemize}
\end{itemize}

2. Interaction with regularization:

- Improper $\lambda$:
- large $\lambda$ => high bias
- small $\lambda$ => high variance
- Choosing $\lambda$:
- try $\lambda=0,0.01,0.02,0.04,...,10$
- select the model with lowest $J_{cv}(\theta)$ without regularization term

3. Interaction with training set size:

- Normal Learning curve:

![Normal learning curve](../../Machine%20Learning/Statistical%20Machine%20Learning/Normal%20learning%20curve.png) 

- Learning curve with high bias:

- where getting more training data **doesn't** help

![Learning curve with high bias](../../Machine%20Learning/Statistical%20Machine%20Learning/Learning%20curve%20with%20high%20bias.png) 

- Learning curve with high variance:

- where getting more training data **helps**

![Learning curve with high variance](../../Machine%20Learning/Statistical%20Machine%20Learning/Learning%20curve%20with%20high%20variance.png) 

4. Ways to fix:

- High bias:
- more features / more polunomial terms of features
- decreasing $\lambda$

- High variance:

- larger data setOrtho
- fewer features
- increasing $\lambda$

- **In neural network:**

- High bias => larger neural networks (more hidden layers / more units in one layer)

- High variance => smaller neural networks

**Larger network with regularization ($\lambda$) is more powerful**

\subsection{Improving Model}

\subsubsection{Bias-Variance Guideline}
\begin{itemize}
\item Solving High (Avoidable) Bias
	\begin{itemize}
	\item Increasing Model Capability
		\begin{itemize}
		\item increase complexity: more weights, latent variable / hidden layer etc.
		\item use more suitable model specifically designed for the data (e.g. CNN for image)
		\end{itemize}
	\end{itemize}
	$\Rightarrow$ until fitting training set well
\item Solving High Variance
	\begin{itemize}
	\item Data Augment
		\begin{itemize}
		\item get/simulate more training data (via crowd sourcing, distortion, GANs, etc.)
		\end{itemize}
	\item Model Regularization
		\begin{itemize}
		\item control the complexity of model (e.g. L0/1/2 normalization)
		\end{itemize}
	\end{itemize}
\item Solving Trade-off
	\begin{itemize}
	\item Iterative Process
		\begin{itemize}
		\item solve bias, then solve variance, iteratively
		\end{itemize}
	\item Complexity + Data/Regularization
		\begin{itemize}
		\item increase complexity to solve bias \\
		without hurting variance (via more data/regularization)
		\item more data/regularization to solve variance \\
		without hurting bias (with enough complexity)
		\end{itemize}
	\end{itemize}
\end{itemize}

\subsubsection{Behavior Detail}
\begin{itemize}
\item Low Bias, High Variance (Over-fitting)
	\begin{itemize}
	\item Symptom
		\begin{itemize}
		\item good performance on train set \& poor generalization (bad on val) \\
		\item $\Rightarrow$ good at fitting train set; bad at representing/modeling underlying data source
		\end{itemize}
	\item Cause
		\begin{itemize}
		\item too much representation ability (to fit even the noise)
		\item directly model the likelihood instead of posterior
		\end{itemize}
	\item Remedy
		\begin{itemize}
		\item larger dataset
		\item regularization (model the posterior by accounting prior)
		\end{itemize}
	\end{itemize}
\item High Bias, Low Variance (Under-fitting)
	\begin{itemize}
	\item Symptom
		\begin{itemize}
		\item bad at fitting training examples \& modeling underlying data source \\ 
		(bad at train \& val)
		\item $\Rightarrow$ poor performance on train set \& good generalization (though meaningless)
		\end{itemize}
	\item Cause
		\begin{itemize}
		\item lack of representation ability (not enough flexibility)
		\end{itemize}
	\item Remedy
		\begin{itemize}
		\item try model with better representative ability (more complexity, flexibility)
		\end{itemize}
	\end{itemize}
\item High Bias, High Variance (Over\&Under-fitting)
	\begin{itemize}
	\item Symptom
		\begin{itemize}
		\item bad at fitting some general cases; while good at some rare and special cases \\
		(especially in high dimensional space) \\
		$\Rightarrow$ fitting largely noise
		\end{itemize}
	\item Cause
		\begin{itemize}
		\item model probably not suitable for the dataset
		\end{itemize}
	\item Remedy
		\begin{itemize}
		\item switch to other types of model
		\item dataset preprocessing
		\end{itemize}
	\end{itemize}
\item Low Variance, High Mismatch
	\begin{itemize}
	\item Symptom
		\begin{itemize}
		\item good at generalizing (on train-val); bad at target data (on val)
		\end{itemize}
	\item Cause
		\begin{itemize}
		\item model not able to generalize across mismatch distribution \\ 
		(yet generalized well in the same distribution: as good at train-val)
		\end{itemize}
	\item Remedy
		\begin{itemize}
		\item transform train data towards (more like) target data \\
		e.g. data synthesis: adding noise that is special in target data, etc.
		\item ensure train set contains enough / assign larger weight to, the desired target data
		\item \hyperref[DL_Learning_Transfer]{transfer learning}, adaptive learning, etc.
		\end{itemize}
	\end{itemize}
\item Low Bias, Low Variance, Low Mismatch, High val-test Variance
	\begin{itemize}
	\item Symptom
		\begin{itemize}
		\item model with specific hyperparameter overfitting the val set \\
		$\Rightarrow$ val set NO longer reveal model generalizability
		\end{itemize}
	\item Cause
		\begin{itemize}
		\item val set overfit by iteration of hyperparameter tunning (as a practice of fitting)
		\end{itemize}
	\item Remedy
		\begin{itemize}
		\item more data for val\&test set
		\item re-design/choose the model after val set overfitting fixed \\ 
		(after true generalizability reported)
		\end{itemize}
	\end{itemize}
	
\item Low Bias, Low Variance
	\begin{itemize}
	\item Behavior
		\begin{itemize}
		\item good at fitting training examples \& modeling underlying data source \\ 
		(good at train \& val)
		\item $\Rightarrow$ good performance on training set \& good generalization
		\end{itemize}
	\end{itemize}
\item High Bias, Low Variance, Lower/Negative Mismatch
	\begin{itemize}
	\item Behavior
		\begin{itemize}
		\item perform better on val\& test set then on train set \\
		$\Rightarrow$ target data distribution easier than train set distribution \\
		$\Rightarrow$ able to do well on desired data, even if not good on train set \\
		(better convinced by measuring human performance on both distribution)
		\end{itemize}
	\end{itemize}
\end{itemize}

\subsubsection{Error Analysis}
\begin{itemize}
\item Categorizing Error Source
	\begin{itemize}
	\item Practice
		\begin{itemize}
		\item create histogram on val set reflecting data categories \\
		(e.g. for image: blurry, rotated, incorrect label, etc...) \\
		$\Rightarrow$ categorize data first
		\item $\Rightarrow$ data leading to error scattered into different categories \\
		$\Rightarrow$ evaluate the contribution to error from different categories
		\item note: 
		\end{itemize}
	\item Understanding
		\begin{itemize}
		\item find out the most important error source \\
		$\Rightarrow$ prioritize the direction of tunning model
		\end{itemize}
	\end{itemize}
	
\item Ceiling Analysis
	\begin{itemize}
	\item Definition \& Practice
		\begin{itemize}
		\item
		\end{itemize}
	\item Understanding
	\end{itemize}
\end{itemize}

\subsubsection{Approaches Analysis}
\begin{itemize}
\item End-to-End Approach
	\begin{itemize}
	\item Definition
		\begin{itemize}
		\item use single network to learn the mapping from input directly to desired output \\
		(no intermediate result)
		\end{itemize}
	\item Pros
		\begin{itemize}
		\item reveal the data statics: avoid any specific prior
		\item large \& auto feature extraction
		\end{itemize}
	\item Cons
		\begin{itemize}
		\item need enough data for effective end-to-end model
		\item hard to inject effective prior into model \\
		$\Rightarrow$ exclude potentially hand-designed component/knowledge
		\end{itemize}
	\item Understanding
		\begin{itemize}
		\item end-to-end model works only when enough data to reveal the problem complexity
		\end{itemize}
	\end{itemize}
\end{itemize}

1. Aim:

- Decide which modules might be the best use of time to improve

2. Procedure:

- Draw a table with 2 column (Component - Accuracy)

- Component: the modules simulated to be perfect (100% accuracy)
- Accuracy: the accuracy of the entire system on the test set (define by chosen evaluation matrix)

- |   **Perfect Component**   |             **Accuracy**              |
| :-----------------------: | :-----------------------------------: |
|           none            |                  $f$                  |
|        module $1$         |            $f+\epsilon_1$             |
|       module $1,2$        |       $f+\epsilon_1+\epsilon_2$       |
| $\cdot \\ \cdot \\ \cdot$ |       $\cdot \\ \cdot \\ \cdot$       |
|    module $1,2,...,n$     | $f+\epsilon_1+...+\epsilon_n = 100\%$ |

- => Improving module $x$ will gain at most $\epsilon_x$ improvement in the overall performance

- Choose the module with most significant $\epsilon$ to improve


1. Procedure:
- Algorithm (trained) misclassifies $n$ data in cross validation set
- Classify these $n$ data and rank them
- Maybe more features are found
2. Feature selection => Numerical evaluation
- => test algorithm with / without this feature on **CV set** (compare error rate)

\subsection{Evaluating Hypothesis}


4. Choosing procedure:

- Minimize traning error $J_{train}(\theta)$ 
- Select a model with lowest $J_{cv}(\theta)$ 
- Estimate generalization error as $J_{test}(\theta)$ 


\subsection{Skewed classes}

1. Precision / Recall

- |                 | **Actual 1**   | **Actual 0**   |
| --------------- | -------------- | -------------- |
| **Predicted 1** | True positive  | False positive |
| **Predicted 0** | False negative | True negative  |

- \textbf{Precision} = $\displaystyle \frac{\text {True positive}}{\text{Predicted positive}} = \frac{\text {True positive}}{\text{True pos + False pos}}$

- **Recall** = $\displaystyle \frac{\text{True positive}}{\text{Actual positive}} = \frac{\text{True positive}}{\text{True pos + False neg}}$

2. Evaluation with precision/recall

- Predict 1 if $ h_\theta(x) \geq \epsilon$, 0 if $h_\theta(x) < \epsilon$

- larger $\epsilon$ => higher precision, lower recall $\small \text{(more confident)}$ 
- smaller $\epsilon$ => lower pecision, higher recall $\small \text{(avoid missing)}$       

![Posiible Precision -Recall curev](../../Machine Learning/Statistical 0Machine Learning/Posiible Precision -Recall curev.png)  

3. Compare precision/recall num

- $\displaystyle \text{F}_1 \space  Score = 2\frac{PR}{P+R}$, $P$ as precision, $R$ as recall
- higher better, on cross validation set

4. High precision \& high recall:

- **large num of features $\small\text{(low bias)}$ + large sets of data $\small\text{(low variance)}$**


\section{Supervised Learning}
\begin{itemize}

\item Feature normalization: $\forall x_{ij} \in X, x_{ij}=\frac{x_{ij}-\mu_j}{\sigma_j^2}$, $ X:[instance][feature]$, without $[1...1]^T$ in 1st column $X=[x_1,x_2,...,x_m]$, m instances in total
\item Regularization: add penalty for $\theta$ being large into cost function
\item $\displaystyle J(\theta)= \space... + \frac{\lambda}{2m}\sum^n_{j=1}\theta_j^2$ , \textbf{bias $\theta_0$ shouldn't be penalized} 

\end{itemize}

\section{Linear Regression}

\begin{itemize}

\item Notation
	\begin{itemize}
	\item $t$: observed data
	\item $\displaystyle y(\mathbf x,\mathbf w)=\sum_{i=0}^{M}\phi_i(\mathbf x)w_i = \mathbf w^T \phi(\mathbf x)$ : model generating ground truth, with
		\begin{itemize}
		\item $\mathbf w$: weight vector
		\item $\phi(\mathbf x)$: basis function for feature vector $\mathbf x$, with usually $\phi_0(\mathbf x) = 1$ as bias
		\end{itemize}
	\end{itemize}

\item Assumption
	\begin{itemize}
	\item Deterministic Model with Observation Noise
		\begin{itemize}
		\item \(t=y(x,w)+\epsilon\), where
		
		$\epsilon \sim \mathcal N(0,\beta^{-1})$ is Gaussian noise where precision (inverse variance) $\beta$
		\item $\Rightarrow$ consequence
			\begin{enumerate}
			\item $\text{likelihood } p(t|\mathbf x,\mathbf w,\beta) = \mathcal N(t|y(\mathbf x,\mathbf w), \beta^{-1})$
			\item $\mathbb E[t|\mathbf x] = \int t\cdot p(t|\mathbf x) dt = y(\mathbf x, \mathbf w) $
			\item unimodal distribution $p(t|\mathbf x) \Rightarrow$ extended by conditional mixture model
			\end{enumerate}

		\end{itemize}
	\end{itemize}

\item Joint Likelihood

	\begin{itemize}
	\item $\displaystyle P(\mathbf t|\mathbf X,\mathbf w,\beta) = \prod_{n=1}^N \mathcal N(t_n|\mathbf w ^T\phi(\mathbf x_n), \beta^{-1})$, where
		\begin{itemize}
		\item $\mathbf X = \{ \mathbf x_1,...,\mathbf x_N \}, \mathbf t = \{ t_1,...,t_N \}$
		\end{itemize}
	\item Log Likelihood 
		\begin{itemize}
			\item $\displaystyle \ln P(\boldsymbol y|X,\theta,\beta) = \frac N 2\ln\beta - \frac N 2 \ln(2\pi) - \beta \frac 1 2\sum_{i=1}^m (h_\theta(x^i) - y^i)^2$
		\end{itemize}
	 
	\end{itemize}

\item Log Posterior leads to regularization

	\begin{itemize}
	\item Maximizing the likelihood function $\Rightarrow$ (often) excessively complex models \& over-fitting
	\item Regularization term comes from the $\text{Prior}$: 
		\begin{itemize}
		\item $\text{assume Prior } p(\theta) =  \mathcal{N}(\theta|0,\alpha^{-1}I) \text{\small , so that Posterior \& Prior are of the same distribution}$ \\
		$\text{to maximize log Posterior}:$ \\ 
		$\displaystyle \Rightarrow \ln p(\theta|X) \propto -\frac \beta 2 \sum_{i=1}^n (y^i - h_\theta(x))^2 - \frac \alpha 2 \theta^T\theta + const$
		
		\item If $\alpha \to 0 \text { (Prior is most useless)}$, maximise $\text{Posterior}$ is equivalent to maximizing likelihood 
		\item Maximize $\text{Posterior} \Leftrightarrow$ Minimize $\text{cost function with regularization}$, where $\lambda = \alpha/\beta$ 
		\end{itemize}
	\end{itemize}
		
\item Predictive Distribution: $p(y|x,X,Y)$ 

	\begin{itemize}
	\item $\displaystyle p(y|x,X,Y) = \int p(y,\theta|x,X,Y)d\theta = \int p(y|\theta,x,X,Y)p(\theta|x,X,Y)d\theta$ 
	
	\item $p(y|\theta,x,X,Y)=p(y|\theta,x) = \mathcal{N}(y|h(x,\theta), \beta^{-1})$ \\ 
	\(\small  \text{based on assumption: } y = y(x,\theta)+\epsilon \text{, where $\epsilon$ is Guassian noise}\) \\ 
	\(p(\theta|x,X,Y)=p(\theta|X,Y) = \text{posterior} \)
	
	\item  $ \displaystyle \Rightarrow p(y|x,X,Y)=\int p(y|\theta,x) p(\theta|X,Y)d\theta$   

	\item  $\text{Expected Lost}  = (bias)^2 + variance + noise$ 
	\end{itemize}
		
\item Notation:

	\begin{itemize}
	\item $t=y(x,w)+\epsilon,\text{ where } \epsilon \text{ is Gaussian noise}$ 
	\item $\hat y$ is prediction function to approximate $y=y(x,w)$ 	
	\end{itemize}

\item Procedure:

	\begin{itemize}
	\item $\mathbb E[(t-\hat y)^2] = \mathbb E[t^2-2t\hat y+\hat y^2] \\ = \mathbb E[t^2] +\mathbb E[\hat y^2] -\mathbb E[2t\hat y] \\ = \text{Var}[t]+\mathbb E[t]^2 + \text{Var}[\hat y]+\mathbb E [\hat y]^2 - 2y\mathbb E[\hat y] \\ = \text{Var}[t] + \text{Var}[\hat y] + (y^2 -2y\mathbb E[\hat y] +\mathbb E[\hat y]^2) \\ =  \text{Var}[t] + \text{Var}[\hat y] + (y-\mathbb E[\hat y])^2 \\ =  \text{Var}[t] + \text{Var}[\hat y] + \mathbb E[t-\hat y]^2 \\ = \sigma^2+\text{Var}[\hat y] + \text{Bias[$\hat y$]}^2$ 
	
	$\text{where } \sigma^2 = \text{Var}[\epsilon] \text{ is the noise}$ 
	
	(formula used: $\text{Var}[x] = \mathbb E[x^2] - \mathbb E[x]^2 \Leftrightarrow \mathbb E[x^2] = \text{Var}[x]+ \mathbb E[x]^2$) 
	
	\item Matrix inverse can be evil \& avoid inverse operation: 
	
	$A = U\Lambda U^T  \text{, where $\Lambda$ is diagonal matirx} \\ => A^{-1} = U\Lambda^{-1}U^T \\ \text {but number on the diagonal line of $\Lambda$ can be small => maybe 0 depending on accuracy of computer}$ 
	\end{itemize}


\end{itemize}


\section{Bayesian Regression}

\begin{itemize}

\item Assumption:

- $t=y(x,w)+\epsilon,\text{ where } \epsilon \text{ is Gaussian noise}; y(x,w)\text{ approximated by }\phi(x)w$ 

\item Bayesian view:

\item Gaussian $\text{Prior}: p(w) = \mathcal N(w|m_0,S_0)$ 

	$\text{Reason: to be conjugate}$ 

\item $\text{Likelihood}: \displaystyle p(\boldsymbol t|w) = \prod_{n=1}^N\mathcal N(t_n|w^T\phi(x_n),\beta^{-1}) = \mathcal N(\boldsymbol t|\Phi w,\beta^{-1}I)$ 

\item $\Rightarrow \text{Posterior}: p(w|\boldsymbol t) = \mathcal N(w|m_N,S_N)$ 

	$\text{where } m_N = S_N(S_0^{-1}m_0+\beta \Phi^T\boldsymbol t),\space S_N^{-1} = S_0^{-1}+\beta \Phi^T\Phi $ 

\item Maximum Likelihood:

	\begin{itemize}
	
	\item $\text{Likelihood}: \displaystyle p(\boldsymbol t|w) = \prod_{n=1}^N\mathcal N(t_n|\phi(x_n)w,\beta^{-1})$ 
		\begin{itemize}
		\item meaning: how probable the observed dataset is w.r.t the model setting (under parameter $w$)
		\end{itemize}
	
	\item $\displaystyle \ln \text{Likelihood} = \sum_{n=1}^N [-\ln \frac {\beta} {\sqrt {2\pi}} - \frac \beta 2 (t_n-\phi(x)w)^2]$ 
	
	\item $\displaystyle \frac {\partial} {\partial w} \ln \text{Likelihood} = \beta \Phi^T(\boldsymbol t-\Phi w)$ 
	
	let $\frac {\partial} {\partial w} \ln \text{Likelihood}=0$ 
	
	$\displaystyle \Rightarrow w_{ML} = (\Phi^T\Phi)^{-1} \Phi^T\boldsymbol t $ 
	
	\item $\displaystyle \frac {\partial} {\partial \beta} \ln \text{Likelihood} = -N\beta^{\frac 1 2} + \beta^{\frac 3 2}(\boldsymbol t-\Phi w)^T(\boldsymbol t-\Phi w) $  
	
	let $\frac {\partial} {\partial \beta} \ln \text{Likelihood}=0$ 
	
	$\displaystyle \Rightarrow \beta^{-1}=\frac 1 N (\boldsymbol t-\Phi w)^T(\boldsymbol t-\Phi w)$ 
	
		\(\text{Note: solve $w=w_{ML}$ first}\) 
	\end{itemize}

\item Maximum Posterior:

	\begin{itemize}
	\item $\text{Posterior}=p(w|\boldsymbol t), \text{Prior}=p(w), \text{Likelihood} = p(\boldsymbol t|w), \text{Normalization}=p(t) \\ \displaystyle \Rightarrow \text{Posterior}=\frac {\text{Likelihood*Prior}} {\text{Normalization}}$ 
	
	$\Rightarrow \text{Posterior} \propto \text{Likelihood*Prior} $  
	
	\item $\text{assume Prior } \displaystyle p(w) =  \mathcal{N}(w|m_0,S_0), \\ \small \text{so that Prior \& Likelihood are conjugate} \Rightarrow \text{Gaussian Posterior}$ 
	
	\item $\text{Likelihood } \displaystyle p(\boldsymbol t|w) = \prod_{n=1}^N\mathcal N(t_n|\phi(x_n)w,\beta^{-1}) =\mathcal N(\boldsymbol t|\Phi w,\beta^{-1}I )$ 
	
	\item $\Rightarrow \text{Posterior } \displaystyle p(w|\boldsymbol t) = \mathcal N(w|m_N,S_N),\\ \text{where } m_N=S_N(S_0^{-1}m_0+\beta\Phi^T \boldsymbol t),\space S_N^{-1}=S_0^{-1} + \beta \Phi^T\Phi$ 
	
	$\Rightarrow w_{MAP} = \text{mean of the Gaussian} = m_N$ 
	
		$\text{Note: can also get } w_{MAP} \text{ from taking gradient}$ 
	\end{itemize}

\item Simple Prior:

$\text{Prior } p(w)=\mathcal N(w|0,\alpha^{-1}I)$ 

$\Rightarrow \text{Posterior } \displaystyle p(w|\boldsymbol t) = \mathcal N(w|m_N,S_N),\\ \text{where } \displaystyle m_N=\beta(\alpha I + \beta \Phi^T\Phi)\Phi^T \boldsymbol t,\space S_N^{-1}=\alpha I + \beta \Phi^T\Phi$ 

$ w_{MAP} \rightarrow w_{ML}, \text{ when }\alpha \rightarrow 0 \text{ (most useless Prior)}$ 

\item Maximize $\text{Posterior} \Leftrightarrow$ Minimize $\text{cost function with regularization}$: 

	$ \text{Simple Prior} \Rightarrow \displaystyle \ln p(w|\boldsymbol t) = -\frac \beta 2 (\boldsymbol t-\Phi w)^T (\boldsymbol t-\Phi w)-\frac \alpha 2 w^Tw+const$ 

\item If $\alpha \to 0 \text { (Prior is most useless)}$, maximize $\text{Posterior}$ is equivalent to maximizing likelihood 
\item Maximize $\text{Posterior}$ equal to minimize sum-of-squares error function with the addition of a quadratic regularization term with $\lambda = \alpha/\beta$ 
\item Regularization term comes from the $\text{Prior}$ 

\item Predictive Distribution:

\item Assume: $\text{Prior}: p(x|\alpha) = \mathcal N(x|0,\alpha^{-1}I),\space \small\text{ (}m_0=0,S_0=\alpha^{-1}I \text{)}$ 

\item $\displaystyle p(t|x,X,\boldsymbol t) = \int p(t|w,x)p(w|X,\boldsymbol t)dw$ 

\item $\Rightarrow p(t|x,X,\boldsymbol t) = \mathcal N(t|m_N^T\phi(x),\sigma^2_N(x))$ 

	$\text{where } \sigma_N^2(x)=\frac 1 \beta + \phi(x)^TS_N\phi(x);\space m_N, S_N \text{ from Posterior} \small (m_N=w_{MAP})$ 

\item Sequential data:

	\begin{itemize}
	\item $\text{Posterior}$ from previous data $\Leftrightarrow$ the $\text{Prior}$ for the arriving data
	\item Sequential data and data in one go is equivalent when finfding the Porsterior
	\end{itemize}

\item Gradient descent

	\begin{itemize}
	\item Hypothesis function: 
		\begin{itemize}
		\item $h_\theta(x)=x\theta$, $\small\theta = [\theta_0, \theta_1, ..., \theta_n]^T$, $\small x=[x_0, x_1,..., x_n], x_0=1$
		\end{itemize}
	
	\item $x$ is one instance
	\item Cost function: $\displaystyle J(\theta)=\frac{1}{2m}\sum_{i=1}^m(h_\theta(x^i)-y^i)^2+\frac{\lambda}{2m}\sum^n_{j=1}\theta_j^2$ 
	\item Update rule: $\forall \theta_j \in \theta, \theta_j := \theta_j-\alpha\large\frac{\partial J(\theta)}{\partial\theta_j}$, $\displaystyle \small\frac{\partial J(\theta)}{\partial\theta_j}=\frac{1}{m}\sum^m_{i=1}[(h_\theta(x^i)-y^i)x_j^i] + \frac\lambda m\theta_j$ 
	- $\displaystyle \frac {d}{d\theta}J(\theta) = \frac 1m ((X\theta-y)^TX)^T + \frac \lambda m [0,\theta_1,...,\theta_m]^T$ ($\theta_0$ shouldn't be penalized)
	\item simultaneously for all $\theta_j \in \theta$ 
	\item Normal equation (mathematical solution)
		\begin{itemize}
		\item $\theta = (X^TX)^{-1}X^Ty$  
		\end{itemize}
	\end{itemize}
\end{itemize}

\section{Logistic Regression (Classification)}

- Decision Theory:

- classes $C_1,...,C_K$, decision regions $\mathcal {R}_1,...,\mathcal{R}_K$ 
- Minimze $\displaystyle p(mistake) = \sum_{k=1}^K (\int_{\mathcal {R}_k} \sum_{i \neq k} p(x,C_i)dx)$ $\small \text{(can have weight on each misclassfication though)}$  
- Maximize $\displaystyle p(correct) = \sum_{k=1}^K \int _{\mathcal {R}_k}p(x,C_k)dx$ 

- Models for Decision Problems:

- Find a discriminant function
- Discriminative Models: $\text{less powerful, yet less parameter => easier to learn}$ 
- Infer **posterior** $p(C_k|x)$, $\small \text{$C_k$: $x \in C_k$, $x$ is examples in trainning set}$ 
- Use decision theory to assign a new $x$ 
- Generative Models: $\text{more powerful, but computationally expensive}$ 
- Infer conditional probabilities $p(x|C_k)$ 
- Infer prior $p(C_k)$ 
- Find **either** **posterior** $p(C_k|x)$, **or** **joint distribution** $p(x, C_k)$ (using Bayes' theorem)
- Use decision theory to assign a new $x$ 
- **=> Able to create synthetic data using $p(x)$** 

- Naive Bayes on Discrete Features:

- Assumption:

- Discrete Features: data point $x\in\{0,1\}^D$ 

- Naive Bayes: all features conditioned on class $C_k$ are independent with each other

$\displaystyle \Rightarrow p(x|C_k)=\prod_{i=1}^D \mu_{ki}^{x_i} \space (1-\mu_{ki})^{1-x_i}$ 

1. Linear Discriminant (Least Squares Approach)

- Prediction:

- $y(x)=xw+w_0 \text{, with bias $=w_0$, where } w = [w_1,...,w_n]^T, x=[x_1,...,x_n]$ 
- $y(x)>0$: positive confidence to assign $x$ to current class 
- $-w_0$ called threshold sometimes

- Decision Boundary $y(x)=w^Tx+w_0=0$: 

- $w$ is orthogonal to vectors on the boundary:

	$\text{assume } x_1,x_2 \text{ on the boundary} \\ \Rightarrow 0=y(x_1) - y_(x_2)=(x_1-x_2)w$ 

- Distance from origin to boundary is $-\frac {w_0} {\|w\|}$: 

	$\text{assume distance is } k \\ \Rightarrow k \frac {w}{\|w\|} \text{on boundary, thus } k \frac {w}{\|w\|} w + w_0=0 \\ \Rightarrow k=-\frac {w_0}{\|w\|}$ 

- $y(x)$ is a signed measure of distance from point $x$ to boundary:

	$\text{assume distance is }r \\ \Rightarrow y(x)=\overbrace{(x_\perp+r \frac {w}{\|w\|})}^x w + w_0 = \overbrace{x_\perp w+w_0}^0 + r\|w\| = r\|w\| \\ \Rightarrow r = \frac{y(x)} {\|w\|}$ 

- Multi-class $\small\text{(k-classes)}$:

- prediction: $x$ is of class ${\mathcal C}_k$ if $\forall j \neq k, y_k(x) > y_j(x)$ 

$\Rightarrow y(x)=xW \text{, where } W=[w_1,...,w_k],\forall x_i \in X,x_{i0}=1 {\small\text{ (bias)}}, y(x) \text{ is 1-of-k coding} $ 

- sum-of-squares error: $E_D(W)=\frac 1 2 \text{tr}\{(XW-T)(XW-T)^T\}$ 

$\Rightarrow \text{optimal } W = (X^TX)^{-1}X^TT$ 

	$\small \text{note that } tr\{ AB \} = A^TB^T$ 

2. Fisher’s Linear Discriminant

- Basic idea:

- Take linear classification $y = w^Tx$ as dimensionality reduction (projection onto 1-D)
- => find a projection (denoted by vector $w$) which maximally preserves the class separation
- => if $y>-w_0$ then class $C_1$, otherwise $C_2$ 

- Goal:

- Distance within one class is small
- Distance between classes is large

- Mean \& Variance of Projected Data:

- Mean: $\displaystyle\widetilde m_k = w^Tm_k, \text{ where } m_k = \frac 1 {N_k}\sum_{x\in C_k} x$ 
- Variance: $\displaystyle \widetilde s_k=\sum_{x\in C_k}(w^Tx-w^Tm_k)^2 =   w^T[\sum_{x\in C_k}(x-m_k)(x-m_k)^T]w$ 

- 2-Classes to 1-D line:

- Maximize Fisher criterion: $\displaystyle J(w) = \frac {|\widetilde m_1 - \widetilde m_2|^2} {{\widetilde s_1}^2+{\widetilde s_2}^2}$ 

- Between-class covariance:  $\displaystyle S_B = (m_1-m_2)(m_1-m_2)^T$ 

- Within-class covariance: $\displaystyle S_k=\sum_{n\in C_k}(x_n-m_k)(x_n-m_k)^T$ 

$\displaystyle\Rightarrow J(w) = \frac {w^TS_B w} {w^T S_W w} $ 

- Lagrangian: $L(w,\lambda) = w^TS_B w + \lambda(1- w^T S_W w)$ 

	$\small \text{fix $w^T S_W w$ to 1 to avoid infinite solution (any multiple of a solution is a solution)}$

$\Rightarrow \displaystyle \frac \partial {\partial w}L = 2S_Bw-2\lambda S_Ww=0 \\ \Rightarrow S_Bw=\lambda S_Ww \\ \Rightarrow ({S_W}^{-1}S_B)w=\lambda w \\ \text{To maximize $J(w)$, $w$ is the largest eigenvector of ${S_W}^{-1}S_B$ if $S_W$ invertible} $ 

- K-classes to a d-D subspace: $\small N_k \text{ is num in class k, $N$ is the total example num}$ 

- Between-class covariance:  $\displaystyle S_B = \sum_{k=1}^K N_k(m_k-m)(m_k-m)^T, \text{where } m=\frac 1 N \sum_{n=1}^N x_n  \\ \small \text{reduce to } (m_1-m_2)(m_1-m_2)^T  \text{ when K=2 (constant ignored)}$ 

- Within-class covariance: $\displaystyle S_W = \sum_{k=1}^KS_k, \small \text{where } S_k=\sum_{n\in C_k}(x_n-m_k)(x_n-m_k)^T, m_k = \frac 1 {N_k} \sum_{n \in C_k}x_n$ 

- Maximize Fisher criterion: $\displaystyle J(w) = \frac {tr\{W^TS_B W\}} {tr\{W^T S_W W\}}$ 

- Lagrangian: 

$\text{Solve for each } w_i\in W \Rightarrow ({S_W}^{-1}S_B)w_i=\lambda_i w_i \\ \Rightarrow W \text{ conosists of the largest d eigenvectors} \\ \small {S_W}^{-1}S_B \text{ is not guaranteed to be symmetric } \Rightarrow W \text{ might not be orthogonal} \\ \small\text{Need to minimize the whole covariance matrix ($J(w)$ as a matrix) $\Rightarrow$ not choosing same eigenvectors twice} $  

- Maximum Possible Projection Directions = $K-1$: 

	$r({S_W}^{-1}S_B) \le \min(r({S_W}^{-1}),r(S_B)) \le r(S_B) \\ r(S_B) \le \displaystyle \sum_K r((m_k-m)(m_k-m)^T) = K \small \text{, as } r(m_k-m)=1 \\ \displaystyle \sum_Km_k = m \Rightarrow r(m_1-m,...,m_K-m)=K-1 \\ \Rightarrow r(S_B)\le K-1 \\ \Rightarrow r({S_W}^{-1}S_B) \le K-1$ 

3. Perceptron Algorithm

- Generalised linear model $y = f(w^T\phi(x))$, where $\phi(x)$ is basis function; $ \phi_0(x) = 1$ 

- Nonlinear activation funtion: $f(a) = \begin{cases} 1,& a \ge0 \\-1,& a<0  \end{cases}$ 

- Target coding: $t = \begin{cases} 1,&\text{if } C_1 \\ -1, &\text{if } C_2 \end{cases}$ 

- Cost function: 

- All correctly classified patterns: $w^T\phi(x_n)t_n>0$ 

- Add the errors for all misclassified patterns (denoted as set $\mathcal{M}$):

$\Rightarrow \displaystyle E_P(w)=-\sum_{n\in \mathcal{M}} w^T \phi(x_n)t_n$ 

- Algorithm: (Aim: minimize total num of misclassified patterns)

- loop 

	choose a traning pair $(x_n, t_n)$ 

	update the weight vector $w$: $w  = w - \eta \nabla E_p(w) = w+\phi_nt_n$ 

		$\text{ where $\eta$=1 because $y=f(\cdot)$ does not depend on $\|w\|$}$ 

- Perceptron Convergence Theorem:

- If the training set is linearly separable, the perceptron algorithm is guaranteed
to find a solution in a finite number of steps

$\small \text{(Also is the algorithm to find whether the set is linearly separable => Halting Problem)}$ 

4. Maximum Likelihood

- Assumption: 
- $p(x|C_k) \sim\mathcal{N}(\mu_k,\Sigma), \text{and all } p(x|C_k) \text{ share the same } \Sigma$ 
- \(p(C_1) = \pi,\space p(C_2)=1-\pi \text{, $\pi$ unknown}\) 
- Likelihood of whole data set $\boldsymbol{X,t}$, $N$ is the num of data
- $\displaystyle p(\boldsymbol{X,t}|\pi, \mu_1, \mu_2, \Sigma) = \prod_{n=1}^N[\pi\mathcal{N}(x_n|\mu_1,\Sigma)]^{t_n} \space [(1-\pi)\mathcal{N}(x_n|\mu_2, \Sigma)^{1-t_n}]$ $\space\small\rightarrow\text{when info of label $t$ lost: mixture of Gaussian}$  
- $\displaystyle \ln (\text{Likelihood}) = \sum_{n=1}^N[t_n(\ln\pi+\ln\mathcal{N}(x_n|\mu_1,\Sigma)) + (1-t_n)(\ln(1-\pi)+\ln\mathcal{N}(x_n|\mu_2, \Sigma))]$ 
- Parameters when maximum reached:
- $\displaystyle \pi = \frac {N_1}{N_1+N_2}$, $N_1$ is the num of class $C_1$ 
- $\displaystyle \mu_1 = \frac 1 {N_1} \sum_{n=1}^Nt_n x_n, \space \mu_2=\frac 1 {N_2}\sum_{n=1}^N(1-t_n)x_n,$ (mean of each class)
- $\displaystyle \Sigma = \frac {N_1}{N}S_1 + \frac {N_2}NS_2, \text {where } S_k = \frac 1 {N_k}\sum_{n \in C_k}(x_n-\mu_k)(x_n-\mu_k)^T $ 

5. Logistic Regression 

- Sigmoid function: $\displaystyle \sigma(a) = \frac 1 {1+e^{-a}}$ 

- $p(x|C_k) \sim \mathcal{N} \implies p(C_k|x)=\sigma(w^Tx+w_0) \small \text{ (2-classes)}$  (Generative model)

- Assumption:

$p(x|C_k) = \mathcal{N}(\mu_k, \Sigma)$ (can also be a number of other distributions)

$\forall k, p(x|C_k) \text{ shares the same } \Sigma$ 

- \begin{align*} \displaystyle p(C_1|x)=\frac{p(x|C_1)p(C_1)}{p(x|C_1)p(C_1)+p(x|C_2)p(C_2)} = \sigma (a), \\  \text{where } a &=\ln \frac {p(x,C_1)}{p(x,C_2)} \\ &= \ln \frac {p(x|C_1)p(C_1)}{p(x|C_2)p(C_2)} \\ &= \dots \text{(assumption applied)} \\ &= \ln \frac {\text{exp}(\mu_1^T\Sigma^{-1}x - \frac 12 \mu_1^T\Sigma^{-1}\mu_1)}{\text{exp}(\mu_2^T\Sigma^{-1}x - \frac 12 \mu_2^T\Sigma^{-1}\mu_2)} + \ln \frac {p(C_1)}{p(C_2)}  \\ \implies & a = w^Tx+w_0  \text{ where, } \\ &\space w = \Sigma^{-1}(\mu_1-\mu_2)  \\ & \space w_0 = -\frac 12\mu_1^T\Sigma^{-1}\mu_1 + \frac 12 \mu_2^T\Sigma^{-1}\mu_2 + \ln \frac {p(C_1)}{p(C_2)} \end{align*} 

- $\displaystyle \implies p(C_1|x) = \sigma(w^Tx+w_0)$ 

$\Rightarrow$ Find parameters in Gaussian model using Maximal Likelihood Sulotion

	as: $p(C_1|x)\propto p(x|C_1)p(C_1)=p(x,C_1)$ 

- Generalize to $\text{K-classes}$:

$\displaystyle a_k(x) = \ln [p(x|C_k)p(C_k)], \space p(C_k|x) = \frac{\text{exp}(a_k)}{\sum_i \text{exp}(a_i)}$ 

$\Rightarrow \displaystyle a_k(x) = w_k^Tx+w_{k0}, \text{where } w_k=\Sigma^{-1}\mu_k; \space w_{k0} = -\frac 1 2 \mu_k^T\Sigma^{-1}\mu_k + p(C_k)$ 

- $\text{Assume directly } p(C_k|x)=\sigma(w^Tx+w_0) \small \text{ (2-classes)}$ (Discriminative model)

- Assume directly: $p(C_1|w,x) = \sigma(w^Tx), \space x_0 = 1$ 

	$\Rightarrow$ less parameters to fit (compared to Gaussian)

- Likelihood function: 

$\displaystyle p(\boldsymbol{t}|w,X) = \prod_{n=1}^Np_n^{t_n}(1-p_n)^{1-t_n}, \text{ where, } p_n=p(C_1|x_n), \space t_n \text{ is the class of } x_n$  

Define error function :

	$\displaystyle E(w) = -\ln(Likelihood) = - \sum_{n=1}^N [t_n\ln p_n + (1-t_n)\ln(1-p_n)]$ 

$\displaystyle \Rightarrow \nabla E(w)=\sum_{n=1}^N(p_n-t_n)x_n$ 

- Find Posterior $p(w|\boldsymbol{t})$: 

$\text{Likelihood}$ is product of sigmoid

Conjugate $\text{Prior}$ for "sigmoid distribution" is unknown

$\Rightarrow$ Assume $\text{Prior } p(w) = \mathcal{N}(w|m_0, S_0)$ 

$\Rightarrow \displaystyle \ln p(w|\boldsymbol{t}) \propto -\frac 1 2 (w-m_0)^TS_0^{-1}(w-m_0) + \sum_{n=1}^N[t_n\ln p_n + (1-t_n) \ln(1-p_n) ]$ 

	find $w_{MAP}$, calculate $\displaystyle S_N = -\nabla \nabla \ln p(w|\boldsymbol{t}) = S_o^{-1} + \sum_{n=1}^N p_n(1-p_n)\phi_n\phi_n^T$ 

$\Rightarrow p(w|\boldsymbol{t}) \simeq \mathcal{N}(w|w_{MAP}, S_N), \small \text{via Laplace Approximation}$ 

- Laplace Approximation: 

- Fit a guassian to $p(z)$ at its **mode** ( mode of $p(z)$: point where $p'(z)=0$ )

- Assume $p(z) = \frac 1 Z f(z), \text{with normalization } Z = \int f(z)dz$ 

Taylor expansion of $\ln f(z)$ at $z_0$: $\ln f(z) \simeq \ln f(z_0)-\frac 1 2A(z-z_0)^2,$ 

	$\text{where } f'(z_0)=0, A=- \frac {d^2}{dz^2}\ln f(z) |_{z=z_0}$ 

Take its exponentiating: $f(z) \simeq f(z_0)\text{exp{$-\frac A 2 (z-z_0)^2$}}$ 

$\displaystyle \Rightarrow \text{Laplace Approximation} = (\frac A {2\pi})^{1/2} \text{exp{$-\frac A 2 (z-z_0)^2$}} \text{, where }A=\frac 1 {\sigma^2}$ 

- Requirement:

	$f(z)$ differentiable to find a critical point

	$f''(z_0) < 0$ to have a maximum \& so that $\nabla \nabla\ln f(z_0)=A>0$ as $A=\frac1 {\sigma^2}$ 

- In Vector Space: approximate $p(z)$ for $z\in \mathcal{R}^M$ 

Assume $p(z)=\frac 1 Z f(z)$

Taylor expansion: $\ln f(z) \simeq \ln f(z_0) - \frac 1 2 (z-z_0)^TA(z-z_0),$ 

	$\text{Hessian } A = - \nabla \nabla \ln f(z)|_{z=z_0} $ 

\begin{align}  \Rightarrow \text{Laplace approximation}&= \frac {|A|^{1/2}}{(2\pi^{M/2})}\text{exp{$-\frac 1 2 (z-z_0)^TA(z-z_0)$}} \\ &= \mathcal{N}(z|z_0, A^{-1}) \end{align}

- Gradient descent:

- Hypothesis function: $h_\theta(x)=\sigma(x\theta)=\large\frac{1}{1+e^{-x\theta}}$ 

- Cost function: 

$\displaystyle J(\theta)=\frac{1}{m}\sum^m_{i=1}[-y^i \ln (h_\theta(x^i))-(1-y^i) \ln (1-h_\theta(x^i))]+\frac{\lambda}{2m}\sum^n_{j=1}\theta_j^2$ 

- Update rule: $\forall \theta_j \in \theta, \theta_j := \theta_j-\alpha\large\frac{\partial J(\theta)}{\partial\theta_j}$, $\displaystyle \small\frac{\partial J(\theta)}{\partial\theta_j}=\frac{1}{m}\sum^m_{i=1}[(h_\theta(x^i)-y^i)x_j^i] + \frac \lambda m\theta_j$ 

\section{Latent Variable Analysis}

\subsection{Principal Component Analysis (PCA)}

1. Motivation:

- Data compression (reduce highly related features)
- Data visualization

2. Assumption:

- Gaussian distributions for both the latent and observed variables

3. Two Equivalent Definition of PCA:

- Linear projection of data onto lower dimensional linear space (principal subspace) such that: 

$\Rightarrow \text{variance of projected data is maximized} \\ \Rightarrow \text{distortion error from projection is minimized} $ 

4. Maximum Variance Formulation

- Goal: 

- project data from D dimension to M while maximizing the variance of projected data

- Eigenvalues $\lambda$ of covariance matrix $S$ express the variance of data set $X$ in direction of corresponding eigenvectors

- Projection Vectors: 

- $U = (u_1,...,u_M), \text{ where } \forall i\in\{1,..,M\},u_i \in \mathbb R^D \text{ s.t. } u_i^Tu_i=1 \space \small \text{(only consider direction)} $ 

- Projected Data:

- $\displaystyle \text{Mean} = {\bar x}^TU \text{, where } \bar x = \frac 1 N\sum_{i=1}^Nx^i$ 
- $\displaystyle \text{Variance} =  tr\{U^T S U\} \text{, where } S = \sum_{i=1}^N(x^i-\bar x)(x^i-\bar x)^T \space\small\text{ (outer product)}$ 

- Lagrangian to maximize $\text{Variance}$: 

- $\displaystyle L(U,\lambda)=tr\{U^TSU\}+tr\{(I-U^TU)\lambda\}$ 

	$\small\text{constraint }u_i^Tu_i=1 \text{ to prevent $u_i \rightarrow +\infty$ } $ 

\begin{gather}\text{For each $u_i\in U$, } \displaystyle \frac \partial {\partial u_i}L = 2Su_i-2\lambda_iu_i=0 \\ \Rightarrow Su_i=\lambda_iu_i \\ \Rightarrow \text{$U$ consists of eigenvectors corresponding to the first M large eigenvalue of $S$}\end{gather}

	\(\text{($S$ symmetric $\Rightarrow U$ orthogonal)}\)

5. Minimum Error Formulation:

- Introduce Orthogonal Basis Vector for D dimension:

- $U=(u_1,...,u_D)$ 

- Data representation:

- Original: $\displaystyle x^n = \sum_{i=1}^D\alpha^n_iu_i$ 
- Projected: \(\displaystyle \widetilde {x^n} = \sum_{i=1}^Mz_i^nu_i+\sum_{i=M+1}^Db_iu_i \\ \small\text{$(z_1^n,...,z_M^n)$ is different for different $x^n$, $(b_{M+1},...,b_D)$ is the same for all  $x^n$}\)

- Cost function: $\displaystyle J=\frac1N \sum^N_{n=1}\|x^n-\widetilde{x^n}\|^2, \text{where } \widetilde{x^n}=\sum_{i=1}^Mz^n_iu_i + \sum_{i=M+1}^Db_iu_i$ 

- $\text{Let } \begin{cases} \displaystyle \frac \partial {\partial z^n_j} J=0 \\ \displaystyle \frac \partial {\partial b_j}J=0 \end{cases} \Rightarrow \begin{cases} \displaystyle \frac 1 N 2(x^n-\widetilde{x^n})^T (-u_j) = \frac 2 N (z_j-(x^n)^Tu_j)=0 \\ \displaystyle \frac 1 N \sum_{n=1}^N2 (x^n-\widetilde {x^n})^T(-u_j) = \frac 2 N \sum_{n=1}^N (b_j-(x^n)^Tu_j)=0 \end{cases}$ 

$\Rightarrow \begin{cases} z_j=(x^n)^Tu_j & j\in\{1,...,M\}\\ b_j=\overline{x}^Tu_j & j\in\{M+1,...,D\} \end{cases}$ 

Noticing $\displaystyle (x^n)^Tu_j=(\sum_{i=1}^D\alpha_i^nu_i^T)u_j = a_j \Rightarrow a_j = (x^n)^Tu_j$ 

$\displaystyle \Rightarrow x^n-\widetilde{x^n} = \sum_{i=M+1}^D[(x^n-\overline x)^Tu_i]u_i$ 

- \begin{align} \boldsymbol \Rightarrow \displaystyle J &= \frac 1 N \sum_{n=1}^N\space \left( \sum_{i=M+1}^D [(x^n-\overline x)^Tu_i]u_i\right)^T \left(\sum_{i=M+1}^D [(x^n-\overline x)^Tu_i]u_i\right) \\ &= \frac 1 N \sum_{n=1}^N \left( \sum_{i=M+1}^Du_i^T ((x^n-\overline x)^Tu_i)\right) \left( \sum_{i=M+1}^D ((x^n-\overline x)^Tu_i)u_i \right) \\ &= \frac 1 N \sum_{n=1}^N \sum_{i=M+1}^D u_i^T(x^n-\overline x)^Tu_iu_i^T(x^n-\overline x)u_i & u_i \text{ orthogonal to each other} \\ &= \sum_{i=M+1}^D u_i^T \left( \frac 1 N \sum_{n=1}^N(x^n-\overline x)^T(x^n-\overline x)\right) u_i & \|u_i\|=1 \\ \end{align} 

$ \displaystyle \boldsymbol \Rightarrow J =  \sum_{i=M+1}^D u_i^TSu_i \text{, where }S=\frac 1 N\sum_{n=1}^N(x^n-\overline x)^T(x^n-\overline x)$

- Lagrangian to Minimize $J$: 

- $\displaystyle L(u_{M+1},...,u_D,\lambda_{M+1},...,\lambda_D) = \sum_{i=M+1}^Du_i^TSu_i + \sum_{i=M_1}^D \lambda_i(1-u_i^Tu_i)$ 

	\(\text{constraint $\|u_i\|=1$ to prevent $u_i=0$}\)

\(\text{For each }u_i, \displaystyle \frac \partial {\partial u_i}L=2Su_i-2\lambda_iu_i=0 \\ \Rightarrow Su_i=\lambda_iu_i \\ \Rightarrow \text{To minmize $J$, take eigenvectors with the first $(D-M)$ small eigenvalue orthogonal to (out of) subspace} \\ \Leftrightarrow \text{define subspace with eigenvectors with the first $M$ large eigenvalue} \)

- \begin{align}\displaystyle \text{Intuition: }\widetilde{x_n}&=\sum_{i=1}^M((x^n)^Tu_i)u_i+\sum_{i=M+!}^D(\overline {x}^Tu_i)u_i \\ &= \overline x + \sum_{i=1}^M[(x^n-\overline x)^Tu_i]u_i \end{align}

1. Singular Value Decomposition - SVD:

- Intorduce matrix $A_{m\times n}$ 

- $(A^TA)_{n\times n}$ symmetric matrix ($\text{actually, Gram matrix} \rightarrow \text{semi-definite}$) 
- \begin{gather} \text{eigenvalue decomposition: } \\ A^TA=VDV^T\text{, $V$ is normalized ($v_i^Tv_i=1$) with column as eigenvector} \end{gather}

- $AV=(Av_1,...,Av_n)_{m\times n}$ 

- Let $r(A)=r$

\begin{align} \Rightarrow & r(A^TA)=r(A)=r \space \\&  r(AV)=\min\{r(A),r(V)\}=\min\{r,n\}=r\end{align}

- Reduce $AV$ to basis $(Av_1,...,Av_r)$  

- Let \(\displaystyle U=(u_1,...,u_r) = (\frac {Av_1} {\sqrt {\lambda_1}},...,\frac{Av_r}{\sqrt{\lambda_r}}) \space \text{, $\lambda_i$ is $i$-thh eigenvalue of $A^TA$}\) 

- Orthogonal: $\forall i\neq j, u_i^Tu_j=\frac 1 {\sqrt{\lambda_i\lambda_j}}v_i^TA^TAv_j=\frac {\lambda_j} {\sqrt{\lambda_i\lambda_j}}v_i^Tv_j=0$ 

- Unit: $\|u_i\|=\frac {\|Av_i\|} {\sqrt{\lambda_i}}=\frac {\sqrt {<Av_i,Av_i>}} {\sqrt{\lambda_i}}=1$ 

$\boldsymbol \Rightarrow U \text{ is standard orthogonal (orthonormal) basis}$ 

- $AV=U\Sigma,\text{ where } \Sigma = D^{\frac 1 2}$ 

- Expand $U$ to orthonormal in $\mathbb R^m: \space (u_i,...,u_m)$ 

- Epand corresponding part in $\Sigma$ with $0$ 

- \(A = U\Sigma V^T,\text{ with singular value in $\Sigma$ in decreasing order}\)

2. SVD with PCA:

- $X$ is data matrix in row ($\text{centered - zero mean}$)

- Eigenvectors of corvariance matrix $S=X^TX$ are in $V, \text{ where }X = U\Sigma V^T$ 

- When using $S=U\Sigma V^T \Rightarrow \space U=V \space \land \space S=V\Sigma V^T$ 

	$\text{reduced to eigenvalue decomposition}$ 

- $S=VDV^T \text{ with $V$ orthonormal}$: 

Eigenvalues $\lambda$ of covariance matrix $S$ express the variance of data set $X$ in direction of corresponding eigenvectors

- Projection:

- $\widetilde X = XV_M, \text{ where $V_M$ contains first M-large eigenvectors}$ 
- Projection direction is **not** unique 

3. Reconstruction (approximate): 

- Data is projected onto $k$ dimension using $\text{SVD}$ with $S = U\Sigma V^T$ 
- $x_{approx} = U_{reduce} \cdot z$,  $U_{reduce}$ is n*k matrix, $z$ is k*1 vector
- ![Reconsturction from data Compression](../../Machine%20Learning/Statistical%20Machine%20Learning/Reconsturction%20from%20data%20Compression.png) 

4. Choosing $k$ (num of principal components):

- choose the **smallest** k making $\displaystyle \frac JV \leq 0.01$ => 99% of variance is retained 

- $[U,S,V]$ = svd(Sigma) => $\displaystyle \frac JV=1-\frac {\sum^k_{i=1}S_{ii}}{\sum^n_{i=1}S_{ii}}$, $S$ is diagonal matrix

=> check $\frac JV$ before compress data 

5. Data Preprocessing:

- PCA $vs.$ Normalization:
- Normalization: Individually normalized but still correlated
- PCA: create decorrelated data $-\text{ whitening}$ 
- Whitening: $\text{ projection with normalization}$ 
- $S = VDV^T \text{, where $S$ is Gram matrix over $X^T$}$ 
- \(\forall n, y_n=D^{-\frac12}V^T(x^n-\overline x) \text{, where $\overline x$ is the mean of $X$}\) 

\begin{align} \Rightarrow & \text{{$y^n$} has zero mean} \\ & \displaystyle cov(\{y^n\}) = \frac 1 N \sum_{n=1}^Ny_ny_n^T= D^{\frac {-1} 2}V^TSVD^{\frac{-1}2}=I \end{align}

6. Tips for PCA:

- Do NOT use PCA to prevent overfitting, use regularization instead
- Try original data before implement PCA
- Train PCA only on trainning set

\subsection{Independent Component Analysis (ICA)}

1. Goal:
- Recover original signals from a mixed observed data
- Source signal $S\in \mathbb R^{N\times K}$; mixing matrix $A$; Observed data $X=SA$
- Maximizes statistical independence
- Find $A^{-1}$ to maximizes independence of columns of $S$
2. Assumption: 
- At most one signal is Gaussian distributed
- Ignorde amplitude and order of recovered signals
- Have at least as many observed mixtures as signals
- $A$ invertible
3. Independence $vs.$ Uncorrelatedness
- Independence $\Rightarrow$ Uncorrelatedness
- $p(x_1,x_2)=p(x_1)p(x_2) \Rightarrow \mathbb E(x_1x_2)-\mathbb E(x_1)\mathbb E(x_2) = 0$ 
4. Central Limit Theorem
5. FastICA algorithm

\subsection{t-SNE}

1. Problem \& Focus
2. Compared to PCA:
- No whitening function to use for new data
- PCA can only capture linear structure inside the data
- t-SNE preserves the <u>local distances</u> in the original data

\subsection{Anomaly Detection}

1. Problem to solve:

- Given dataset {$x^1,x^2,...,x^m$}, build density estimation model $p(x)$
- $p(x^{test}<\epsilon)$ => $x^{test}$ anomaly 

2. Hypothesis function: 

- $\displaystyle p(x)=\prod^n_{i=1} p(x_i), \space x \in R^n,\forall i \in [1,n], \space x_i \sim N(\mu_i,\sigma_i^2)$ 
- $\displaystyle \mu=\frac1m \sum^m_{i=1}x^i,\space \sigma^2=\frac1m \sum^m_{i=1}(x^i-\mu)^2$ 
- assume $x_1,...,x_n$ independent from each other

3. Multivariate Gaussian:

- $\displaystyle p(x;\mu,\Sigma)=\frac1{(2\pi)^{\frac n2} |\Sigma|^{\frac 12}} exp(-\frac12 (x-\mu)^T \Sigma^{-1} (x-\mu)),$  

$x\in R^n, \mu\in R^n,\Sigma\in R^{n\times n}$, where $\Sigma$ is covariance matrix

- $\displaystyle \mu=\frac1m \sum^m_{i=1}x^i,\space \Sigma=\frac 1m \sum^m_{i=1}(x^i-\mu)(x^i-\mu)^T$ 
- $x_1,...x_n$ can be correlated but **not** linearly dependent
- need $m > n$ $(m\ge10n\space suggested)$ or elas $\Sigma$ non-invertible

4. Algorithm:

- choose features
- compute $\mu$, $\sigma$
- compute $p(x)$ for new example, anomaly if $p(x) < \epsilon $ 

5. Evaluation (real-number):

- Labeled data into normal/anomalous set

(okay if some anomalies slip into normal set)

- training set: unlabeled data from normal set (60%)
- CV set: labeled data from normal (20%) & anomalous (50%) set
- test set: labeled data from normal (20%) & anomalous (50%) set

- Use evaluation metrics (skewed data)

6. When to use:

- Anomaly detection:
- Very small num of positive data (0-20 commonly); Large num of negative data
- Difficult to learn from positive data (not enough data, too many features...)
- Future anomalies may look nothing like given data
- Supervised Learning:
- Larger num of positive \& negative data
- Enough positive data for algorithm to learn
- Future positive example is likely to be similar to given data

7. Example:

- Anomaly detection:
- Fraud detection, Manufacturing, Monitoring machines in data center...
- Supervised learning:
- Email spam classification (enough data), Weather prediction (sunny/rainy/etc), Cancer classification...

8. Tips:

- Non-guassian feature: transformation / using other distribution
- Choosing features: compare anomaly data with normal data


\subsection{Recommender System}

1. Problem Formulation:

- $r_{i,j}=1$ if item $i$ is rated by user $j$ 

- $y_{i,j}$ = rating of item $i$ given by user $j$ 

- $\theta^j$ = parameter vector for user $j$ 

- $x^i$ = feature vector for movie $i$ 

=> for user $j$, movie $i$, ($r_{i,j}=0$), predict rating $x^i\theta^j$

2. Content Based Recommendations:

- Treat each user as a seperate linear regression problem with the feature vectors of its rated items as traning set

**Assume features for each items ($x^i$) are available and known**

=> given $X$ estimate $\Theta$ 

- Cost Function for $\theta_j$: 

$\displaystyle J(\theta^j)= \frac1 {2} \sum_{i:r_{i,j}=1}(x^i\theta^j-y_{i,j})^2+\frac \lambda {2} \sum_{k=1}^n(\theta^j_k)^2, \theta^j \in R^{n+1} (\theta_0 \text{ not regularized)}$

- Cost Function for $\Theta$:

$\displaystyle J(\Theta)= \frac1{2} \sum_{j=1}^{n_u} \sum_{i:r_{i,j}=1}(x^i\theta^j-y_{i,j})^2+\frac \lambda {2} \sum_{j=1}^{n_u} \sum_{k=1}^n(\theta^j_k)^2, \\ \theta^j \in R^{n+1} (\theta_0 \text{ not regularized)}, \space n_u \text{ is num of users}$ 

- Update Rule: \(\forall \theta^j_k \in \theta^j, \theta^j_k := \theta^j_k-\alpha\large\frac{\partial J(\Theta)}{\partial\theta^j_k}\), \(\displaystyle \frac{\partial J(\Theta)}{\partial\theta^j_k}=\sum_{i:r_{i,j}=1}(x^i\theta^j-y_{i,j})x_k^i+ \lambda \theta_k^j, \space \text{for $k \neq 0$ ($\theta^j \in R^{n+1}$)}\) 

3. Collaborative Filtering

- \textbf{Assume preference of each users ($\theta^j$) are available and known}

=> given $\Theta$ estimate $X$ 

- Cost Function for $x^i$: $\displaystyle J(x^i)=\frac 1 2 \sum_{j:r_{i,j}=1} (x^i\theta^j - y_{i,j})^2 + \frac \lambda 2 \sum ^n_{k=1} (x_k^i)^2$ 
- Cost Function for $X$: $\displaystyle J(X)=\frac 1 2 \sum_{i=1}^{n_m} \sum_{j:r_{i,j}=1} (x^i\theta^j - y_{i,j})^2 + \frac \lambda 2 \sum_{i=1}^{n_m} \sum ^n_{k=1} (x_k^i)^2 \\ x^j \in R^{n+1} (x_0 \text{ not regularized)}, \space n_m \text{ is num of items}$ 
- Update Rule: \(\forall x^i_k \in x^i, x^i_k := x^i_k-\alpha\large\frac{\partial J(X)}{\partial x^i_k}\), \(\displaystyle \frac{\partial J(X)}{\partial x^i_k}=\sum_{j:r_{i,j}=1}(\theta^jx^i-y_{i,j})\theta_k^j+ \lambda x_k^i, \space \text{for $k \neq 0$ $(x^i \in R^{n+1}$)}\)

- Basic Idea: 

- Randomly initialize $\Theta$ 

- loop:

	Estimate $X$ 

	Estimate $\Theta$ 

- Cost Function: 

$\displaystyle J(X,\Theta) = \frac 1 2 \sum_{(i,j):r_{i,j}=1}(x^i\theta^j - y_{i,j})^2 + \frac \lambda 2 \sum_{i=1}^{n_m}\sum_{k=1}^n(x_k^i)^2 + \frac \lambda 2 \sum_{j=1}^{n_u}\sum_{k=1}^n (\theta_k^j)^2, \space x \in R^n, \space \theta \in R^n$ 

(the sum term in $J(\Theta)$, $J(X)$, and $J(X,\Theta)$ is the same)

- Update Rule: 

- $\forall x^i_k \in x^i, x^i_k := x^i_k-\alpha\large\frac{\partial J(X,\Theta)}{\partial x^i_k}$, $\displaystyle \frac{\partial J(X,\Theta)}{\partial x^i_k} = \frac{\partial J(X)}{\partial x^i_k} =\sum_{j:r_{i,j}=1}(\theta^jx^i-y_{i,j})\theta_k^j+ \lambda x_k^i, \space x^i \in R^n$ 
- $\forall \theta^j_k \in \theta^j, \theta^j_k := \theta^j_k-\alpha\large\frac{\partial J(X,\Theta)}{\partial\theta^j_k}$, $\displaystyle \frac{\partial J(X,\Theta)}{\partial\theta^j_k} = \frac{\partial J(\Theta)}{\partial\theta^j_k} = \sum_{i:r_{i,j}=1}(\theta^jx^i-y_{i,j})x_k^i+ \lambda \theta_k^j, \space \theta^j \in R^n$ 

- \textbf{Algorithm}

- Initialize $X, \Theta$ to **small random values**

=> for symmetry breaking (similar to random initialization in neural network) 

=> so that algorithm learns features $x^1,...,x^{n_m}$ that are different from each other

- Minimize $J(X,\Theta)$ 

- Predict $y_{i,j} = x^i\theta^j$ ($Y = X\Theta$)

- Finding Related Item to Recommend

- $||x^i-x^j||$ is samll => item $i$ and $j$ is similar

- Mean Normalization:

- Problem: if user $j$ hasn't rated any movie, $\theta^j = [0,...,0]$  

=> predicted rating of user $j$ on all item $=0$ 

=> useless prediction

- Algorithm (row version):

	compute vector $\mu, \space \forall \mu_i \in \mu, \mu_i = \text{mean of $Y_i$, where $Y_i$ is the $i^{th}$ row in $Y$}$ 

	manipulate $Y$: $\forall y_{i,j} \in Y \land r_{i,j}=1, \space y_{i,j} -= \mu_i$  => the mean of each row in $Y$ is $0$ 

	predict rating for user $j$ on item $i = x^i\theta^j + \mu_i$ 

- For item $i$ with no rating

=> apply column version of mean normalization

(but user with no rating is generally more important)

\section{Large Scale Machine Learning}

- Compute $cost(\theta,(x^i,y^i))$ before updating 

For every $k$ update iterations, plot average $cost(\theta,(x^i,y^i))$ over the last $k$ examples

- Checking curves:

Increasing $k$ result in smoother line and less noise, but the result is more delayed

\subsection{Online Learning}

1. Situation:
- Has too many data (can be considered as infinite)
- When data comes in as a continuous stream
- Can adapt to changing user preference
2. Procedure:
- Use one example only once (Similar to stochastic gradient decent in this sense

\subsection{Map-reduce}

1. In Batch Gradient Descent:

- Update rule $\displaystyle \theta_j = \theta_j - \alpha \frac 1 m \sum^m_{i=1} (h_\theta(x^i)-y^i)x_j^i$ 
- Parallelize the computation of $\displaystyle \sum^m_{i=1} (h_\theta(x^i)-y^i)x_j^i$ by dividing the data set into multiple sections

2. Ability to reduce:

- Contain operation over the whole data set

(Neural Network can be map-reduced)









