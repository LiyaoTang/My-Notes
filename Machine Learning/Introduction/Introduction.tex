\chapter{Introduction}

\section{General Concern}
\subsection{Types of Learning}

\subsubsection{Supervised Learning}
\begin{itemize}
\item Overview
	\begin{itemize}
	\item training data comprises examples of input vectors with corresponding target vectors
	\end{itemize}
\item Regression
	\begin{itemize}
	\item output one or more continuous variable
	\end{itemize}

\item Classification
	\begin{itemize}
	\item assign input to one of a finite number of discrete categories
	\end{itemize}
\end{itemize}

\subsubsection{Unsupervised Learning}
\begin{itemize}
\item Overview
	\begin{itemize}
	\item training data consists of a set of input
	vectors without target vectors
	\end{itemize}
\item Clustering
	\begin{itemize}
	\item Goal: discover groups of similar examples
	\end{itemize}
\item Density Estimation
	\begin{itemize}
	\item Goal: determine the distribution of data within the input space
	\end{itemize}
\item Dimension Reduction
	\begin{itemize}
	\item Goal: project data into low dimension, for the purpose of such as visualization
	\end{itemize}
\end{itemize}

\subsubsection{Reinforcement Learning}
\begin{itemize}
\item Overview
	\begin{itemize}
	\item input with time series \& discover optimal output by a process of trial and error
	\end{itemize}
\item Goal
	\begin{itemize}
	\item find actions to take under given circumstance to maximize a reward
	\end{itemize}
\end{itemize}

%\subsection{Uncertainty}
%\subsubsection{Cause}
%\begin{itemize}
%
%\end{itemize}

%%%%%%%%%%%%%%%%%%%%%%%%%%%%%%%%%%%%%%%%%%%%%%%%%%%%%%%%%%%%%%%%%%%%%%%%%%%%%%%%%%%%%%%%%%%%%%%%%%%%%%%%%%%%%%%%%%%%%%%%%%%%%%%%
\section{Decision Theory}
\subsection{}
 
\section{Information Theory}
 


%%%%%%%%%%%%%%%%%%%%%%%%%%%%%%%%%%%%%%%%%%%%%%%%%%%%%%%%
\section{Recommended Practice}
\subsection{Data \& Dataset}
\subsubsection{Train-Val-Test}
\begin{itemize}
\item Reason
	\begin{itemize}
	\item Iterative Process
		\begin{figure}[!ht]
		\includegraphics[width=0.3\linewidth, center]{"./Deep Learning/experiment practice-iterative process".png}
		\end{figure}
		\begin{itemize}
		\item intuition usually do NOT transfer across domains (NLP, CV, Search, etc.)
		\item do NOT hope to have the correct hyperparameters at the first try
		\end{itemize}
		$\Rightarrow$ need feedback from experiment result \\
		$\Rightarrow$ make sure the feedback is CORRECT and FAST
	\end{itemize}
\item Recommended Usage
	\begin{itemize}
	\item Splitting 
		\begin{itemize}
		\item classic split for small dataset $\Rightarrow$ train:val:test $= 80:20:20$, or K-fold
		\item in big data (e.g. 100 million) $\Rightarrow$ train:val:test $= 98:1:1$
		\end{itemize}
		(as long as val-test sets cover enough data variance)
	\item Training Set
		\begin{itemize}
		\item data used to find the model parameter estimation (used for learning process of model) \\
		$\Rightarrow$ over-fit by complex model
		\end{itemize}
	\item Validation Set (Val)
		\begin{itemize}
		\item data to indicate generalization ability of a range of trained models \\
		$\Rightarrow$ for model comparison, selection \& hyperparameters tunning \\
		(feedback is correct if enough various input covered)
		\end{itemize}
	\item Test Set
		\begin{itemize}
		\item evaluate the \textbf{generalization ability} of final selected \& trained model
		(indication correct if enough various input covered)
		\end{itemize}
	\end{itemize}
	
\item Special Usage
	\begin{itemize}
	\item No Val Set
		\begin{itemize}
		\item may use the "test" set as val set $\Rightarrow$ generalization ability NOT reported
		\end{itemize}
	\item Training on Train-Val Set
		\begin{itemize}
		\item to utilize as many data as possible for ultimate performance \\ 
		$\Rightarrow$ okay case of "no val set"
		\end{itemize}
	\end{itemize}

\item Potential Problem
	\begin{itemize}
	\item Mismatched Distribution across Sets
		\begin{itemize}
		\item classic supervised learning assumption: all sets drawn from SAME distribution
		\item other (e.g. transfer/adaptive) learning focus on violation of such assumption
		\end{itemize}
		$\Rightarrow$ yet, make sure val\&test set from the SAME distribution as the desired one
	\item \textbf{Overfitting Val Set}
		\begin{itemize}
		\item iteratively tunning model is a processing of learning (fitting to the val set) \\
		$\Rightarrow$ with enough iteration, val set can be overfitted
		\item may consider test set as $2^\text{nd}$ val set, and further have $3^\text{rd}, 4^\text{th}...$ val sets
		\end{itemize}
	\item Limited Data
		\begin{itemize}
		\item better model $\Rightarrow$ more training data 
		\item $\Rightarrow$ less validation $\Rightarrow$ noisy estimation of generalization ability
		\end{itemize}
	\end{itemize}
\end{itemize}

\subsubsection{Train-Test}
\begin{itemize}
\item K-fold Cross Validation
	\begin{itemize}
	\item Procedure
		\begin{itemize}
		\item split all data into $K$ folds, $K-1$ folds for train, $1$ for validation
		
		\item $\Rightarrow$ average over all $C^1_K$ combination to indicate the generalization ability
	
		\item extreme case: leave-out-one $\Rightarrow K=N$, where $N$ is number of all data
		\end{itemize}
	
	\item Cons:
		\begin{itemize}
		\item $\mathcal O(K) \Rightarrow$ slow, especially if training process already slow \\
		$\Rightarrow$ trade off between time vs. constraint on validation
		\item hence, \textbf{not} often used in big data era
		\end{itemize}
	\end{itemize}
\end{itemize}

\subsubsection{Data Augmentation}

\subsubsection{Data Preprocessing}
\begin{itemize}
\item Mean Centering
	\begin{itemize}
	\item Practice
		\begin{itemize}
		\item for all training examples, compute mean (on each features) $\displaystyle \mu=\frac 1 N \sum_{n=1}^N{\mathbf x_n}$, \\
		where $\{\mathbf x_1,...,\mathbf x_N\}=X_\text{train}$ the training set
		\item preprocess each $\mathbf x \in X_\text{train}, X_\text{val}, X_\text{test}$ to be $\mathbf x'=\mathbf x -\mu$
		(all data go through the same process)
		\end{itemize}
	\item Pros
		\begin{itemize}
		\item (training) data has a zero mean (statistically, most data close to $0$)
		\end{itemize}
	\item Cons
		\begin{itemize}
		\item different features may reside in various scales
		\end{itemize}
	\end{itemize}
\item Standardizing
	\begin{itemize}
	\item Practice
		\begin{itemize}
		\item compute mean $\mu$, standard deviation $\displaystyle \sigma=\left( \frac 1 N \sum_{n=1}^N{(\mathbf x_n-\mu)^2} \right)^{1/2}$, \\
		where $\{\mathbf x_1,...,\mathbf x_N\}=X_\text{train}$ the training set
		\item preprocess each $\mathbf x \in X_\text{train}, X_\text{val}, X_\text{test}$ to be $\mathbf x'=\frac{\mathbf x -\mu} \sigma$ \\ 
		(all data go through the same process)
		\item note: with big data, usually computed iteratively due to limited memory
		\end{itemize}
	\item Pros
		\begin{itemize}
		\item (training) data has zero mean \& unit variance \\
		$\Rightarrow$ approximated to normal distribution
		\item for deep learning: different features in same small range close to $0$ \\
		$\Rightarrow$ weights for different features are in roughly the same scale \\
		$\Rightarrow$ easier to train
		\end{itemize}
	\end{itemize}
\end{itemize}

\subsection{Project Structuring}

%%%%%%%%%%%%%%%%%%%%%%%%%%%%%%%%%%%%%%%%%%%%%%%%%%%%%%%%%%%%%%%%%%%%%%%%%%%%%%%%%%%%%%%%%%%%%%%%%%%%%%%%%%%%%%%%%%%%%%%%%%%%%%%%%%%%%
\section{Model Analysis}

\subsection{Bias and Variance}
\subsubsection{Overview}
\begin{itemize}
\item Low Bias, High Variance (Over-fitting)
	\begin{itemize}
	\item Symptom
		\begin{itemize}
		\item good performance on training set \& poor generalization \\
		(good on train; bad on val)
		\item $\Rightarrow$ good at fitting training set; bad at representing modeling underlying data source
		\end{itemize}
	\item Cause
		\begin{itemize}
		\item too much representation ability (to fit even the noise)
		\item directly model the likelihood instead of posterior
		\end{itemize}
	\item Remedy
		\begin{itemize}
		\item larger dataset
		\item regularization (model the posterior by accounting prior)
		\end{itemize}
	\end{itemize}
\item High Bias, Low Variance (Under-fitting)
	\begin{itemize}
	\item Symptom
		\begin{itemize}
		\item bad at fitting training examples \& modeling underlying data source \\ 
		(bad at train \& val)
		\item $\Rightarrow$ poor performance on training set \& good generalization (though meaningless)
		\end{itemize}
	\item Cause
		\begin{itemize}
		\item lack of representation ability (not enough flexibility)
		\end{itemize}
	\item Remedy
		\begin{itemize}
		\item try model with better representative ability (more complexity, flexibility)
		\end{itemize}
	\end{itemize}
\item High Bias, High Variance (Over\&Under-fitting)
	\begin{itemize}
	\item Symptom
		\begin{itemize}
		\item bad at fitting some general cases; while good at some rare and special cases \\
		(especially in high dimensional space)
		\end{itemize}
	\item Cause
		\begin{itemize}
		\item model probably not suitable for the dataset
		\end{itemize}
	\item Remedy
		\begin{itemize}
		\item switch to other types of model
		\item dataset preprocessing
		\end{itemize}
	\end{itemize}
\item Low Bias, Low Variance
	\begin{itemize}
	\item Behavior
		\begin{itemize}
		\item good at fitting training examples \& modeling underlying data source \\ 
		(good at train \& val)
		\item $\Rightarrow$ good performance on training set \& good generalization \\
		("good"= close to base performance, e.g. human ability in supervised learning)
		\end{itemize}
	\end{itemize}
\end{itemize}

\subsubsection{Guideline}
\begin{itemize}
\item Solving High Bias
	\begin{itemize}
	\item Increasing Model Capability
		\begin{itemize}
		\item increase complexity: more weights, latent variable / hidden layer etc.
		\item use more suitable model specifically designed for the data (e.g. CNN for image)
		\end{itemize}
	\end{itemize}
	$\Rightarrow$ until fitting training set well
\item Solving High Variance
	\begin{itemize}
	\item Data Augment
		\begin{itemize}
		\item get/simulate more training data (via crowd sourcing, distortion, GANs, etc.)
		\end{itemize}
	\item Model Regularization
		\begin{itemize}
		\item control the complexity of model (e.g. L0/1/2 normalization)
		\end{itemize}
	\end{itemize}
\item Solving Trade-off
	\begin{itemize}
	\item Iterative Process
		\begin{itemize}
		\item solve bias, then solve variance, iteratively
		\end{itemize}
	\item Complexity + Data/Regularization
		\begin{itemize}
		\item increase complexity to solve bias \\
		without hurting variance (via more data/regularization)
		\item more data/regularization to solve variance \\
		without hurting bias (with enough complexity)
		\end{itemize}
	\end{itemize}
\end{itemize}

L2 regularization also called "weight decay", as in gradient decent, weight is multiplied by a $<1$ number due to L2 term 

2. Interaction with regularization:

- Improper $\lambda$:
- large $\lambda$ => high bias
- small $\lambda$ => high variance
- Choosing $\lambda$:
- try $\lambda=0,0.01,0.02,0.04,...,10$
- select the model with lowest $J_{cv}(\theta)$ without regularization term

3. Interaction with training set size:

- Normal Learning curve:

![Normal learning curve](../../Machine%20Learning/Statistical%20Machine%20Learning/Normal%20learning%20curve.png) 

- Learning curve with high bias:

- where getting more training data **doesn't** help

![Learning curve with high bias](../../Machine%20Learning/Statistical%20Machine%20Learning/Learning%20curve%20with%20high%20bias.png) 

- Learning curve with high variance:

- where getting more training data **helps**

![Learning curve with high variance](../../Machine%20Learning/Statistical%20Machine%20Learning/Learning%20curve%20with%20high%20variance.png) 

4. Ways to fix:

- High bias:
- more features / more polunomial terms of features
- decreasing $\lambda$

- High variance:

- larger data set
- fewer features
- increasing $\lambda$

- **In neural network:**

- High bias => larger neural networks (more hidden layers / more units in one layer)

- High variance => smaller neural networks

**Larger network with regularization ($\lambda$) is more powerful**

\subsection{Evaluating Hypothesis}

1. Data set => training set (60%), cross validation set (20%), test set (20%)

- randomly split

2. Cross Validation:

- estimation of the generalization error, inaccurate because:

- finite trainning set
- finite corss validation set

- S-fold Cross Validation:

- devide whole set of data into S sets

- for $i \in (1,S)$ 

	choose the $i^{th}$ set as cross validation (or test set in some cases)

	rest of the sets are called traning set

	run the procedure (mentioned later) on the traning set

	estimate generalization in CV set 

3. Test set error

- Linear regression:  $\displaystyle J_{test}(\theta)=\frac1{2m_{test}}\sum^{m_{test}}_{i=1}(h_\theta(x^i_{test})-y^i_{test})^2$ 
- Logistic regression: $\displaystyle J_{test}(\theta)=-\frac1{m_{test}}\sum^{m_{test}}_{i=1}(y_{test}^i log\space h_\theta (x_{test}^i) + (1-y_{test}^i)log\space h_\theta(x_{test}^i))$ 
- Misclassfication error: $\displaystyle J_{test}(\theta) = \frac1 {m_{test}}\sum_{i=1}^{m_{test}}err(h_\theta(x_{test}^i), \space y_{test}^i)$ 
- \begin{gather} err(h_\theta(x), \space y) = \begin{cases}1 & \text{if }h_\theta(x) \geq 0.5 \and y=0 \text{ or $h_\theta(x) < 0.5 \and y=1$} \\ 0 & \text{otherwise} \end{cases} \end{gather}

4. Choosing procedure:

- Minimize traning error $J_{train}(\theta)$ 
- Select a model with lowest $J_{cv}(\theta)$ 
- Estimate generalization error as $J_{test}(\theta)$ 

\subsection{Error Analysis}

1. Procedure:
- Algorithm (trained) misclassifies $n$ data in cross validation set
- Classify these $n$ data and rank them
- Maybe more features are found
2. Feature selection => Numerical evaluation
- => test algorithm with / without this feature on **CV set** (compare error rate)

\subsection{Skewed classes}

1. Precision / Recall

- |                 | **Actual 1**   | **Actual 0**   |
| --------------- | -------------- | -------------- |
| **Predicted 1** | True positive  | False positive |
| **Predicted 0** | False negative | True negative  |

- \textbf{Precision} = $\displaystyle \frac{\text {True positive}}{\text{Predicted positive}} = \frac{\text {True positive}}{\text{True pos + False pos}}$

- **Recall** = $\displaystyle \frac{\text{True positive}}{\text{Actual positive}} = \frac{\text{True positive}}{\text{True pos + False neg}}$

2. Evaluation with precision/recall

- Predict 1 if $ h_\theta(x) \geq \epsilon$, 0 if $h_\theta(x) < \epsilon$

- larger $\epsilon$ => higher precision, lower recall $\small \text{(more confident)}$ 
- smaller $\epsilon$ => lower pecision, higher recall $\small \text{(avoid missing)}$       

![Posiible Precision -Recall curev](../../Machine Learning/Statistical 0Machine Learning/Posiible Precision -Recall curev.png)  

3. Compare precision/recall num

- $\displaystyle \text{F}_1 \space  Score = 2\frac{PR}{P+R}$, $P$ as precision, $R$ as recall
- higher better, on cross validation set

4. High precision \& high recall:

- **large num of features $\small\text{(low bias)}$ + large sets of data $\small\text{(low variance)}$**


\section{Supervised Learning}
\begin{itemize}

\item Feature normalization: $\forall x_{ij} \in X, x_{ij}=\frac{x_{ij}-\mu_j}{\sigma_j^2}$, $ X:[instance][feature]$, without $[1...1]^T$ in 1st column $X=[x_1,x_2,...,x_m]$, m instances in total
\item Regularization: add penalty for $\theta$ being large into cost function
\item $\displaystyle J(\theta)= \space... + \frac{\lambda}{2m}\sum^n_{j=1}\theta_j^2$ , \textbf{bias $\theta_0$ shouldn't be penalized} 

\end{itemize}

\section{Linear Regression}

\begin{itemize}

\item Notation
	\begin{itemize}
	\item $t$: observed data
	\item $\displaystyle y(\mathbf x,\mathbf w)=\sum_{i=0}^{M}\phi_i(\mathbf x)w_i = \mathbf w^T \phi(\mathbf x)$ : model generating ground truth, with
		\begin{itemize}
		\item $\mathbf w$: weight vector
		\item $\phi(\mathbf x)$: basis function for feature vector $\mathbf x$, with usually $\phi_0(\mathbf x) = 1$ as bias
		\end{itemize}
	\end{itemize}

\item Assumption
	\begin{itemize}
	\item Deterministic Model with Observation Noise
		\begin{itemize}
		\item \(t=y(x,w)+\epsilon\), where
		
		$\epsilon \sim \mathcal N(0,\beta^{-1})$ is Gaussian noise where precision (inverse variance) $\beta$
		\item $\Rightarrow$ consequence
			\begin{enumerate}
			\item $\text{likelihood } p(t|\mathbf x,\mathbf w,\beta) = \mathcal N(t|y(\mathbf x,\mathbf w), \beta^{-1})$
			\item $\mathbb E[t|\mathbf x] = \int t\cdot p(t|\mathbf x) dt = y(\mathbf x, \mathbf w) $
			\item unimodal distribution $p(t|\mathbf x) \Rightarrow$ extended by conditional mixture model
			\end{enumerate}

		\end{itemize}
	\end{itemize}

\item Joint Likelihood

	\begin{itemize}
	\item $\displaystyle P(\mathbf t|\mathbf X,\mathbf w,\beta) = \prod_{n=1}^N \mathcal N(t_n|\mathbf w ^T\phi(\mathbf x_n), \beta^{-1})$, where
		\begin{itemize}
		\item $\mathbf X = \{ \mathbf x_1,...,\mathbf x_N \}, \mathbf t = \{ t_1,...,t_N \}$
		\end{itemize}
	\item Log Likelihood 
		\begin{itemize}
			\item $\displaystyle \ln P(\boldsymbol y|X,\theta,\beta) = \frac N 2\ln\beta - \frac N 2 \ln(2\pi) - \beta \frac 1 2\sum_{i=1}^m (h_\theta(x^i) - y^i)^2$
		\end{itemize}
	 
	\end{itemize}

\item Log Posterior leads to regularization

	\begin{itemize}
	\item Maximizing the likelihood function $\Rightarrow$ (often) excessively complex models \& over-fitting
	\item Regularization term comes from the $\text{Prior}$: 
		\begin{itemize}
		\item $\text{assume Prior } p(\theta) =  \mathcal{N}(\theta|0,\alpha^{-1}I) \text{\small , so that Posterior \& Prior are of the same distribution}$ \\
		$\text{to maximize log Posterior}:$ \\ 
		$\displaystyle \Rightarrow \ln p(\theta|X) \propto -\frac \beta 2 \sum_{i=1}^n (y^i - h_\theta(x))^2 - \frac \alpha 2 \theta^T\theta + const$
		
		\item If $\alpha \to 0 \text { (Prior is most useless)}$, maximise $\text{Posterior}$ is equivalent to maximizing likelihood 
		\item Maximize $\text{Posterior} \Leftrightarrow$ Minimize $\text{cost function with regularization}$, where $\lambda = \alpha/\beta$ 
		\end{itemize}
	\end{itemize}
		
\item Predictive Distribution: $p(y|x,X,Y)$ 

	\begin{itemize}
	\item $\displaystyle p(y|x,X,Y) = \int p(y,\theta|x,X,Y)d\theta = \int p(y|\theta,x,X,Y)p(\theta|x,X,Y)d\theta$ 
	
	\item $p(y|\theta,x,X,Y)=p(y|\theta,x) = \mathcal{N}(y|h(x,\theta), \beta^{-1})$ \\ 
	\(\small  \text{based on assumption: } y = y(x,\theta)+\epsilon \text{, where $\epsilon$ is Guassian noise}\) \\ 
	\(p(\theta|x,X,Y)=p(\theta|X,Y) = \text{posterior} \)
	
	\item  $ \displaystyle \Rightarrow p(y|x,X,Y)=\int p(y|\theta,x) p(\theta|X,Y)d\theta$   

	\item  $\text{Expected Lost}  = (bias)^2 + variance + noise$ 
	\end{itemize}
		
\item Notation:

	\begin{itemize}
	\item $t=y(x,w)+\epsilon,\text{ where } \epsilon \text{ is Gaussian noise}$ 
	\item $\hat y$ is prediction function to approximate $y=y(x,w)$ 	
	\end{itemize}

\item Procedure:

	\begin{itemize}
	\item $\mathbb E[(t-\hat y)^2] = \mathbb E[t^2-2t\hat y+\hat y^2] \\ = \mathbb E[t^2] +\mathbb E[\hat y^2] -\mathbb E[2t\hat y] \\ = \text{Var}[t]+\mathbb E[t]^2 + \text{Var}[\hat y]+\mathbb E [\hat y]^2 - 2y\mathbb E[\hat y] \\ = \text{Var}[t] + \text{Var}[\hat y] + (y^2 -2y\mathbb E[\hat y] +\mathbb E[\hat y]^2) \\ =  \text{Var}[t] + \text{Var}[\hat y] + (y-\mathbb E[\hat y])^2 \\ =  \text{Var}[t] + \text{Var}[\hat y] + \mathbb E[t-\hat y]^2 \\ = \sigma^2+\text{Var}[\hat y] + \text{Bias[$\hat y$]}^2$ 
	
	$\text{where } \sigma^2 = \text{Var}[\epsilon] \text{ is the noise}$ 
	
	(formula used: $\text{Var}[x] = \mathbb E[x^2] - \mathbb E[x]^2 \Leftrightarrow \mathbb E[x^2] = \text{Var}[x]+ \mathbb E[x]^2$) 
	
	\item Matrix inverse can be evil \& avoid inverse operation: 
	
	$A = U\Lambda U^T  \text{, where $\Lambda$ is diagonal matirx} \\ => A^{-1} = U\Lambda^{-1}U^T \\ \text {but number on the diagonal line of $\Lambda$ can be small => maybe 0 depending on accuracy of computer}$ 
	\end{itemize}


\end{itemize}


\section{Bayesian Regression}

\begin{itemize}

\item Assumption:

- $t=y(x,w)+\epsilon,\text{ where } \epsilon \text{ is Gaussian noise}; y(x,w)\text{ approximated by }\phi(x)w$ 

\item Bayesian view:

\item Gaussian $\text{Prior}: p(w) = \mathcal N(w|m_0,S_0)$ 

	$\text{Reason: to be conjugate}$ 

\item $\text{Likelihood}: \displaystyle p(\boldsymbol t|w) = \prod_{n=1}^N\mathcal N(t_n|w^T\phi(x_n),\beta^{-1}) = \mathcal N(\boldsymbol t|\Phi w,\beta^{-1}I)$ 

\item $\Rightarrow \text{Posterior}: p(w|\boldsymbol t) = \mathcal N(w|m_N,S_N)$ 

	$\text{where } m_N = S_N(S_0^{-1}m_0+\beta \Phi^T\boldsymbol t),\space S_N^{-1} = S_0^{-1}+\beta \Phi^T\Phi $ 

\item Maximum Likelihood:

	\begin{itemize}
	
	\item $\text{Likelihood}: \displaystyle p(\boldsymbol t|w) = \prod_{n=1}^N\mathcal N(t_n|\phi(x_n)w,\beta^{-1})$ 
		\begin{itemize}
		\item meaning: how probable the observed dataset is w.r.t the model setting (under parameter $w$)
		\end{itemize}
	
	\item $\displaystyle \ln \text{Likelihood} = \sum_{n=1}^N [-\ln \frac {\beta} {\sqrt {2\pi}} - \frac \beta 2 (t_n-\phi(x)w)^2]$ 
	
	\item $\displaystyle \frac {\partial} {\partial w} \ln \text{Likelihood} = \beta \Phi^T(\boldsymbol t-\Phi w)$ 
	
	let $\frac {\partial} {\partial w} \ln \text{Likelihood}=0$ 
	
	$\displaystyle \Rightarrow w_{ML} = (\Phi^T\Phi)^{-1} \Phi^T\boldsymbol t $ 
	
	\item $\displaystyle \frac {\partial} {\partial \beta} \ln \text{Likelihood} = -N\beta^{\frac 1 2} + \beta^{\frac 3 2}(\boldsymbol t-\Phi w)^T(\boldsymbol t-\Phi w) $  
	
	let $\frac {\partial} {\partial \beta} \ln \text{Likelihood}=0$ 
	
	$\displaystyle \Rightarrow \beta^{-1}=\frac 1 N (\boldsymbol t-\Phi w)^T(\boldsymbol t-\Phi w)$ 
	
		\(\text{Note: solve $w=w_{ML}$ first}\) 
	\end{itemize}

\item Maximum Posterior:

	\begin{itemize}
	\item $\text{Posterior}=p(w|\boldsymbol t), \text{Prior}=p(w), \text{Likelihood} = p(\boldsymbol t|w), \text{Normalization}=p(t) \\ \displaystyle \Rightarrow \text{Posterior}=\frac {\text{Likelihood*Prior}} {\text{Normalization}}$ 
	
	$\Rightarrow \text{Posterior} \propto \text{Likelihood*Prior} $  
	
	\item $\text{assume Prior } \displaystyle p(w) =  \mathcal{N}(w|m_0,S_0), \\ \small \text{so that Prior \& Likelihood are conjugate} \Rightarrow \text{Gaussian Posterior}$ 
	
	\item $\text{Likelihood } \displaystyle p(\boldsymbol t|w) = \prod_{n=1}^N\mathcal N(t_n|\phi(x_n)w,\beta^{-1}) =\mathcal N(\boldsymbol t|\Phi w,\beta^{-1}I )$ 
	
	\item $\Rightarrow \text{Posterior } \displaystyle p(w|\boldsymbol t) = \mathcal N(w|m_N,S_N),\\ \text{where } m_N=S_N(S_0^{-1}m_0+\beta\Phi^T \boldsymbol t),\space S_N^{-1}=S_0^{-1} + \beta \Phi^T\Phi$ 
	
	$\Rightarrow w_{MAP} = \text{mean of the Gaussian} = m_N$ 
	
		$\text{Note: can also get } w_{MAP} \text{ from taking gradient}$ 
	\end{itemize}

\item Simple Prior:

$\text{Prior } p(w)=\mathcal N(w|0,\alpha^{-1}I)$ 

$\Rightarrow \text{Posterior } \displaystyle p(w|\boldsymbol t) = \mathcal N(w|m_N,S_N),\\ \text{where } \displaystyle m_N=\beta(\alpha I + \beta \Phi^T\Phi)\Phi^T \boldsymbol t,\space S_N^{-1}=\alpha I + \beta \Phi^T\Phi$ 

$ w_{MAP} \rightarrow w_{ML}, \text{ when }\alpha \rightarrow 0 \text{ (most useless Prior)}$ 

\item Maximize $\text{Posterior} \Leftrightarrow$ Minimize $\text{cost function with regularization}$: 

	$ \text{Simple Prior} \Rightarrow \displaystyle \ln p(w|\boldsymbol t) = -\frac \beta 2 (\boldsymbol t-\Phi w)^T (\boldsymbol t-\Phi w)-\frac \alpha 2 w^Tw+const$ 

\item If $\alpha \to 0 \text { (Prior is most useless)}$, maximize $\text{Posterior}$ is equivalent to maximizing likelihood 
\item Maximize $\text{Posterior}$ equal to minimize sum-of-squares error function with the addition of a quadratic regularization term with $\lambda = \alpha/\beta$ 
\item Regularization term comes from the $\text{Prior}$ 

\item Predictive Distribution:

\item Assume: $\text{Prior}: p(x|\alpha) = \mathcal N(x|0,\alpha^{-1}I),\space \small\text{ (}m_0=0,S_0=\alpha^{-1}I \text{)}$ 

\item $\displaystyle p(t|x,X,\boldsymbol t) = \int p(t|w,x)p(w|X,\boldsymbol t)dw$ 

\item $\Rightarrow p(t|x,X,\boldsymbol t) = \mathcal N(t|m_N^T\phi(x),\sigma^2_N(x))$ 

	$\text{where } \sigma_N^2(x)=\frac 1 \beta + \phi(x)^TS_N\phi(x);\space m_N, S_N \text{ from Posterior} \small (m_N=w_{MAP})$ 

\item Sequential data:

	\begin{itemize}
	\item $\text{Posterior}$ from previous data $\Leftrightarrow$ the $\text{Prior}$ for the arriving data
	\item Sequential data and data in one go is equivalent when finfding the Porsterior
	\end{itemize}

\item Gradient descent

	\begin{itemize}
	\item Hypothesis function: 
		\begin{itemize}
		\item $h_\theta(x)=x\theta$, $\small\theta = [\theta_0, \theta_1, ..., \theta_n]^T$, $\small x=[x_0, x_1,..., x_n], x_0=1$
		\end{itemize}
	
	\item $x$ is one instance
	\item Cost function: $\displaystyle J(\theta)=\frac{1}{2m}\sum_{i=1}^m(h_\theta(x^i)-y^i)^2+\frac{\lambda}{2m}\sum^n_{j=1}\theta_j^2$ 
	\item Update rule: $\forall \theta_j \in \theta, \theta_j := \theta_j-\alpha\large\frac{\partial J(\theta)}{\partial\theta_j}$, $\displaystyle \small\frac{\partial J(\theta)}{\partial\theta_j}=\frac{1}{m}\sum^m_{i=1}[(h_\theta(x^i)-y^i)x_j^i] + \frac\lambda m\theta_j$ 
	- $\displaystyle \frac {d}{d\theta}J(\theta) = \frac 1m ((X\theta-y)^TX)^T + \frac \lambda m [0,\theta_1,...,\theta_m]^T$ ($\theta_0$ shouldn't be penalized)
	\item simultaneously for all $\theta_j \in \theta$ 
	\item Normal equation (mathematical solution)
		\begin{itemize}
		\item $\theta = (X^TX)^{-1}X^Ty$  
		\end{itemize}
	\end{itemize}
\end{itemize}

\section{Logistic Regression (Classification)}

- Decision Theory:

- classes $C_1,...,C_K$, decision regions $\mathcal {R}_1,...,\mathcal{R}_K$ 
- Minimze $\displaystyle p(mistake) = \sum_{k=1}^K (\int_{\mathcal {R}_k} \sum_{i \neq k} p(x,C_i)dx)$ $\small \text{(can have weight on each misclassfication though)}$  
- Maximize $\displaystyle p(correct) = \sum_{k=1}^K \int _{\mathcal {R}_k}p(x,C_k)dx$ 

- Models for Decision Problems:

- Find a discriminant function
- Discriminative Models: $\text{less powerful, yet less parameter => easier to learn}$ 
- Infer **posterior** $p(C_k|x)$, $\small \text{$C_k$: $x \in C_k$, $x$ is examples in trainning set}$ 
- Use decision theory to assign a new $x$ 
- Generative Models: $\text{more powerful, but computationally expensive}$ 
- Infer conditional probabilities $p(x|C_k)$ 
- Infer prior $p(C_k)$ 
- Find **either** **posterior** $p(C_k|x)$, **or** **joint distribution** $p(x, C_k)$ (using Bayes' theorem)
- Use decision theory to assign a new $x$ 
- **=> Able to create synthetic data using $p(x)$** 

- Naive Bayes on Discrete Features:

- Assumption:

- Discrete Features: data point $x\in\{0,1\}^D$ 

- Naive Bayes: all features conditioned on class $C_k$ are independent with each other

$\displaystyle \Rightarrow p(x|C_k)=\prod_{i=1}^D \mu_{ki}^{x_i} \space (1-\mu_{ki})^{1-x_i}$ 

1. Linear Discriminant (Least Squares Approach)

- Prediction:

- $y(x)=xw+w_0 \text{, with bias $=w_0$, where } w = [w_1,...,w_n]^T, x=[x_1,...,x_n]$ 
- $y(x)>0$: positive confidence to assign $x$ to current class 
- $-w_0$ called threshold sometimes

- Decision Boundary $y(x)=w^Tx+w_0=0$: 

- $w$ is orthogonal to vectors on the boundary:

	$\text{assume } x_1,x_2 \text{ on the boundary} \\ \Rightarrow 0=y(x_1) - y_(x_2)=(x_1-x_2)w$ 

- Distance from origin to boundary is $-\frac {w_0} {\|w\|}$: 

	$\text{assume distance is } k \\ \Rightarrow k \frac {w}{\|w\|} \text{on boundary, thus } k \frac {w}{\|w\|} w + w_0=0 \\ \Rightarrow k=-\frac {w_0}{\|w\|}$ 

- $y(x)$ is a signed measure of distance from point $x$ to boundary:

	$\text{assume distance is }r \\ \Rightarrow y(x)=\overbrace{(x_\perp+r \frac {w}{\|w\|})}^x w + w_0 = \overbrace{x_\perp w+w_0}^0 + r\|w\| = r\|w\| \\ \Rightarrow r = \frac{y(x)} {\|w\|}$ 

- Multi-class $\small\text{(k-classes)}$:

- prediction: $x$ is of class ${\mathcal C}_k$ if $\forall j \neq k, y_k(x) > y_j(x)$ 

$\Rightarrow y(x)=xW \text{, where } W=[w_1,...,w_k],\forall x_i \in X,x_{i0}=1 {\small\text{ (bias)}}, y(x) \text{ is 1-of-k coding} $ 

- sum-of-squares error: $E_D(W)=\frac 1 2 \text{tr}\{(XW-T)(XW-T)^T\}$ 

$\Rightarrow \text{optimal } W = (X^TX)^{-1}X^TT$ 

	$\small \text{note that } tr\{ AB \} = A^TB^T$ 

2. Fisher’s Linear Discriminant

- Basic idea:

- Take linear classification $y = w^Tx$ as dimensionality reduction (projection onto 1-D)
- => find a projection (denoted by vector $w$) which maximally preserves the class separation
- => if $y>-w_0$ then class $C_1$, otherwise $C_2$ 

- Goal:

- Distance within one class is small
- Distance between classes is large

- Mean \& Variance of Projected Data:

- Mean: $\displaystyle\widetilde m_k = w^Tm_k, \text{ where } m_k = \frac 1 {N_k}\sum_{x\in C_k} x$ 
- Variance: $\displaystyle \widetilde s_k=\sum_{x\in C_k}(w^Tx-w^Tm_k)^2 =   w^T[\sum_{x\in C_k}(x-m_k)(x-m_k)^T]w$ 

- 2-Classes to 1-D line:

- Maximize Fisher criterion: $\displaystyle J(w) = \frac {|\widetilde m_1 - \widetilde m_2|^2} {{\widetilde s_1}^2+{\widetilde s_2}^2}$ 

- Between-class covariance:  $\displaystyle S_B = (m_1-m_2)(m_1-m_2)^T$ 

- Within-class covariance: $\displaystyle S_k=\sum_{n\in C_k}(x_n-m_k)(x_n-m_k)^T$ 

$\displaystyle\Rightarrow J(w) = \frac {w^TS_B w} {w^T S_W w} $ 

- Lagrangian: $L(w,\lambda) = w^TS_B w + \lambda(1- w^T S_W w)$ 

	$\small \text{fix $w^T S_W w$ to 1 to avoid infinite solution (any multiple of a solution is a solution)}$

$\Rightarrow \displaystyle \frac \partial {\partial w}L = 2S_Bw-2\lambda S_Ww=0 \\ \Rightarrow S_Bw=\lambda S_Ww \\ \Rightarrow ({S_W}^{-1}S_B)w=\lambda w \\ \text{To maximize $J(w)$, $w$ is the largest eigenvector of ${S_W}^{-1}S_B$ if $S_W$ invertible} $ 

- K-classes to a d-D subspace: $\small N_k \text{ is num in class k, $N$ is the total example num}$ 

- Between-class covariance:  $\displaystyle S_B = \sum_{k=1}^K N_k(m_k-m)(m_k-m)^T, \text{where } m=\frac 1 N \sum_{n=1}^N x_n  \\ \small \text{reduce to } (m_1-m_2)(m_1-m_2)^T  \text{ when K=2 (constant ignored)}$ 

- Within-class covariance: $\displaystyle S_W = \sum_{k=1}^KS_k, \small \text{where } S_k=\sum_{n\in C_k}(x_n-m_k)(x_n-m_k)^T, m_k = \frac 1 {N_k} \sum_{n \in C_k}x_n$ 

- Maximize Fisher criterion: $\displaystyle J(w) = \frac {tr\{W^TS_B W\}} {tr\{W^T S_W W\}}$ 

- Lagrangian: 

$\text{Solve for each } w_i\in W \Rightarrow ({S_W}^{-1}S_B)w_i=\lambda_i w_i \\ \Rightarrow W \text{ conosists of the largest d eigenvectors} \\ \small {S_W}^{-1}S_B \text{ is not guaranteed to be symmetric } \Rightarrow W \text{ might not be orthogonal} \\ \small\text{Need to minimize the whole covariance matrix ($J(w)$ as a matrix) $\Rightarrow$ not choosing same eigenvectors twice} $  

- Maximum Possible Projection Directions = $K-1$: 

	$r({S_W}^{-1}S_B) \le \min(r({S_W}^{-1}),r(S_B)) \le r(S_B) \\ r(S_B) \le \displaystyle \sum_K r((m_k-m)(m_k-m)^T) = K \small \text{, as } r(m_k-m)=1 \\ \displaystyle \sum_Km_k = m \Rightarrow r(m_1-m,...,m_K-m)=K-1 \\ \Rightarrow r(S_B)\le K-1 \\ \Rightarrow r({S_W}^{-1}S_B) \le K-1$ 

3. Perceptron Algorithm

- Generalised linear model $y = f(w^T\phi(x))$, where $\phi(x)$ is basis function; $ \phi_0(x) = 1$ 

- Nonlinear activation funtion: $f(a) = \begin{cases} 1,& a \ge0 \\-1,& a<0  \end{cases}$ 

- Target coding: $t = \begin{cases} 1,&\text{if } C_1 \\ -1, &\text{if } C_2 \end{cases}$ 

- Cost function: 

- All correctly classified patterns: $w^T\phi(x_n)t_n>0$ 

- Add the errors for all misclassified patterns (denoted as set $\mathcal{M}$):

$\Rightarrow \displaystyle E_P(w)=-\sum_{n\in \mathcal{M}} w^T \phi(x_n)t_n$ 

- Algorithm: (Aim: minimize total num of misclassified patterns)

- loop 

	choose a traning pair $(x_n, t_n)$ 

	update the weight vector $w$: $w  = w - \eta \nabla E_p(w) = w+\phi_nt_n$ 

		$\text{ where $\eta$=1 because $y=f(\cdot)$ does not depend on $\|w\|$}$ 

- Perceptron Convergence Theorem:

- If the training set is linearly separable, the perceptron algorithm is guaranteed
to find a solution in a finite number of steps

$\small \text{(Also is the algorithm to find whether the set is linearly separable => Halting Problem)}$ 

4. Maximum Likelihood

- Assumption: 
- $p(x|C_k) \sim\mathcal{N}(\mu_k,\Sigma), \text{and all } p(x|C_k) \text{ share the same } \Sigma$ 
- \(p(C_1) = \pi,\space p(C_2)=1-\pi \text{, $\pi$ unknown}\) 
- Likelihood of whole data set $\boldsymbol{X,t}$, $N$ is the num of data
- $\displaystyle p(\boldsymbol{X,t}|\pi, \mu_1, \mu_2, \Sigma) = \prod_{n=1}^N[\pi\mathcal{N}(x_n|\mu_1,\Sigma)]^{t_n} \space [(1-\pi)\mathcal{N}(x_n|\mu_2, \Sigma)^{1-t_n}]$ $\space\small\rightarrow\text{when info of label $t$ lost: mixture of Gaussian}$  
- $\displaystyle \ln (\text{Likelihood}) = \sum_{n=1}^N[t_n(\ln\pi+\ln\mathcal{N}(x_n|\mu_1,\Sigma)) + (1-t_n)(\ln(1-\pi)+\ln\mathcal{N}(x_n|\mu_2, \Sigma))]$ 
- Parameters when maximum reached:
- $\displaystyle \pi = \frac {N_1}{N_1+N_2}$, $N_1$ is the num of class $C_1$ 
- $\displaystyle \mu_1 = \frac 1 {N_1} \sum_{n=1}^Nt_n x_n, \space \mu_2=\frac 1 {N_2}\sum_{n=1}^N(1-t_n)x_n,$ (mean of each class)
- $\displaystyle \Sigma = \frac {N_1}{N}S_1 + \frac {N_2}NS_2, \text {where } S_k = \frac 1 {N_k}\sum_{n \in C_k}(x_n-\mu_k)(x_n-\mu_k)^T $ 

5. Logistic Regression 

- Sigmoid function: $\displaystyle \sigma(a) = \frac 1 {1+e^{-a}}$ 

- $p(x|C_k) \sim \mathcal{N} \implies p(C_k|x)=\sigma(w^Tx+w_0) \small \text{ (2-classes)}$  (Generative model)

- Assumption:

$p(x|C_k) = \mathcal{N}(\mu_k, \Sigma)$ (can also be a number of other distributions)

$\forall k, p(x|C_k) \text{ shares the same } \Sigma$ 

- \begin{align*} \displaystyle p(C_1|x)=\frac{p(x|C_1)p(C_1)}{p(x|C_1)p(C_1)+p(x|C_2)p(C_2)} = \sigma (a), \\  \text{where } a &=\ln \frac {p(x,C_1)}{p(x,C_2)} \\ &= \ln \frac {p(x|C_1)p(C_1)}{p(x|C_2)p(C_2)} \\ &= \dots \text{(assumption applied)} \\ &= \ln \frac {\text{exp}(\mu_1^T\Sigma^{-1}x - \frac 12 \mu_1^T\Sigma^{-1}\mu_1)}{\text{exp}(\mu_2^T\Sigma^{-1}x - \frac 12 \mu_2^T\Sigma^{-1}\mu_2)} + \ln \frac {p(C_1)}{p(C_2)}  \\ \implies & a = w^Tx+w_0  \text{ where, } \\ &\space w = \Sigma^{-1}(\mu_1-\mu_2)  \\ & \space w_0 = -\frac 12\mu_1^T\Sigma^{-1}\mu_1 + \frac 12 \mu_2^T\Sigma^{-1}\mu_2 + \ln \frac {p(C_1)}{p(C_2)} \end{align*} 

- $\displaystyle \implies p(C_1|x) = \sigma(w^Tx+w_0)$ 

$\Rightarrow$ Find parameters in Gaussian model using Maximal Likelihood Sulotion

	as: $p(C_1|x)\propto p(x|C_1)p(C_1)=p(x,C_1)$ 

- Generalize to $\text{K-classes}$:

$\displaystyle a_k(x) = \ln [p(x|C_k)p(C_k)], \space p(C_k|x) = \frac{\text{exp}(a_k)}{\sum_i \text{exp}(a_i)}$ 

$\Rightarrow \displaystyle a_k(x) = w_k^Tx+w_{k0}, \text{where } w_k=\Sigma^{-1}\mu_k; \space w_{k0} = -\frac 1 2 \mu_k^T\Sigma^{-1}\mu_k + p(C_k)$ 

- $\text{Assume directly } p(C_k|x)=\sigma(w^Tx+w_0) \small \text{ (2-classes)}$ (Discriminative model)

- Assume directly: $p(C_1|w,x) = \sigma(w^Tx), \space x_0 = 1$ 

	$\Rightarrow$ less parameters to fit (compared to Gaussian)

- Likelihood function: 

$\displaystyle p(\boldsymbol{t}|w,X) = \prod_{n=1}^Np_n^{t_n}(1-p_n)^{1-t_n}, \text{ where, } p_n=p(C_1|x_n), \space t_n \text{ is the class of } x_n$  

Define error function :

	$\displaystyle E(w) = -\ln(Likelihood) = - \sum_{n=1}^N [t_n\ln p_n + (1-t_n)\ln(1-p_n)]$ 

$\displaystyle \Rightarrow \nabla E(w)=\sum_{n=1}^N(p_n-t_n)x_n$ 

- Find Posterior $p(w|\boldsymbol{t})$: 

$\text{Likelihood}$ is product of sigmoid

Conjugate $\text{Prior}$ for "sigmoid distribution" is unknown

$\Rightarrow$ Assume $\text{Prior } p(w) = \mathcal{N}(w|m_0, S_0)$ 

$\Rightarrow \displaystyle \ln p(w|\boldsymbol{t}) \propto -\frac 1 2 (w-m_0)^TS_0^{-1}(w-m_0) + \sum_{n=1}^N[t_n\ln p_n + (1-t_n) \ln(1-p_n) ]$ 

	find $w_{MAP}$, calculate $\displaystyle S_N = -\nabla \nabla \ln p(w|\boldsymbol{t}) = S_o^{-1} + \sum_{n=1}^N p_n(1-p_n)\phi_n\phi_n^T$ 

$\Rightarrow p(w|\boldsymbol{t}) \simeq \mathcal{N}(w|w_{MAP}, S_N), \small \text{via Laplace Approximation}$ 

- Laplace Approximation: 

- Fit a guassian to $p(z)$ at its **mode** ( mode of $p(z)$: point where $p'(z)=0$ )

- Assume $p(z) = \frac 1 Z f(z), \text{with normalization } Z = \int f(z)dz$ 

Taylor expansion of $\ln f(z)$ at $z_0$: $\ln f(z) \simeq \ln f(z_0)-\frac 1 2A(z-z_0)^2,$ 

	$\text{where } f'(z_0)=0, A=- \frac {d^2}{dz^2}\ln f(z) |_{z=z_0}$ 

Take its exponentiating: $f(z) \simeq f(z_0)\text{exp{$-\frac A 2 (z-z_0)^2$}}$ 

$\displaystyle \Rightarrow \text{Laplace Approximation} = (\frac A {2\pi})^{1/2} \text{exp{$-\frac A 2 (z-z_0)^2$}} \text{, where }A=\frac 1 {\sigma^2}$ 

- Requirement:

	$f(z)$ differentiable to find a critical point

	$f''(z_0) < 0$ to have a maximum \& so that $\nabla \nabla\ln f(z_0)=A>0$ as $A=\frac1 {\sigma^2}$ 

- In Vector Space: approximate $p(z)$ for $z\in \mathcal{R}^M$ 

Assume $p(z)=\frac 1 Z f(z)$

Taylor expansion: $\ln f(z) \simeq \ln f(z_0) - \frac 1 2 (z-z_0)^TA(z-z_0),$ 

	$\text{Hessian } A = - \nabla \nabla \ln f(z)|_{z=z_0} $ 

\begin{align}  \Rightarrow \text{Laplace approximation}&= \frac {|A|^{1/2}}{(2\pi^{M/2})}\text{exp{$-\frac 1 2 (z-z_0)^TA(z-z_0)$}} \\ &= \mathcal{N}(z|z_0, A^{-1}) \end{align}

- Gradient descent:

- Hypothesis function: $h_\theta(x)=\sigma(x\theta)=\large\frac{1}{1+e^{-x\theta}}$ 

- Cost function: 

$\displaystyle J(\theta)=\frac{1}{m}\sum^m_{i=1}[-y^i \ln (h_\theta(x^i))-(1-y^i) \ln (1-h_\theta(x^i))]+\frac{\lambda}{2m}\sum^n_{j=1}\theta_j^2$ 

- Update rule: $\forall \theta_j \in \theta, \theta_j := \theta_j-\alpha\large\frac{\partial J(\theta)}{\partial\theta_j}$, $\displaystyle \small\frac{\partial J(\theta)}{\partial\theta_j}=\frac{1}{m}\sum^m_{i=1}[(h_\theta(x^i)-y^i)x_j^i] + \frac \lambda m\theta_j$ 

\section{Latent Variable Analysis}

\subsection{Principal Component Analysis (PCA)}

1. Motivation:

- Data compression (reduce highly related features)
- Data visualization

2. Assumption:

- Gaussian distributions for both the latent and observed variables

3. Two Equivalent Definition of PCA:

- Linear projection of data onto lower dimensional linear space (principal subspace) such that: 

$\Rightarrow \text{variance of projected data is maximized} \\ \Rightarrow \text{distortion error from projection is minimized} $ 

4. Maximum Variance Formulation

- Goal: 

- project data from D dimension to M while maximizing the variance of projected data

- Eigenvalues $\lambda$ of covariance matrix $S$ express the variance of data set $X$ in direction of corresponding eigenvectors

- Projection Vectors: 

- $U = (u_1,...,u_M), \text{ where } \forall i\in\{1,..,M\},u_i \in \mathbb R^D \text{ s.t. } u_i^Tu_i=1 \space \small \text{(only consider direction)} $ 

- Projected Data:

- $\displaystyle \text{Mean} = {\bar x}^TU \text{, where } \bar x = \frac 1 N\sum_{i=1}^Nx^i$ 
- $\displaystyle \text{Variance} =  tr\{U^T S U\} \text{, where } S = \sum_{i=1}^N(x^i-\bar x)(x^i-\bar x)^T \space\small\text{ (outer product)}$ 

- Lagrangian to maximize $\text{Variance}$: 

- $\displaystyle L(U,\lambda)=tr\{U^TSU\}+tr\{(I-U^TU)\lambda\}$ 

	$\small\text{constraint }u_i^Tu_i=1 \text{ to prevent $u_i \rightarrow +\infty$ } $ 

\begin{gather}\text{For each $u_i\in U$, } \displaystyle \frac \partial {\partial u_i}L = 2Su_i-2\lambda_iu_i=0 \\ \Rightarrow Su_i=\lambda_iu_i \\ \Rightarrow \text{$U$ consists of eigenvectors corresponding to the first M large eigenvalue of $S$}\end{gather}

	\(\text{($S$ symmetric $\Rightarrow U$ orthogonal)}\)

5. Minimum Error Formulation:

- Introduce Orthogonal Basis Vector for D dimension:

- $U=(u_1,...,u_D)$ 

- Data representation:

- Original: $\displaystyle x^n = \sum_{i=1}^D\alpha^n_iu_i$ 
- Projected: \(\displaystyle \widetilde {x^n} = \sum_{i=1}^Mz_i^nu_i+\sum_{i=M+1}^Db_iu_i \\ \small\text{$(z_1^n,...,z_M^n)$ is different for different $x^n$, $(b_{M+1},...,b_D)$ is the same for all  $x^n$}\)

- Cost function: $\displaystyle J=\frac1N \sum^N_{n=1}\|x^n-\widetilde{x^n}\|^2, \text{where } \widetilde{x^n}=\sum_{i=1}^Mz^n_iu_i + \sum_{i=M+1}^Db_iu_i$ 

- $\text{Let } \begin{cases} \displaystyle \frac \partial {\partial z^n_j} J=0 \\ \displaystyle \frac \partial {\partial b_j}J=0 \end{cases} \Rightarrow \begin{cases} \displaystyle \frac 1 N 2(x^n-\widetilde{x^n})^T (-u_j) = \frac 2 N (z_j-(x^n)^Tu_j)=0 \\ \displaystyle \frac 1 N \sum_{n=1}^N2 (x^n-\widetilde {x^n})^T(-u_j) = \frac 2 N \sum_{n=1}^N (b_j-(x^n)^Tu_j)=0 \end{cases}$ 

$\Rightarrow \begin{cases} z_j=(x^n)^Tu_j & j\in\{1,...,M\}\\ b_j=\overline{x}^Tu_j & j\in\{M+1,...,D\} \end{cases}$ 

Noticing $\displaystyle (x^n)^Tu_j=(\sum_{i=1}^D\alpha_i^nu_i^T)u_j = a_j \Rightarrow a_j = (x^n)^Tu_j$ 

$\displaystyle \Rightarrow x^n-\widetilde{x^n} = \sum_{i=M+1}^D[(x^n-\overline x)^Tu_i]u_i$ 

- \begin{align} \boldsymbol \Rightarrow \displaystyle J &= \frac 1 N \sum_{n=1}^N\space \left( \sum_{i=M+1}^D [(x^n-\overline x)^Tu_i]u_i\right)^T \left(\sum_{i=M+1}^D [(x^n-\overline x)^Tu_i]u_i\right) \\ &= \frac 1 N \sum_{n=1}^N \left( \sum_{i=M+1}^Du_i^T ((x^n-\overline x)^Tu_i)\right) \left( \sum_{i=M+1}^D ((x^n-\overline x)^Tu_i)u_i \right) \\ &= \frac 1 N \sum_{n=1}^N \sum_{i=M+1}^D u_i^T(x^n-\overline x)^Tu_iu_i^T(x^n-\overline x)u_i & u_i \text{ orthogonal to each other} \\ &= \sum_{i=M+1}^D u_i^T \left( \frac 1 N \sum_{n=1}^N(x^n-\overline x)^T(x^n-\overline x)\right) u_i & \|u_i\|=1 \\ \end{align} 

$ \displaystyle \boldsymbol \Rightarrow J =  \sum_{i=M+1}^D u_i^TSu_i \text{, where }S=\frac 1 N\sum_{n=1}^N(x^n-\overline x)^T(x^n-\overline x)$

- Lagrangian to Minimize $J$: 

- $\displaystyle L(u_{M+1},...,u_D,\lambda_{M+1},...,\lambda_D) = \sum_{i=M+1}^Du_i^TSu_i + \sum_{i=M_1}^D \lambda_i(1-u_i^Tu_i)$ 

	\(\text{constraint $\|u_i\|=1$ to prevent $u_i=0$}\)

\(\text{For each }u_i, \displaystyle \frac \partial {\partial u_i}L=2Su_i-2\lambda_iu_i=0 \\ \Rightarrow Su_i=\lambda_iu_i \\ \Rightarrow \text{To minmize $J$, take eigenvectors with the first $(D-M)$ small eigenvalue orthogonal to (out of) subspace} \\ \Leftrightarrow \text{define subspace with eigenvectors with the first $M$ large eigenvalue} \)

- \begin{align}\displaystyle \text{Intuition: }\widetilde{x_n}&=\sum_{i=1}^M((x^n)^Tu_i)u_i+\sum_{i=M+!}^D(\overline {x}^Tu_i)u_i \\ &= \overline x + \sum_{i=1}^M[(x^n-\overline x)^Tu_i]u_i \end{align}

1. Singular Value Decomposition - SVD:

- Intorduce matrix $A_{m\times n}$ 

- $(A^TA)_{n\times n}$ symmetric matrix ($\text{actually, Gram matrix} \rightarrow \text{semi-definite}$) 
- \begin{gather} \text{eigenvalue decomposition: } \\ A^TA=VDV^T\text{, $V$ is normalized ($v_i^Tv_i=1$) with column as eigenvector} \end{gather}

- $AV=(Av_1,...,Av_n)_{m\times n}$ 

- Let $r(A)=r$

\begin{align} \Rightarrow & r(A^TA)=r(A)=r \space \\&  r(AV)=\min\{r(A),r(V)\}=\min\{r,n\}=r\end{align}

- Reduce $AV$ to basis $(Av_1,...,Av_r)$  

- Let \(\displaystyle U=(u_1,...,u_r) = (\frac {Av_1} {\sqrt {\lambda_1}},...,\frac{Av_r}{\sqrt{\lambda_r}}) \space \text{, $\lambda_i$ is $i$-thh eigenvalue of $A^TA$}\) 

- Orthogonal: $\forall i\neq j, u_i^Tu_j=\frac 1 {\sqrt{\lambda_i\lambda_j}}v_i^TA^TAv_j=\frac {\lambda_j} {\sqrt{\lambda_i\lambda_j}}v_i^Tv_j=0$ 

- Unit: $\|u_i\|=\frac {\|Av_i\|} {\sqrt{\lambda_i}}=\frac {\sqrt {<Av_i,Av_i>}} {\sqrt{\lambda_i}}=1$ 

$\boldsymbol \Rightarrow U \text{ is standard orthogonal (orthonormal) basis}$ 

- $AV=U\Sigma,\text{ where } \Sigma = D^{\frac 1 2}$ 

- Expand $U$ to orthonormal in $\mathbb R^m: \space (u_i,...,u_m)$ 

- Epand corresponding part in $\Sigma$ with $0$ 

- \(A = U\Sigma V^T,\text{ with singular value in $\Sigma$ in decreasing order}\)

2. SVD with PCA:

- $X$ is data matrix in row ($\text{centered - zero mean}$)

- Eigenvectors of corvariance matrix $S=X^TX$ are in $V, \text{ where }X = U\Sigma V^T$ 

- When using $S=U\Sigma V^T \Rightarrow \space U=V \space \land \space S=V\Sigma V^T$ 

	$\text{reduced to eigenvalue decomposition}$ 

- $S=VDV^T \text{ with $V$ orthonormal}$: 

Eigenvalues $\lambda$ of covariance matrix $S$ express the variance of data set $X$ in direction of corresponding eigenvectors

- Projection:

- $\widetilde X = XV_M, \text{ where $V_M$ contains first M-large eigenvectors}$ 
- Projection direction is **not** unique 

3. Reconstruction (approximate): 

- Data is projected onto $k$ dimension using $\text{SVD}$ with $S = U\Sigma V^T$ 
- $x_{approx} = U_{reduce} \cdot z$,  $U_{reduce}$ is n*k matrix, $z$ is k*1 vector
- ![Reconsturction from data Compression](../../Machine%20Learning/Statistical%20Machine%20Learning/Reconsturction%20from%20data%20Compression.png) 

4. Choosing $k$ (num of principal components):

- choose the **smallest** k making $\displaystyle \frac JV \leq 0.01$ => 99% of variance is retained 

- $[U,S,V]$ = svd(Sigma) => $\displaystyle \frac JV=1-\frac {\sum^k_{i=1}S_{ii}}{\sum^n_{i=1}S_{ii}}$, $S$ is diagonal matrix

=> check $\frac JV$ before compress data 

5. Data Preprocessing:

- PCA $vs.$ Normalization:
- Normalization: Individually normalized but still correlated
- PCA: create decorrelated data $-\text{ whitening}$ 
- Whitening: $\text{ projection with normalization}$ 
- $S = VDV^T \text{, where $S$ is Gram matrix over $X^T$}$ 
- \(\forall n, y_n=D^{-\frac12}V^T(x^n-\overline x) \text{, where $\overline x$ is the mean of $X$}\) 

\begin{align} \Rightarrow & \text{{$y^n$} has zero mean} \\ & \displaystyle cov(\{y^n\}) = \frac 1 N \sum_{n=1}^Ny_ny_n^T= D^{\frac {-1} 2}V^TSVD^{\frac{-1}2}=I \end{align}

6. Tips for PCA:

- Do NOT use PCA to prevent overfitting, use regularization instead
- Try original data before implement PCA
- Train PCA only on trainning set

\subsection{Independent Component Analysis (ICA)}

1. Goal:
- Recover original signals from a mixed observed data
- Source signal $S\in \mathbb R^{N\times K}$; mixing matrix $A$; Observed data $X=SA$
- Maximizes statistical independence
- Find $A^{-1}$ to maximizes independence of columns of $S$
2. Assumption: 
- At most one signal is Gaussian distributed
- Ignorde amplitude and order of recovered signals
- Have at least as many observed mixtures as signals
- $A$ invertible
3. Independence $vs.$ Uncorrelatedness
- Independence $\Rightarrow$ Uncorrelatedness
- $p(x_1,x_2)=p(x_1)p(x_2) \Rightarrow \mathbb E(x_1x_2)-\mathbb E(x_1)\mathbb E(x_2) = 0$ 
4. Central Limit Theorem
5. FastICA algorithm

\subsection{t-SNE}

1. Problem \& Focus
2. Compared to PCA:
- No whitening function to use for new data
- PCA can only capture linear structure inside the data
- t-SNE preserves the <u>local distances</u> in the original data

\subsection{Anomaly Detection}

1. Problem to solve:

- Given dataset {$x^1,x^2,...,x^m$}, build density estimation model $p(x)$
- $p(x^{test}<\epsilon)$ => $x^{test}$ anomaly 

2. Hypothesis function: 

- $\displaystyle p(x)=\prod^n_{i=1} p(x_i), \space x \in R^n,\forall i \in [1,n], \space x_i \sim N(\mu_i,\sigma_i^2)$ 
- $\displaystyle \mu=\frac1m \sum^m_{i=1}x^i,\space \sigma^2=\frac1m \sum^m_{i=1}(x^i-\mu)^2$ 
- assume $x_1,...,x_n$ independent from each other

3. Multivariate Gaussian:

- $\displaystyle p(x;\mu,\Sigma)=\frac1{(2\pi)^{\frac n2} |\Sigma|^{\frac 12}} exp(-\frac12 (x-\mu)^T \Sigma^{-1} (x-\mu)),$  

$x\in R^n, \mu\in R^n,\Sigma\in R^{n\times n}$, where $\Sigma$ is covariance matrix

- $\displaystyle \mu=\frac1m \sum^m_{i=1}x^i,\space \Sigma=\frac 1m \sum^m_{i=1}(x^i-\mu)(x^i-\mu)^T$ 
- $x_1,...x_n$ can be correlated but **not** linearly dependent
- need $m > n$ $(m\ge10n\space suggested)$ or elas $\Sigma$ non-invertible

4. Algorithm:

- choose features
- compute $\mu$, $\sigma$
- compute $p(x)$ for new example, anomaly if $p(x) < \epsilon $ 

5. Evaluation (real-number):

- Labeled data into normal/anomalous set

(okay if some anomalies slip into normal set)

- training set: unlabeled data from normal set (60%)
- CV set: labeled data from normal (20%) & anomalous (50%) set
- test set: labeled data from normal (20%) & anomalous (50%) set

- Use evaluation metrics (skewed data)

6. When to use:

- Anomaly detection:
- Very small num of positive data (0-20 commonly); Large num of negative data
- Difficult to learn from positive data (not enough data, too many features...)
- Future anomalies may look nothing like given data
- Supervised Learning:
- Larger num of positive \& negative data
- Enough positive data for algorithm to learn
- Future positive example is likely to be similar to given data

7. Example:

- Anomaly detection:
- Fraud detection, Manufacturing, Monitoring machines in data center...
- Supervised learning:
- Email spam classification (enough data), Weather prediction (sunny/rainy/etc), Cancer classification...

8. Tips:

- Non-guassian feature: transformation / using other distribution
- Choosing features: compare anomaly data with normal data


\subsection{Recommender System}

1. Problem Formulation:

- $r_{i,j}=1$ if item $i$ is rated by user $j$ 

- $y_{i,j}$ = rating of item $i$ given by user $j$ 

- $\theta^j$ = parameter vector for user $j$ 

- $x^i$ = feature vector for movie $i$ 

=> for user $j$, movie $i$, ($r_{i,j}=0$), predict rating $x^i\theta^j$

2. Content Based Recommendations:

- Treat each user as a seperate linear regression problem with the feature vectors of its rated items as traning set

**Assume features for each items ($x^i$) are available and known**

=> given $X$ estimate $\Theta$ 

- Cost Function for $\theta_j$: 

$\displaystyle J(\theta^j)= \frac1 {2} \sum_{i:r_{i,j}=1}(x^i\theta^j-y_{i,j})^2+\frac \lambda {2} \sum_{k=1}^n(\theta^j_k)^2, \theta^j \in R^{n+1} (\theta_0 \text{ not regularized)}$

- Cost Function for $\Theta$:

$\displaystyle J(\Theta)= \frac1{2} \sum_{j=1}^{n_u} \sum_{i:r_{i,j}=1}(x^i\theta^j-y_{i,j})^2+\frac \lambda {2} \sum_{j=1}^{n_u} \sum_{k=1}^n(\theta^j_k)^2, \\ \theta^j \in R^{n+1} (\theta_0 \text{ not regularized)}, \space n_u \text{ is num of users}$ 

- Update Rule: \(\forall \theta^j_k \in \theta^j, \theta^j_k := \theta^j_k-\alpha\large\frac{\partial J(\Theta)}{\partial\theta^j_k}\), \(\displaystyle \frac{\partial J(\Theta)}{\partial\theta^j_k}=\sum_{i:r_{i,j}=1}(x^i\theta^j-y_{i,j})x_k^i+ \lambda \theta_k^j, \space \text{for $k \neq 0$ ($\theta^j \in R^{n+1}$)}\) 

3. Collaborative Filtering

- \textbf{Assume preference of each users ($\theta^j$) are available and known}

=> given $\Theta$ estimate $X$ 

- Cost Function for $x^i$: $\displaystyle J(x^i)=\frac 1 2 \sum_{j:r_{i,j}=1} (x^i\theta^j - y_{i,j})^2 + \frac \lambda 2 \sum ^n_{k=1} (x_k^i)^2$ 
- Cost Function for $X$: $\displaystyle J(X)=\frac 1 2 \sum_{i=1}^{n_m} \sum_{j:r_{i,j}=1} (x^i\theta^j - y_{i,j})^2 + \frac \lambda 2 \sum_{i=1}^{n_m} \sum ^n_{k=1} (x_k^i)^2 \\ x^j \in R^{n+1} (x_0 \text{ not regularized)}, \space n_m \text{ is num of items}$ 
- Update Rule: \(\forall x^i_k \in x^i, x^i_k := x^i_k-\alpha\large\frac{\partial J(X)}{\partial x^i_k}\), \(\displaystyle \frac{\partial J(X)}{\partial x^i_k}=\sum_{j:r_{i,j}=1}(\theta^jx^i-y_{i,j})\theta_k^j+ \lambda x_k^i, \space \text{for $k \neq 0$ $(x^i \in R^{n+1}$)}\)

- Basic Idea: 

- Randomly initialize $\Theta$ 

- loop:

	Estimate $X$ 

	Estimate $\Theta$ 

- Cost Function: 

$\displaystyle J(X,\Theta) = \frac 1 2 \sum_{(i,j):r_{i,j}=1}(x^i\theta^j - y_{i,j})^2 + \frac \lambda 2 \sum_{i=1}^{n_m}\sum_{k=1}^n(x_k^i)^2 + \frac \lambda 2 \sum_{j=1}^{n_u}\sum_{k=1}^n (\theta_k^j)^2, \space x \in R^n, \space \theta \in R^n$ 

(the sum term in $J(\Theta)$, $J(X)$, and $J(X,\Theta)$ is the same)

- Update Rule: 

- $\forall x^i_k \in x^i, x^i_k := x^i_k-\alpha\large\frac{\partial J(X,\Theta)}{\partial x^i_k}$, $\displaystyle \frac{\partial J(X,\Theta)}{\partial x^i_k} = \frac{\partial J(X)}{\partial x^i_k} =\sum_{j:r_{i,j}=1}(\theta^jx^i-y_{i,j})\theta_k^j+ \lambda x_k^i, \space x^i \in R^n$ 
- $\forall \theta^j_k \in \theta^j, \theta^j_k := \theta^j_k-\alpha\large\frac{\partial J(X,\Theta)}{\partial\theta^j_k}$, $\displaystyle \frac{\partial J(X,\Theta)}{\partial\theta^j_k} = \frac{\partial J(\Theta)}{\partial\theta^j_k} = \sum_{i:r_{i,j}=1}(\theta^jx^i-y_{i,j})x_k^i+ \lambda \theta_k^j, \space \theta^j \in R^n$ 

- \textbf{Algorithm}

- Initialize $X, \Theta$ to **small random values**

=> for symmetry breaking (similar to random initialization in neural network) 

=> so that algorithm learns features $x^1,...,x^{n_m}$ that are different from each other

- Minimize $J(X,\Theta)$ 

- Predict $y_{i,j} = x^i\theta^j$ ($Y = X\Theta$)

- Finding Related Item to Recommend

- $||x^i-x^j||$ is samll => item $i$ and $j$ is similar

- Mean Normalization:

- Problem: if user $j$ hasn't rated any movie, $\theta^j = [0,...,0]$  

=> predicted rating of user $j$ on all item $=0$ 

=> useless prediction

- Algorithm (row version):

	compute vector $\mu, \space \forall \mu_i \in \mu, \mu_i = \text{mean of $Y_i$, where $Y_i$ is the $i^{th}$ row in $Y$}$ 

	manipulate $Y$: $\forall y_{i,j} \in Y \land r_{i,j}=1, \space y_{i,j} -= \mu_i$  => the mean of each row in $Y$ is $0$ 

	predict rating for user $j$ on item $i = x^i\theta^j + \mu_i$ 

- For item $i$ with no rating

=> apply column version of mean normalization

(but user with no rating is generally more important)

\section{Large Scale Machine Learning}

\subsection{Gradient Descent with Large Dataset}

1. Stochastic Gradient Descent
- Problem in Big Data: 
- Updating $\theta$ becomes computationally expensive in batch gradient decent

- Cost Function: 
- Cost function on single data: $cost(\theta, (x^i,y^i)) = \frac 1 2 (h_\theta(x^i)-y^i)^2$ 
- Overall Cost Function: $\displaystyle J_{train}(\theta) = \frac 1 m \sum_{i=1}^m cost(\theta, (x^i,y^i))$

- Procedure:

- Randomly shuffle dataset

- Repeat

	for $i \in [1,m]$ 

		\begin{align}\begin{split} \displaystyle \theta_j &= \theta_j - \alpha \frac {\partial }{\partial \theta_j} cost(\theta,(x^i,y^i)) \\ \displaystyle &= \theta_j - \alpha \space (h_\theta(x^i)-y^i) \cdot  x^i_j \end{split} \\ \text{(for $j=0,...,n$)} \\ \small => \text{make progress with each single data} \end{align}

- Convergence:

- Wanting $\theta$ to converge => slowly decrease $\alpha$ over time (but more parameters)

\( ( \displaystyle \text{E.g } \alpha = \frac {\text{const\_1}} {\text{iteration num + const\_2}} ) \)

- Compute $cost(\theta,(x^i,y^i))$ before updating 

For every $k$ update iterations, plot average $cost(\theta,(x^i,y^i))$ over the last $k$ examples

- Checking curves:

Increasing $k$ result in smoother line and less noise, but the result is more delayed

Use smaller learning rate $\alpha$ will generally have slight benefit

Curve goes up => smaller $\alpha$ 

- $vs$ Batch Gradient Descent:

- use $1$ example un each update iteration => make progress earlier => faster
- Result may not be the optimal but in its neighbourhood

1. Mini-batch Gradient Descent
- Use $b$ examples in each update iteration
- $vs$  Batch Gradient Descent:
- start to make progress earlier => faster
- Result may not be the optimal but in its neighbourhood
- $vs$ Stochatistic Gradient Descent:t
- can partially parallelize computation over $b$ examples => faster under a good vectorized implementation \& appropriate $b$ 
- introduce extra parameter $b$ 

\subsection{Online Learning}

1. Situation:
- Has too many data (can be considered as infinite)
- When data comes in as a continuous stream
- Can adapt to changing user preference
2. Procedure:
- Use one example only once (Similar to stochastic gradient decent in this sense

\subsection{Map-reduce}

1. In Batch Gradient Descent:

- Update rule $\displaystyle \theta_j = \theta_j - \alpha \frac 1 m \sum^m_{i=1} (h_\theta(x^i)-y^i)x_j^i$ 
- Parallelize the computation of $\displaystyle \sum^m_{i=1} (h_\theta(x^i)-y^i)x_j^i$ by dividing the data set into multiple sections

2. Ability to reduce:

- Contain operation over the whole data set

(Neural Network can be map-reduced)

\section{Building Machine Learning System}

- Under the example of Photo OCR (Optical Character Recognition)

\subsection{Pipeline}

1. Break ML system into modules

2. Example:

- Image -> Text detection -> Character segmentation -> Character recognition

- Text Detection: 

- Sliding window detection: 

	set different sizes of the window (mostly rectangle), for each size:

		take a image patch

		resize the patch into desired size

		run ML algorithm on the small patch

		slide the window by $\text{step\_size}$  (eventually through the image)

- Expansion: expand the related region to create a bigger region

- Chraracter Segmentation:

- 1-D sliding window

- Character Recognition

\subsection{Getting More Data}

1. **Artificial Data Synthesis** 
- Creating New Data: Use available resource and combine them
- Example (in Character Recognition): Paste different fonts in the randomly chosen backgrounds
- Amplify Data Set: Intorduce distortions to the original data set
- Need to identify the appropriate distortion
- Usually adding purely/random/meanless noise

- Prerequisite:
- Having a low bias/high variance hypothesis is 

1. Collect/Label Data Manually
- Usually a surprise to find how little time it needs to get 10,000 data
- Caculate the time it needs before decide to/not to collect the data

\subsection{Ceilling Analysis}

1. Aim:

- Decide which modules might be the best use of time to improve

2. Procedure:

- Draw a table with 2 column (Component - Accuracy)

- Component: the modules simulated to be perfect (100% accuracy)
- Accuracy: the accuracy of the entire system on the test set (define by chosen evaluation matrix)

- |   **Perfect Component**   |             **Accuracy**              |
| :-----------------------: | :-----------------------------------: |
|           none            |                  $f$                  |
|        module $1$         |            $f+\epsilon_1$             |
|       module $1,2$        |       $f+\epsilon_1+\epsilon_2$       |
| $\cdot \\ \cdot \\ \cdot$ |       $\cdot \\ \cdot \\ \cdot$       |
|    module $1,2,...,n$     | $f+\epsilon_1+...+\epsilon_n = 100\%$ |

- => Improving module $x$ will gain at most $\epsilon_x$ improvement in the overall performance

- Choose the module with most significant $\epsilon$ to improve









