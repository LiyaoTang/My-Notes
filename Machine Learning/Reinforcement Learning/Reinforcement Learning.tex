\chapter{Reinforcement Learning}

\section{Introduction}

\subsection{Learning Procedure}
\subsubsection{Overview}
\includegraphics[width=0.5\linewidth, center]{"./Reinforcement Learning/plot/intro - overall structure".jpg}
\begin{itemize}
\item Formulation - Markov Decision Process
	\begin{itemize}
	\item Notation
		\begin{itemize}
		\item state $s \in \mathcal S$ 
		\item action $a \in \mathcal A$ 
		\item reward $r \sim \mathcal R$, the distribution of reward, conditioned on action and state
		\item $\mathbb P$ the state transition probability \\
		$\Rightarrow p(s',r|s,a)$ the joint probability of reaching state $s'$ with reward $r$ \\
		(given taking action $a$ in current state $s$) \\
		\item return $G_t$, the measurement of future reward sequence $R_{t+1},...,R_T$, \\
		where $R_t$ the reward gained at time $t$
			\begin{itemize}
			\item $G_t = \sum_{k=0}^{T-t} R_{t+1+k}$, where $T$ the final time step
			\item discounted return $G_t = \sum_{k=0}^\infty \gamma^k R_{t+1+k}$, where $\gamma\in[0,1]$ the discount rate
			\end{itemize}
		\item policy $\pi$, a decision making policy \\
		$\Rightarrow \pi (a|s)$ gives the action $a$ to take in state $s$
		\end{itemize}
	\end{itemize}
\end{itemize}

\subsubsection{Q-Learning}
\begin{itemize}
\item Overview
	\begin{itemize}
	\item Goal
		\begin{itemize}
		\item learns the utility of performing actions in states
		\end{itemize}
	\item Notation
		\begin{itemize}
		\item Q table, a matrix for $\mathcal S\times  \mathcal A$, with score $Q(s,a)$ for taking action $a$ in state $s$ \\
		$\Rightarrow Q_\pi (s,a) = { \mathbb E }_\pi [G_t|S_t=s, A_t=a]$ the expected return \\ 
		(by taking action $a\in A(s)$ at state $s$ and following policy $\pi$ thereafter)
		\item $v_\pi(s) = {\mathbb E}_\pi [G_t|S_t=s]$, the expected return \\
		(starting from state $s$ and following policy $\pi$ thereafter)
		\end{itemize}
	\item Bellman Equation - using Markov Decision Process
		\begin{itemize}
		\Item \begin{align*} \displaystyle v_\pi(s) &= {\mathbb E}_\pi [G_t|S_t=s] \\
		&= {\mathbb E}_\pi \left[R_{t+1} + \gamma G_{t+1} \bigg|S_t=s \right] \\
		&= \sum_{s',r} p(s',r|s) \left[ r + \gamma \mathbb E\left[ G_{t+1} \big| S_{t+1} = s' \right] \right], \text{ where } r\in \mathcal R \text{ the reward received at } R_{t+1} \\
		&= \sum_{a\in \mathcal A(s)} \pi(a|s) \sum_{s'}\sum_r p(s',r|s,a) \left[ r + \gamma \mathbb E\left[ G_{t+1} \big| S_{t+1} = s' \right] \right] \\
		&= \sum_a \pi(a|s) \sum_{s',r} p(s', r|s,a) [r + \gamma v_\pi(s')] \end{align*}
		\Item \begin{align*} \displaystyle Q_\pi(s,a) &= {\mathbb E}_\pi [G_t | S_t=s,A_t=a] \\
		&= \displaystyle {\mathbb E}_\pi \left[R_{t+1} + \gamma G_{t+1} \bigg|S_t=s, A_t=a \right]  \\
		&= \displaystyle \sum_{s',r} p (s', r|s, a) \left[ r + \gamma {\mathbb E}_\pi \left[ G_{t+1} \big | S_{t+1} = s' \right] \right] \\
		&= \displaystyle \sum_{s',r} p(s',r|s,a)[r + \gamma v_\pi(s')] \end{align*}
		\item $\Rightarrow v_\pi(s) = \sum_a \pi(a|s) Q_\pi(s,a)$
		\item given an optimal deterministic policy $\pi^\star$, the optimal Q-value $Q^\star$ satisfy \\
		$\begin{alignedat}{2} &Q^\star(s,a) &&= \max_{\pi}\mathbb E\left[ G_t | S_t=s, A_t=a \right] \\
		& &&= \mathbb E_{s',r \sim p(s',t|s,a)}\left[r + \gamma \max_{a'}Q^\star(s', a')| s,a\right]
		\end{alignedat}$
		\end{itemize}
	\end{itemize}
\item Deep Q-Learning
	\begin{itemize}
	\item Deep Q-Function
		\begin{itemize}
		\item $q_\pi(s,a)$ becomes a deep net \\
		(instead of a lookup func on pre-computed Q-table) \\
		\item to reduce input size, network as mapping $s\rightarrow q_\pi(s, a)$ for all $a\in \mathcal A(s)$
		\end{itemize}
	\item Inference
		\begin{itemize}
		\item input state $s$, predict $Q(s, a) \Rightarrow$ let $\pi$ to be $\arg\max_{a} Q(s, a)$ \\ 
		i.e. simple deterministic policy as choosing the action with the maximal $q$ value
		\end{itemize}
	\item Training
		\begin{itemize}
		\item loss $L = (y-Q(s, a))^2$, where $y= R_{t+1}(s, a) + \gamma \max_{a'}[Q(s_\text{next}, a')]$ \\
		$\Rightarrow$ estimating the optimal Q-table using the current network (as a guess) \\
		$\Rightarrow$ iterative self-training by enforcing Bellman equation
		\item for more stable training, fix the estimator (a copy of net itself) until loss is small \\
		(then used the now better network as estimator of $y$)
		\item i.e. for each episode, loop over time
			\begin{itemize}
			\item with current state $s$, predict $Q(s, a)$ from Q-network
			\item take action $a$ with highest $Q(s,a)$
			\item observe reward $r$ and reach the next state $s$
			\item estimate $y$ by network predicting $Q(s', a)$, if $s'$ not terminal state \\
			(otherwise, let $y=r$ if terminal state reached)
			\item backprop \& update the network \\
			(break the episode if in terminal state)
			\end{itemize}
		\end{itemize}
	\item Experience Replay
		\begin{itemize}
		\item store experience $<s, a, r, s'>$ that network encounter
		\item for each step, sample an experience \& use network to continue from that state \\
		$\Rightarrow$ exploring new states (added to experience set)
		\item for each update, sample a batch from existing experiences \& update the network
		\item motivation
			\begin{itemize}
			\item re-train on past experience/exploration $\Rightarrow$ enable data reuse \\
			(as some state may never be visited again due to network update)
			\item avoid training bias i.e. gradient update biased to one action in a time period \\
			(leads to unstable training)
			$\Rightarrow$ de-correlate the successive training updates
			\item trading between exploration - available computation
			\end{itemize}
		\end{itemize}
	\item Epsilon Greedy Action
		\begin{itemize}
		\item set hyper-param $\epsilon$ and draw a random variable $r$ at each step
		$\Rightarrow$ take $a$ with highest $Q(s, a)$, if $r<\epsilon$ (being greedy); take random action otherwise %$\begin{cases}  \\ \end{cases}$
		\item motivation: balance the trade-off between exploration - exploitation
		\end{itemize}
	\item Challenges
		\begin{itemize}
		\item efficient exploration v.s. exploitation
		\item potential large state-action pairs
		\end{itemize}
	\end{itemize}
\end{itemize}

\subsubsection{Policy Gradient}
\begin{itemize}
\item Overview
	\begin{itemize}
	\item Notation
		\begin{itemize}
		\item a class of parameterized policies $\Pi = \{ \pi_\theta \}$, where $\theta$ the weights
		\item let $r(\tau) = (r_0,r_1,...)$, the reward sequence of trajectory $\tau=([s_0,a_0], [s_1, a_1], ...)$
		\item policy value $J=\mathbb E [ G_t|\pi_\theta ] = \int_\tau r(\tau) p(\tau|\pi_\theta) \D \tau$
		\end{itemize}
	\item Inference
		\begin{itemize}
		\item sample a trajectory $\tau$ following the current policy $\pi_\theta$
		\end{itemize}
	\item Training
		\begin{itemize}
		\Item \begin{align*}\displaystyle \nabla_\theta J(\theta) &= \int_\tau r(\tau) \nabla_\theta p(\tau|\pi_\theta) \D\tau \\
		&= \int_\tau r(\tau) p(r|\pi_\theta) \nabla_\theta \log p(r|\pi_\theta)  \D\tau, \text{ as } \D y = y\frac{\D y}{y} = y\D (\log y) \\
		&= \mathbb E_{\tau\sim p(\tau|\pi_\theta)} \left[ r(\tau) \nabla_\theta p(\tau|\pi_\theta) \right]
		\end{align*}
		\item inference to sample trajectories $\Rightarrow$ multiple $\tau$ to estimate the expectation
		\item backprop as a batch of size $\sum_{n=1}^{N} T_n$, \\
		where $N$ the sampling times, $T_n$ the total step in each sampled $\tau_n$ \\
		\includegraphics[width=0.7\linewidth, center]{"./Reinforcement Learning/plot/intro - policy gradient structure".png}
		\item simple strategy: if $r(\tau)$ high, recognize all actions as good \& increase their prob \\
		(similarly, decrease prop if $r(\tau)$ low) \\
		$\Rightarrow$ use expectation to averages out \\
		$\Rightarrow$ suffer from high variation \& large demands on sampling
		\end{itemize}
	\item Variance Reduction
		\begin{itemize}
		\item use an indicator for each action \\
		(instead of using $r(\tau)$ over entire sequence to guide the sequence) \\
		$\Rightarrow$ measured against expected reward $Q(s,a)$ for each $(s,a)$
		$\Rightarrow$ Actor-Critic Algorithm
		\end{itemize}
	\item Challenge
		\begin{itemize}
		\item sample efficiency
		\item credit assignment over the actions (alleviate by using a better indicator)
		\end{itemize}
	\end{itemize}
\item Actor-Critic Algorithm
	\begin{itemize}
	\item Motivation
		\begin{itemize}
		\item combine policy gradient and Q-learning
		\item use policy gradient to directly optimize policy (actor)
		\item use Q-learning to be a more accurate indicator for $(s,a)$ pair (critic)
		\end{itemize}
	\item Notation
		\begin{itemize}
		\item $A_\pi(s,a) = Q_\pi(s,a)-v(s)$ the advantage function of action $a$ in state $s$
		\end{itemize}
	\item Training
		\begin{itemize}
		\item optimize $Q$ guided by Bellman equation
		\item optimize the policy with $A(s,a)$ as indicator for each $(s,a)$ in a trajectory $\tau$
		\end{itemize}
	\end{itemize}
\end{itemize}


\subsubsection{Temporal Difference Learning}
\begin{itemize}
\item 
%	\begin{itemize}
%	\item
%	\end{itemize}
% \item 
%	\begin{itemize}
%	\item
%	\end{itemize}
% \item 
%	\begin{itemize}
%	\item
%	\end{itemize}
\end{itemize}

\subsection{Multi-Agent Learning}
\subsubsection{Overview}
\begin{itemize}
\item Multi-Agent System
	\begin{itemize}
	\item Multiple Agents
		\begin{itemize}
		\item asynchronous computation
		\item internal knowledge
		\item characterstics (control hierarchy, computability etc.)
		\end{itemize}
	\item Envirnment
		\begin{itemize}
		\item constraints
		\item feedback
		\end{itemize}
	\item Interaction
		\begin{itemize}
		\item between agents
		\item between agents and environment
		\end{itemize}
	\item Learning Types
		\begin{itemize}
		\item team learning\item a single learner discovering behaviours for the team
			\begin{itemize}
			\item homogeneous\item all agents assigned identical behaviour
			\item heterogeneous\item agents assigned differently, with speciality concerned
			\item hybrid\item hierarchy of sub-teams
			\item pros: focus on team outcome, easily adapted from single agent
			\item cons: ignore inter-agent credit assignment \& interaction, combinatorial explosion in state space
			\end{itemize}
		\item concurrent learning\item multiple learners
			\begin{itemize}
			\item different credit assignment methods
			\item approaches to dynamic environment and co-adaption
			\item ways to model others (interaction \& collabration between agents)
			\item pros: reduction of search space (disjoint behviours in agents), more computability
			\end{itemize}
		\end{itemize}
	\end{itemize}

\item Credit Assignment
	\begin{itemize}
	\item Global Reward
		\begin{itemize}
		\item split the team reward equally (same reward for everyone) \\
		$\Rightarrow$ no sufficient feedback for agent-specific behaviour \\
		\item also, global reward can somtimes be hard to compute
		\end{itemize}
	\item Local Reward
		\begin{itemize}
		\item solely based on individual behaviour  $\Rightarrow$ discourages laziness \\
		(not necessarily better than global reward)
		\item yet, greedy behaviour is encouraged
		\end{itemize}
	\item Social Reinforcement
		\begin{itemize}
		\item observational reinforcement: obtained by observing and imitating other agents' behaviour
		\item vicarious reinforcement: received when target agents are directly rewards 
		\item $\Rightarrow$ spread rare behaviour to others \& balance between global and local rewards
		\end{itemize}
	\item Contribution Reward
		\begin{itemize}
		\item agents assume observed reward is the sum of each agent's direct contribution
		$\Rightarrow$ apportion reward based on the agent's contribution to the global reward
		\item or, apportion reward based on how team would be disadvantaged had the agent never existed
		\item claim to be better than both global \& local rewards (balance between them)
		\end{itemize}
	\item Non-disconted Reward Averaged over Time
		\begin{itemize}
		\item averaging rewards of a sequence of tasks over time
		\item encourage collaboration \\
		(relief the problem of not enough rewards for agents not directly achieving the goal)
		\end{itemize}
	\end{itemize}

\item Dynamic Environment
	\begin{itemize}
	\item Fully Cooperative Scenario
		\begin{itemize}
		\item feature
			\begin{itemize}
			\item global reward with equal divvy
			\item increase one agent's reward $\Leftrightarrow$ increase others reaward \\
			parato optimum agrees with Nash equilibria
			\end{itemize}
		\item deterministic repeated games
			\begin{itemize}
			\item update by estimating the likely best cooperation possible for agent's action
			\item update with stochastic sampling with agreed protocol for exploration \& exploitation
			\end{itemize}
		\item stochastic (repeated) games
			\begin{itemize}
			\item optimal adaptive learning algorithm \\
			(guaranteed convergence to optimal Nash equilibria under finite actions and states)
			\end{itemize}
		\item partially observable markov decision process \\
		(stochastic game with constraint on agents' ability to observe the state of environment)
		\end{itemize}
	\item General Sum Game
		\begin{itemize}
		\item feature
			\begin{itemize}
			\item a game that has unequal-share credit assignment \& is not zero-sum \\
			$\Rightarrow$ parato optimum may not agree with Nash equilibria
			\item may inadvertently cause non-cooperative scenarios
			\end{itemize}
		\item stochastic games \\
			\begin{itemize}
			\item learn not only its table of Q-values but also table for other agents
			$\Rightarrow$ other tables used to approximate actions from other
			\item directly approximate others policies, instead of Q-values
			\item EXORL algorithm: learn optimal response policies to both adaptive and fixed policies
			\end{itemize}
		\end{itemize}
	\item Competitive Learning
		\begin{itemize}
		\item feature
			\begin{itemize}
			\item need to not only cooperate but also compete e.g. predator-prey game \\
			$\Rightarrow$ desirable to learn at similar pace
			\item loss of gradient problem: always win/lose may occur
			$\Rightarrow$ insufficient feed back for both sides
			\item red-queen effect: result of competition can be meaningless 
			$\Rightarrow$ agents not learning, but degenerate, at different paces \\
			(cyclic behaviours: comptetitors exchange behaviours, with no actual gain)
			\end{itemize}
		\end{itemize}
	\end{itemize}
\item Teammate Modeling
	\begin{itemize}
	\item Goal
		\begin{itemize}
		\item learn other agents in the environment $\Rightarrow$ approximate their expected behaviours
		\item $\Rightarrow$ cooperate / response accordingly
		\end{itemize}
	\item Approches
		\begin{itemize}
		\item simple learner (Bayesian) and model for other agents
		\item learning the others belief, capability, preference etc. (statistical info)
		$\Rightarrow$ store a set of such model with their probability of being correct given the observation
		\item communication protocol to exchange model, or subcontract tasks \\
		(yet, need to model constrainted communication ability)
		\end{itemize}
	\end{itemize}
\item Communication in Learning
	\begin{itemize}
	\item Direct Communication
		\begin{itemize}
		\item share immediate sensor information \\
		i.e. inform its state to others, under assumption (prior knowledge)
		\item share past experience in form of episodes \\
		e.g. a sequence of <state, action, reward>
		\item share directly current policies
		\item can make communication learnable: communication as an action
		\end{itemize}
	\item Indirect Communication
		\begin{itemize}
		\item actions influence environment $\Rightarrow$ information observable by others
		\item imitating others 
		\end{itemize}
	\end{itemize}
\item Chanllenges
	\begin{itemize}
	\item Scalability
		\begin{itemize}
		\item currently multi-agents sys consists of 2~3 agents, limited actions (2~3) \\
		yet, we need dozens, hundreds...
		\end{itemize}
	\item Adaptive Dynamics and Nash Equilibria
		\begin{itemize}
		\item affected by credit assugnment scheme 
		\item usually converge to sub-optimal Nash equilibria
		\item ways to record level of learning level (ration level)
		\end{itemize}
	\item Problem Decomposition
		\begin{itemize}
		\item overwhelming state space
		\item some current approaches...
			\begin{itemize}
			\item domain knowledge $\Rightarrow$ smaller yet more powerful set of customised actions
			\item heuristically \& hierarchically decomposing the problem
			layered learning: learn basic behaviours, then compose more complex one based on them \\
			(bottom-up decomposition)
			\item shaping: gradually change reward function to favour those composed complex behaviours \\
			(soft layered learning)
			\item coordination graph: partially decompose joint Q-value based on coordination graph \\
			$\Rightarrow$ coordinate high-level behaivours \\
			(balance between full joint utility table and separate independent tables) 
			\end{itemize}
		\end{itemize}
	\end{itemize}
\item Problem Domains and Applications
	\begin{itemize}
	\item Emodied Agents
		\begin{itemize}
		\item cooperative navigation
		\item cooperative target observation
		\end{itemize}
	\item Game Theory
		\begin{itemize}
		\item social delimma
		\end{itemize}
	\item Real-Time Distributed Decision Making
		\begin{itemize}
		\item air traffic control
		\item supply chains (JSP)
		\item hierarchical multi-agent systems problems
		\item social interaction modeling
		\end{itemize}
	\end{itemize}
\end{itemize}




















